\section{Conclusion}

In this paper, we introduced the Maximum Matrix Contraction problem (MMC). 
We proved this problem is NP-Complete. However, we also proved that every algorithm which solve this problem is an $O(\sqrt{n})$-approximation algorithm. Considering that the NP-Completeness was derived from the Maximum Clique problem, and that this problem cannot be polynomially approximated to within $n^{\frac{1}{2}-\varepsilon}$, MMC is very likely to not being approximable to within the same ratio. Such a result would almost tight the approximability of MMC.

Moreover, we studied four algorithms to solve the problem, an integer linear program, a first-come-first-served algorithm and two greedy algorithms, and gave numerical results. It appears firstly that integer linear programming is not adapted to MMC while the three other heuristics returns really good quality solutions in short amount of time even for large instances. Those results seems to disconfirm the $n^{\frac{1}{2}-\varepsilon}$ inapproximability ratio. It would be interesting to deepen the study in order to produce a constant-factor polynomial approximation algorithm or a polynomial-time approximation scheme if such an algorithm exists.
