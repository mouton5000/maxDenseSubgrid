\section{Complexity}
\label{sect:complexity}
 
This section is dedicated to proving the NP-Completeness of the problem.

  \begin{theorem}
  \label{theo:complexity}
  The decision version of (MMC) is NP-Complete.
  \end{theorem}
  \begin{proof}
 

    Let $M$ be an instance of MMC. Given an integer $K$, a sublist $I$ of $\llbracket 1; p-1 \rrbracket$ and a sublist $J$ of $\llbracket 1; q-1 \rrbracket$, we can compute in polynomial time the matrix $C(M,I,J)$ and check if the contraction is valid and if $d(C(M,I,J)) \geq K$. This proves the problem belongs to NP.

    In order to prove the NP-Hardness, we describe a polynomial reduction from the NP-Complete Maximum Clique problem \cite{Karp1972}. Lets $G(V,E)$ be an instance of the Maximum Clique problem, we build an instance $M$ of MMC with $p = q = (4|V| + 6)$. We arbitrarily number the nodes of $G$ : $V = \{v_1, v_2, \dots v_{|V|}\}$.

    Let $l_i$ and $c_i$ be respectively the $6+4(i-1)+1$-th line and the $6+4(i-1)+1$-th column. We associate the four lines $l_i, l_i+1, l_i+2, l_i+3$ and the four columns $c_i, c_i +1, c_i+2, c_i + 3$ to $v_i$. The key idea of the reduction is that each node $v$ is associated with two 1 of the matrix. If we choose the node $v$ to be in the clique, then, firstly, the two 1 associated with $v$ are moved next to each other and this increases the density by one; and secondly, for every node $w$ such that $(v,w) \not\in E$, the two 1 associated to cannot be moved anymore.

A complete example is given in Figure~\ref{fig:reduction:example}. For each node $v_i$, we set $M_{l_i,c_i} = M_{l_i+2,c_i+2} = 1$. If the nodes $v_i$ and $v_j$ are not linked with an edge, we set $M_{l_i,c_j} = M_{l_i+1,c_j+1} = 1$. If, on the contrary, there is an edge $(v_i,v_j)$, then the intersections of the lines of $v_i$ and the column of $v_j$ is filled with 0. Finally, we add some 1 in the six first columns and the six first lines of the matrix such that only the contractions of the line $l_i$ and the column $c_i$ for $i \in \llbracket 1;n \rrbracket$ are valid.

%    We use three gadgets in this reduction, described in Figure~\ref{fig:reduction:gadgets}. The gadget $(\alpha_i)$ is added at the intersection of the lines and the columns associated with the node $v_i$. The gadget $(\beta_{i,j})$ is added at the intersections of the lines of $v_i$ and the columns of $v_j$ such that $(v_i,v_j) \not\in E$. If $v_i$ and $v_j$ are linked with an edge in $G$, we do not add the gadget and all the entries equal 0. We add the gadget $(\gamma_i)$ for each node $v_i$ at line 1, and a transposed gadget at column 1. Finally, we set $M_{1,j} = M_{j,1} = 1$ for each $j \in \llbracket 1;6 \rrbracket$. A complete example is given in Figure~\ref{fig:reduction:example}.

%\begin{figure}
\centering
		\begin{tikzpicture}
		\coordinate (O) at (0,0);

	\clip (-1,-1.25) rectangle (2.5,2.1);

    \prgrid{O}{4}{4}
    
    \prbul{O}{1}{1}
    \prbul{O}{3}{3}
    \draw ($(O)+(0.25,-0.3)$) node {$c_i$};
    \draw ($(O)+(1.75,-0.3)$) node {$c_i+3$};
    \draw ($(O)+(-0,0.25)$) node[anchor=east] {$l_i$};
    \draw ($(O)+(-0,0.75)$) node[anchor=east] {$l_i+1$};
    \draw ($(O)+(-0,1.25)$) node[anchor=east] {$l_i+2$};
    \draw ($(O)+(-0,1.75)$) node[anchor=east] {$l_i+3$};
    \draw ($(O)+(1,-1)$) node {$(\alpha_i)$};
    \end{tikzpicture}
		\begin{tikzpicture}
		\coordinate (O) at (0,0);
		\clip (-1,-1.25) rectangle (2.5,2.1);

    \prgrid{O}{4}{4}
    
    \prbul{O}{1}{1}
    \prbul{O}{2}{2}
    \draw ($(O)+(0.25,-0.3)$) node {$c_i$};
    \draw ($(O)+(1.75,-0.3)$) node {$c_i+3$};
    \draw ($(O)+(0,0.25)$) node[anchor=east] {$l_j$};
    \draw ($(O)+(0,0.75)$) node[anchor=east] {$l_j+1$};
    \draw ($(O)+(0,1.25)$) node[anchor=east] {$l_j+2$};
    \draw ($(O)+(0,1.75)$) node[anchor=east] {$l_j+3$};
    \draw ($(O)+(1,-1)$) node {$(\beta_{i,j})$};
    \end{tikzpicture}
		\begin{tikzpicture}
		\coordinate (O) at (0,0);
		\clip (-1,-1.25) rectangle (3,2.1);

    \prgrid{O}{4}{5}
    
    \prbul{O}{1}{1};
    \prbul{O}{2}{3};
    \prbul{O}{3}{3};
    \prbul{O}{3}{5};
    \prbul{O}{4}{5};
    \prbul{O}{4}{1};
    
    \draw ($(O)+(0.25,-0.3)$) node {$1$};
    \draw ($(O)+(0.75,-0.3)$) node {$2$};
    \draw ($(O)+(1.25,-0.3)$) node {$3$};
    \draw ($(O)+(1.75,-0.3)$) node {$4$};
    \draw ($(O)+(2.25,-0.3)$) node {$5$};
    \draw ($(O)+(-0,0.25)$) node[anchor=east] {$l_i$};
    \draw ($(O)+(-0,0.75)$) node[anchor=east] {$l_i+1$};
    \draw ($(O)+(-0,1.25)$) node[anchor=east] {$l_i+2$};
    \draw ($(O)+(-0,1.75)$) node[anchor=east] {$l_i+3$};
    \draw ($(O)+(1.25,-1)$) node {$(\gamma_i)$};
    \end{tikzpicture}

\caption{The three gadgets of the reduction. We specify on each gadget the indexes of its lines and columns.}
   \label{fig:reduction:gadgets}
\end{figure}
\begin{figure}
\centering
\scalebox{0.6}{
		\begin{tikzpicture}
		
		\clip (-0.25,-5.25) rectangle (3.25,7.25);
		\tikzset{tinoeud/.style={draw, circle, minimum height=0.5cm}}
		\node[tinoeud] (U) at (0,0) {};
		\node[tinoeud] (V) at (2,0) {};
		\node[tinoeud] (W) at (0,2) {};
		\node[tinoeud] (T) at (2,2) {};

    \draw (U) -- (V);
    \draw (V) -- (W);
    \draw (W) -- (U);
    \draw (W) -- (T);

    \draw (U) node {$1$};
    \draw (V) node {$2$};
    \draw (W) node {$3$};
    \draw (T) node {$4$};

		\end{tikzpicture}

		\begin{tikzpicture}
		
		\coordinate (O) at (0,0);

    \prgrid{O}{22}{22}

    \prvtline{O}{6}{22}
    \prvtline{O}{10}{22}
    \prvtline{O}{14}{22}
    \prvtline{O}{18}{22}
    
    \prvtcolumn{O}{6}{22}
    \prvtcolumn{O}{10}{22}
    \prvtcolumn{O}{14}{22}
    \prvtcolumn{O}{18}{22}
    
    \prvtdline{O}{7}{22}
    \prvtdline{O}{11}{22}
    \prvtdline{O}{15}{22}
    \prvtdline{O}{19}{22}
    
    \prvtdcolumn{O}{7}{22}
    \prvtdcolumn{O}{11}{22}
    \prvtdcolumn{O}{15}{22}
    \prvtdcolumn{O}{19}{22}
    
    \draw ($(O)+(-0.5,4)$) node {$v_1$};
    \draw ($(O)+(-0.5,6)$) node {$v_2$};
    \draw ($(O)+(-0.5,8)$) node {$v_3$};
    \draw ($(O)+(-0.5,10)$) node {$v_4$};
    
    \draw ($(O)+(4,-0.5)$) node {$v_1$};
    \draw ($(O)+(6,-0.5)$) node {$v_2$};
    \draw ($(O)+(8,-0.5)$) node {$v_3$};
    \draw ($(O)+(10,-0.5)$) node {$v_4$};

    \prone{O}{1}{1};
    \prone{O}{1}{2};
    \prone{O}{1}{3};
    \prone{O}{1}{4};
    \prone{O}{1}{5};
    \prone{O}{1}{6};
    
    \prone{O}{2}{1};
    \prone{O}{3}{1};
    \prone{O}{4}{1};
    \prone{O}{5}{1};
    \prone{O}{6}{1};
 
    \prone{O}{7}{1};
    \prone{O}{8}{3};
    \prone{O}{9}{3};
    \prone{O}{9}{5};
    \prone{O}{10}{5};
    \prone{O}{10}{1};
 
    \prone{O}{11}{1};
    \prone{O}{12}{3};
    \prone{O}{13}{3};
    \prone{O}{13}{5};
    \prone{O}{14}{5};
    \prone{O}{14}{1};
 
    \prone{O}{15}{1};
    \prone{O}{16}{3};
    \prone{O}{17}{3};
    \prone{O}{17}{5};
    \prone{O}{18}{5};
    \prone{O}{18}{1};
 
    \prone{O}{19}{1};
    \prone{O}{20}{3};
    \prone{O}{21}{3};
    \prone{O}{21}{5};
    \prone{O}{22}{5};
    \prone{O}{22}{1};
		
 
    \prone{O}{1}{7};
    \prone{O}{3}{8};
    \prone{O}{3}{9};
    \prone{O}{5}{9};
    \prone{O}{5}{10};
    \prone{O}{1}{10};
 
    \prone{O}{1}{11};
    \prone{O}{3}{12};
    \prone{O}{3}{13};
    \prone{O}{5}{13};
    \prone{O}{5}{14};
    \prone{O}{1}{14};
 
    \prone{O}{1}{15};
    \prone{O}{3}{16};
    \prone{O}{3}{17};
    \prone{O}{5}{17};
    \prone{O}{5}{18};
    \prone{O}{1}{18};
 
    \prone{O}{1}{19};
    \prone{O}{3}{20};
    \prone{O}{3}{21};
    \prone{O}{5}{21};
    \prone{O}{5}{22};
    \prone{O}{1}{22};


    \prone{O}{7}{7};
    \prone{O}{9}{9};

    \prone{O}{11}{11};
    \prone{O}{13}{13};

    \prone{O}{15}{15};
    \prone{O}{17}{17};

    \prone{O}{19}{19};
    \prone{O}{21}{21};


    \prone{O}{7}{19};
    \prone{O}{8}{20};

    \prone{O}{19}{7};
    \prone{O}{20}{8};

    \prone{O}{11}{19};
    \prone{O}{12}{20};

    \prone{O}{19}{11};
    \prone{O}{20}{12};

  
    \end{tikzpicture}}

\caption{This figure illustrates, on the left, a graph in which we search for a maximum clique and, on the right, the matrix obtained built with the reduction. We do not show the 0 entries of the matrix for readability. Each dotted line and column represent the valid contractions.}
\label{fig:reduction:example}
\end{figure}

The initial density in this matrix is $d_0 = 11 + 6|V| + (|V| (|V|-1) - 2|E|)$. Note that, in order to add one to the density of the matrix, the only way is to choose a node $v_i$ and contract the column $c_i$ and the line $l_i$. If the column $c_i$ is contracted and if $(v_i, v_j) \not \in E$, the two entries $M_{l_j,c_i}$ and $M_{l_j+1,c_i+1}$ are moved on the same column. Similarly, if the line $l_i$ is contracted, the two entries $M_{l_i,c_j}$ and $M_{l_i+1,c_j+1}$ are moved on the same line. This prohibits the contraction of the line $l_j$ and the column $c_j$. Consequently, in order to add $C$ to the density, we must find a clique of size $C$ in the graph and contract every line and column associated with the nodes of that clique.

Thus, there is a clique of size $K$ if and only if there is a feasible solution for $M$ of density $d_0 + K$. This concludes the proof of NP-Completeness.
\end{proof}
  
The Maximum Clique problem cannot be approximated to within $|V|^{\frac{1}{2} - \varepsilon}$ in polynomial time unless P = NP \cite{Hastad1999}. Unfortunately, the previous reduction cannot be used to prove a negative approximability result occurs for MMC. Indeed, the density of any feasible solution of the MMC instance we produce is between $d_0 + 1$ and $d_0 + |V|$, with $d_0 = O(|V|^2 - |E|)$. Consequently, the optimal density is at most $(1 + 1/|V|)$ times the worst density. A way to prove a higher inapproximability ratio for MMC would be to modify the reduction such that the gap between the optimal solution and another feasible solution increases.

In the next section, we prove that a $n^{\frac{1}{2} - \varepsilon}$ harness ratio would almost tight the approximability of MMC as there exists a $2\sqrt{n}$-approximation algorithm.
