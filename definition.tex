\section{Problem definition}

The following definitions formalize the problem we want to solve with binary matrices. A binary matrix is a matrix with entries from $\{0,1\}$. Such a matrix modelizes a grid in which each dot is a 1 in the matrix.

\begin{definition}
Let $M$ be a binary matrix with $p$ lines and $q$ columns. For each $i \in \llbracket 0; p-2 \rrbracket$ and each $j \in \llbracket 0; q-2 \rrbracket$, we define the \emph{line contraction matrix} $L_i$ and the \emph{column contraction matrix} $C_j$ by 

$$L_i = \hspace{0.2cm}\begin{blockarray}{ccccccccccccc}
& & & & & & & & & & &\\
\begin{block}{(cccccccccc)ccc}
1      &  0     & \cdots & 0      & 0 & 0 &  0  & 0 & \cdots & 0 &&&\\
0      &  1     & \cdots & 0      & 0 & 0 &  0  & 0 & \cdots & 0 &&&\\
\vdots & \vdots & \ddots & \vdots & \vdots & \vdots &  \vdots  & \vdots & \ddots  & \vdots  &&&\\
0      &   0    &\cdots  & 1      & 0 & 0 &  0  & 0 & \cdots & 0 &&&\\ 
0      &   0    &\cdots  & 0      & 1 & 1 &  0  & 0 & \cdots & 0 & && i\\ 
0      &   0    &\cdots  & 0      & 0 & 0 &  1  & 0 & \cdots & 0 &&&\\
0      &   0    &\cdots  & 0      & 0 & 0 &  0  & 1 & \cdots & 0 &&&\\ 
\vdots      &   \vdots    &\ddots  & \vdots     & \vdots & \vdots & \vdots & \vdots & \ddots & \vdots  &&&\\
0      &   0    &\cdots  & 0      & 0 & 0  &  0 & 0 & \cdots & 1 &&&\\
0      &   0    &\cdots  & 0      & 0 & 0      &   0    & 0      &\cdots  & 0 &&&\\
\end{block}
\end{blockarray}
\hspace{0.5cm}
C_j = \hspace{0.2cm}\begin{blockarray}{ccccccccccc}
 & & & & j & & & & &\\
\begin{block}{(cccccccccc)c}
1      &  0     & \cdots & 0      & 0 & 0 &  0  & \cdots & 0 & 0 &\\
0      &  1     & \cdots & 0      & 0 & 0 &  0  & \cdots & 0 & 0 &\\
\vdots & \vdots & \ddots & \vdots & \vdots & \vdots &  \vdots  & \ddots & \vdots  & \vdots  &\\
0      &   0    &\cdots  & 1      & 0 & 0 &  0  & \cdots & 0 & 0 &\\ 
0      &   0    &\cdots  & 0      & 1 & 0 &  0  & \cdots & 0 & 0 &\\ 
0      &   0    &\cdots  & 0      & 1 & 0 &  0  & \cdots & 0 & 0 &\\
0      &   0    &\cdots  & 0      & 0 & 1 &  0  & \cdots & 0 & 0 &\\ 
0      &   0    &\cdots  & 0      & 0 & 0  &  1 & \cdots & 0 & 0 &\\
\vdots      &   \vdots    &\ddots  & \vdots     & \vdots & \vdots & \vdots & \ddots & \vdots & \vdots  &\\
0      &   0    &\cdots  & 0      & 0 & 0      &   0    &\cdots  & 1      & 0 &\\
\end{block}
\end{blockarray}
$$
\end{definition}

\begin{definition}
Let $M$ be a binary matrix of size $p \times q$, $I = \{i_1, i_2, \dots, i_{|I|}\}$ a sublist of $\llbracket 0;p-1 \rrbracket$ and $J = \{j_1, j_2, \dots, j_{|I|}\}$ a sublist of $\llbracket 0;q-1 \rrbracket$. We assume $I$ and $J$ are sorted. We define the \emph{contraction $C(M,I,J)$ of the lines $I$ and the columns $J$ of $M$} by the following matrix
$$
C(M,I,J) = \left(\prod\limits_{k = 1}^{|I|} L_{i_k}\right) \cdot M \cdot \left(\prod\limits_{k = |J|}^{1} C_{j_k}\right)
$$
\end{definition}

\begin{definition}
	A contraction $C(M,I,J)$ is said \emph{legal} if and only if $C(M,I,J)$ is a binary matrix.
\end{definition}

For example, the contraction of the $j$-th column of a matrix such that $M_{i,j} = M_{i,j+1} = 1$ is not legal because it contains a $2$ at line $i$ and column $j$.

\begin{definition}
	Let $M$ be a binary matrix of size $p \times q$. The \emph{density} is the number of neighbor pairs of 1 in the matrix (including the diagonal pairs). This value may be computed with the following formula :
	$$ 
	d(M) = \frac{1}{2} \cdot \sum\limits_{i,j} \left( M_{i,j} \cdot \left(\sum\limits_{\delta = -1}^1 \sum\limits_{\gamma = -1}^1  M_{i+\delta,j+\gamma}\right) - 1 \right)
	$$
\end{definition}

\begin{problem}
	\emph{Maximum Matrix Contraction problem} (MMC). Given a binary matrix $M$ of size $p \times q$ such that $n$ entries equal 1 and $p \cdot q - n$ entries equal 0, the Maximum Matrix Contraction problem consists in the search for two sublists $I$ of $\llbracket 0;p-1 \rrbracket$ and $J$ of $\llbracket 0;q-1 \rrbracket$ such that the contraction $C(M,I,J)$ is legal and maximize $d(C(M,I,J))$.
\end{problem}

\begin{comment}
Doit-on parler du programme mathématique trivial ici?
\end{comment}

\todo{Petite transition?}

