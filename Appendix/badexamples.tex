\section{An instance with an $O(\sqrt{n})$ gap between the worst and the best solution.}

\label{apx:badinstance}


Theorem~\ref{theo:sqrtnapprox} of Section~\ref{sect:approx} proves a default $2\sqrt{n}$ upper bound of the approximation ratio for every algorithm returning a maximal solution.

We give, in this appendix, in Figure~\ref{fig:badExample}, an instance in which the ratio between an optimal density and the lowest density of a maximal solution is $O(\sqrt{n})$. This instance contains a square of $r \times r$ blue ones (in Figure~\ref{fig:badExample}, $r = 4$) and two diagonal lines of $r$ red ones and $r$ green ones. Note that, as $r$ approaches infinity, $r^2$ is asymptotically equivalent to $n$. An optimal solution, given in Figure~\ref{fig:badexampleGood}, consists in maximally contracting the blue square. In that case, most of the blue ones have 8 neighbors (the maximum number of neighbor an entry can have in the matrix). The density of this matrix is $O(r^2) = O(n)$. On the contrary, an algorithm which contracts the lines containing the green ones and the columns containing the red ones would obtain the matrix in Figure~\ref{fig:badExampleBad}. In this matrix, no line or column can be contracted and most of the blue ones have no neighbor. The density is $O(r) = O(\sqrt{n})$. 


\begin{figure}
	
		\begin{tikzpicture}
		
		\coordinate (O) at (0,0);

    \prgrid{O}{17}{17}

    \prbul{O}{1}{9};
    \prbul{O}{2}{10};
    \prbul{O}{3}{11};
    \prbul{O}{4}{12};
    \prbul{O}{5}{13};
    \prbul{O}{6}{14};
    \prbul{O}{7}{15};
    \prbul{O}{8}{16};
    \prbul{O}{9}{17};
    
    \prbul{O}{17}{1};
    \prbul{O}{16}{2};
    \prbul{O}{15}{3};
    \prbul{O}{14}{4};
    \prbul{O}{13}{5};
    \prbul{O}{12}{6};
    \prbul{O}{11}{7};
    \prbul{O}{10}{8};
    
    
    \prbul{O}{11}{11};
    \prbul{O}{13}{11};
    \prbul{O}{15}{11};
    \prbul{O}{17}{11};

    \prbul{O}{11}{13};
    \prbul{O}{13}{13};
    \prbul{O}{15}{13};
    \prbul{O}{17}{13};
    
    \prbul{O}{11}{15};
    \prbul{O}{13}{15};
    \prbul{O}{15}{15};
    \prbul{O}{17}{15};
  
    \prbul{O}{11}{17};
    \prbul{O}{13}{17};
    \prbul{O}{15}{17};
    \prbul{O}{17}{17};
    \end{tikzpicture}
		\begin{tikzpicture}
		
		\coordinate (O) at (0,0);

    \prgrid{O}{9}{9}

    \prbul{O}{1}{1};
    \prbul{O}{2}{1};
    \prbul{O}{3}{1};
    \prbul{O}{4}{1};
    \prbul{O}{5}{1};
    \prbul{O}{6}{1};
    \prbul{O}{7}{1};
    \prbul{O}{8}{1};
    \prbul{O}{9}{1};
    
    \prbul{O}{1}{2};
    \prbul{O}{1}{3};
    \prbul{O}{1}{4};
    \prbul{O}{1}{5};
    \prbul{O}{1}{6};
    \prbul{O}{1}{7};
    \prbul{O}{1}{8};
    \prbul{O}{1}{9};

    \prbul{O}{3}{3};
    \prbul{O}{5}{3};
    \prbul{O}{7}{3};
    \prbul{O}{9}{3};
    
    \prbul{O}{3}{5};
    \prbul{O}{5}{5};
    \prbul{O}{7}{5};
    \prbul{O}{9}{5};
  
    \prbul{O}{3}{7};
    \prbul{O}{5}{7};
    \prbul{O}{7}{7};
    \prbul{O}{9}{7};

    \prbul{O}{3}{9};
    \prbul{O}{5}{9};
    \prbul{O}{7}{9};
    \prbul{O}{9}{9};
    \end{tikzpicture}
		\begin{tikzpicture}
		
		\coordinate (O) at (0,0);

    \prgrid{O}{9}{9}

    \prbul{O}{1}{5};
    \prbul{O}{1}{6};
    \prbul{O}{2}{6};
    \prbul{O}{2}{7};
    \prbul{O}{3}{7};
    \prbul{O}{3}{8};
    \prbul{O}{4}{8};
    \prbul{O}{4}{9};
    \prbul{O}{5}{9};
    
    \prbul{O}{9}{1};
    \prbul{O}{9}{2};
    \prbul{O}{8}{2};
    \prbul{O}{8}{3};
    \prbul{O}{7}{3};
    \prbul{O}{7}{4};
    \prbul{O}{6}{4};
    \prbul{O}{6}{5};
    
    \prbul{O}{6}{6};
    \prbul{O}{7}{6};
    \prbul{O}{8}{6};
    \prbul{O}{9}{6};
    
    \prbul{O}{6}{7};
    \prbul{O}{7}{7};
    \prbul{O}{8}{7};
    \prbul{O}{9}{7};

    \prbul{O}{6}{8};
    \prbul{O}{7}{8};
    \prbul{O}{8}{8};
    \prbul{O}{9}{8};

    \prbul{O}{6}{9};
    \prbul{O}{7}{9};
    \prbul{O}{8}{9};
    \prbul{O}{9}{9};
    \end{tikzpicture}
\caption{In this instance, the density of an optimal solution is $O(n)$ and there is a maximal solution of density $O(\sqrt{n})$.}

\label{fig:badinstance}
  \end{figure}
  
  \begin{comment}
  Il faudra probablement mettre cette figure en annexe.
  \end{comment}
\begin{figure}
    \centering

    \begin{tikzpicture}
        \coordinate (O) at (0,0);
        \prgrid{O}{9}{9}

        \prone{O}{1}{1};
        \prone{O}{1}{2};
        \prone{O}{1}{3};
        \prone{O}{1}{4};
        \prone{O}{1}{5};
        \prone{O}{1}{6};
        \prone{O}{1}{8};
        \prone{O}{1}{9};

        \prone{O}{2}{1};
        \prone{O}{2}{2};
        \prone{O}{2}{3};
        \prone{O}{2}{4};
        \prone{O}{2}{7};
        \prone{O}{2}{8};

        \prone{O}{3}{1};
        \prone{O}{3}{2};
        \prone{O}{3}{3};
        \prone{O}{3}{4};
        \prone{O}{3}{6};
        \prone{O}{3}{7};

        \prone{O}{4}{1};
        \prone{O}{4}{2};
        \prone{O}{4}{3};
        \prone{O}{4}{4};
        \prone{O}{4}{6};

        \prone{O}{5}{1};

        \prone{O}{6}{1};
        \prone{O}{6}{3};
        \prone{O}{6}{4};

        \prone{O}{7}{2};
        \prone{O}{7}{3};

        \prone{O}{8}{1};
        \prone{O}{8}{2};

        \prone{O}{9}{1};


    \end{tikzpicture}
    \caption{An optimal solution of the instance given in Figure~\ref{fig:badExample}. The density of this solution is $O(n^2)$.}
    \label{fig:badexampleGood}
\end{figure}

\begin{figure}
    \centering

 \begin{tikzpicture}
 \coordinate (O) at (0,0);
 \prgrid{O}{10}{10}
 
 \definecolor{tempcolor}{rgb}{0.0,0.0,1.0}
 \pronec{O}{1}{1}{tempcolor};
 \definecolor{tempcolor}{rgb}{0.0,0.0,1.0}
 \pronec{O}{1}{3}{tempcolor};
 \definecolor{tempcolor}{rgb}{0.0,0.0,1.0}
 \pronec{O}{1}{5}{tempcolor};
 \definecolor{tempcolor}{rgb}{0.0,0.0,1.0}
 \pronec{O}{1}{7}{tempcolor};
 \definecolor{tempcolor}{rgb}{0.0,0.0,0.0}
 \pronec{O}{1}{8}{tempcolor};
 \definecolor{tempcolor}{rgb}{0.0,0.0,0.0}
 \pronec{O}{1}{9}{tempcolor};
 \definecolor{tempcolor}{rgb}{1.0,0.0,0.0}
 \pronec{O}{1}{10}{tempcolor};
 
 \definecolor{tempcolor}{rgb}{1.0,0.0,0.0}
 \pronec{O}{2}{10}{tempcolor};
 
 \definecolor{tempcolor}{rgb}{0.0,0.0,1.0}
 \pronec{O}{3}{1}{tempcolor};
 \definecolor{tempcolor}{rgb}{0.0,0.0,1.0}
 \pronec{O}{3}{3}{tempcolor};
 \definecolor{tempcolor}{rgb}{0.0,0.0,1.0}
 \pronec{O}{3}{5}{tempcolor};
 \definecolor{tempcolor}{rgb}{0.0,0.0,1.0}
 \pronec{O}{3}{7}{tempcolor};
 \definecolor{tempcolor}{rgb}{1.0,0.0,0.0}
 \pronec{O}{3}{10}{tempcolor};
 
 \definecolor{tempcolor}{rgb}{1.0,0.0,0.0}
 \pronec{O}{4}{10}{tempcolor};
 
 \definecolor{tempcolor}{rgb}{0.0,0.0,1.0}
 \pronec{O}{5}{1}{tempcolor};
 \definecolor{tempcolor}{rgb}{0.0,0.0,1.0}
 \pronec{O}{5}{3}{tempcolor};
 \definecolor{tempcolor}{rgb}{0.0,0.0,1.0}
 \pronec{O}{5}{5}{tempcolor};
 \definecolor{tempcolor}{rgb}{0.0,0.0,1.0}
 \pronec{O}{5}{7}{tempcolor};
 \definecolor{tempcolor}{rgb}{1.0,0.0,0.0}
 \pronec{O}{5}{10}{tempcolor};
 
 \definecolor{tempcolor}{rgb}{1.0,0.0,0.0}
 \pronec{O}{6}{10}{tempcolor};
 
 \definecolor{tempcolor}{rgb}{0.0,0.0,1.0}
 \pronec{O}{7}{1}{tempcolor};
 \definecolor{tempcolor}{rgb}{0.0,0.0,1.0}
 \pronec{O}{7}{3}{tempcolor};
 \definecolor{tempcolor}{rgb}{0.0,0.0,1.0}
 \pronec{O}{7}{5}{tempcolor};
 \definecolor{tempcolor}{rgb}{0.0,0.0,1.0}
 \pronec{O}{7}{7}{tempcolor};
 \definecolor{tempcolor}{rgb}{1.0,0.0,0.0}
 \pronec{O}{7}{10}{tempcolor};
 
 \definecolor{tempcolor}{rgb}{0.0,0.0,0.0}
 \pronec{O}{8}{1}{tempcolor};
 
 \definecolor{tempcolor}{rgb}{0.0,0.0,0.0}
 \pronec{O}{9}{1}{tempcolor};
 
 \definecolor{tempcolor}{rgb}{1.0,0.0,0.0}
 \pronec{O}{10}{1}{tempcolor};
 \definecolor{tempcolor}{rgb}{1.0,0.0,0.0}
 \pronec{O}{10}{2}{tempcolor};
 \definecolor{tempcolor}{rgb}{1.0,0.0,0.0}
 \pronec{O}{10}{3}{tempcolor};
 \definecolor{tempcolor}{rgb}{1.0,0.0,0.0}
 \pronec{O}{10}{4}{tempcolor};
 \definecolor{tempcolor}{rgb}{1.0,0.0,0.0}
 \pronec{O}{10}{5}{tempcolor};
 \definecolor{tempcolor}{rgb}{1.0,0.0,0.0}
 \pronec{O}{10}{6}{tempcolor};
 \definecolor{tempcolor}{rgb}{1.0,0.0,0.0}
 \pronec{O}{10}{7}{tempcolor};
 
 \draw ($(O)+(0,2.25)$) node[anchor=east] {$5$};
 \draw ($(O)+(0,4.75)$) node[anchor=east] {$10$};
 \draw ($(O)+(2.25,-0.3)$) node {$5$};
 \draw ($(O)+(4.75,-0.3)$) node {$10$};
 
 \end{tikzpicture}
    \caption{A solution of the instance given in Figure~\ref{fig:badExample} for which the density is $O(n)$.}
    \label{fig:badExampleBad}
\end{figure}




Note that this is not a hard instance for the greedy algorithm, the neighborization algorithm or the LCL algorithm to this instance as the density of the resulting matrix of each algorithm is $O(n)$. Indeed, the greedy algorithm and the neighborization contract the blue square first. The LCL algorithm either contracts all the lines of the matrix or contracts all the columns: every blue one has at least one neighbor and the density is $O(r^2)$. However, for each algorithm, we can adapt this instance such that the density returned by this algorithm is $O(\sqrt{n})$. Each of the following subsections is dedicated to an algorithm and its adapted instance. 

\subsection{Adaptation to the greedy algorithm}

In Figure~\ref{fig:greedyExample}, we give an instance adapted from the instance of Figure~\ref{fig:badExample}. This instance contains, from coordinates (1,1) to (11,11), is the square of blue 1. An optimal solution is obtained by fully contracting this square. To do so, we have to contract lines and columns 2, 4, 6, 8 and 10 (the blue lines and columns). This is illustrated with Figure~\ref{fig:greedyExampleGood}. The worst maximal solution is obtained by not contracting it. This instance contains also 4 groups of ones in the last 6 columns and the last 6 lines (the green 1). We can increase the density of the matrix by contracting lines 13 and 16 and by contracting columns 13 and 16 (the green lines and columns). However, due to the red 1, it is not allowed to contract all the green lines and columns \textbf{and} all the blue lines and columns. A consequence is that we can either contract all the green 1 or contract all the red 1 (or a part of each). Finally, note that it in not allowed to contract any other line or column.

What columns or lines does the greedy algorithm contract in this instance? The algorithm search for the line or the column such that the contraction of that line or column maximally increases the density of the matrix. It appears that contracting any green line or column increases the density by 10 and that contracting any blue line or column increased the density by 7. Thus, the greedy algorithm chooses one of the green lines or columns and starts again. The same phenomenon occurs in the resulting matrix : an increase of 7 with the blue lines and columns and an increase of 10 with the left green lines and columns. The algorithm then always choose a green line or column until the solution is maximal. The returned solution is given in Figure~\ref{fig:greedyExampleBad}.



\renewcommand{\gridsize}{0.5}
\begin{figure}
    \centering

    \begin{tikzpicture}
        \coordinate (O) at (0,0);
        \prgrid{O}{27}{27}
        
        \definecolor{tempcolor}{rgb}{0.0,0.0,1.0}
        \pronec{O}{1}{1}{tempcolor};
        \definecolor{tempcolor}{rgb}{0.0,0.0,1.0}
        \pronec{O}{1}{3}{tempcolor};
        \definecolor{tempcolor}{rgb}{0.0,0.0,1.0}
        \pronec{O}{1}{5}{tempcolor};
        \definecolor{tempcolor}{rgb}{0.0,0.0,1.0}
        \pronec{O}{1}{7}{tempcolor};
        \definecolor{tempcolor}{rgb}{0.0,0.0,1.0}
        \pronec{O}{1}{9}{tempcolor};
        \definecolor{tempcolor}{rgb}{0.0,0.0,1.0}
        \pronec{O}{1}{11}{tempcolor};
        \definecolor{tempcolor}{rgb}{1.0,0.0,0.0}
        \pronec{O}{1}{14}{tempcolor};
        
        \definecolor{tempcolor}{rgb}{1.0,0.0,0.0}
        \pronec{O}{2}{14}{tempcolor};
        
        \definecolor{tempcolor}{rgb}{0.0,0.0,1.0}
        \pronec{O}{3}{1}{tempcolor};
        \definecolor{tempcolor}{rgb}{0.0,0.0,1.0}
        \pronec{O}{3}{3}{tempcolor};
        \definecolor{tempcolor}{rgb}{0.0,0.0,1.0}
        \pronec{O}{3}{5}{tempcolor};
        \definecolor{tempcolor}{rgb}{0.0,0.0,1.0}
        \pronec{O}{3}{7}{tempcolor};
        \definecolor{tempcolor}{rgb}{0.0,0.0,1.0}
        \pronec{O}{3}{9}{tempcolor};
        \definecolor{tempcolor}{rgb}{0.0,0.0,1.0}
        \pronec{O}{3}{11}{tempcolor};
        \definecolor{tempcolor}{rgb}{1.0,0.0,0.0}
        \pronec{O}{3}{13}{tempcolor};
        \definecolor{tempcolor}{rgb}{1.0,0.0,0.0}
        \pronec{O}{3}{18}{tempcolor};
        
        \definecolor{tempcolor}{rgb}{1.0,0.0,0.0}
        \pronec{O}{4}{18}{tempcolor};
        
        \definecolor{tempcolor}{rgb}{0.0,0.0,1.0}
        \pronec{O}{5}{1}{tempcolor};
        \definecolor{tempcolor}{rgb}{0.0,0.0,1.0}
        \pronec{O}{5}{3}{tempcolor};
        \definecolor{tempcolor}{rgb}{0.0,0.0,1.0}
        \pronec{O}{5}{5}{tempcolor};
        \definecolor{tempcolor}{rgb}{0.0,0.0,1.0}
        \pronec{O}{5}{7}{tempcolor};
        \definecolor{tempcolor}{rgb}{0.0,0.0,1.0}
        \pronec{O}{5}{9}{tempcolor};
        \definecolor{tempcolor}{rgb}{0.0,0.0,1.0}
        \pronec{O}{5}{11}{tempcolor};
        \definecolor{tempcolor}{rgb}{1.0,0.0,0.0}
        \pronec{O}{5}{14}{tempcolor};
        \definecolor{tempcolor}{rgb}{1.0,0.0,0.0}
        \pronec{O}{5}{17}{tempcolor};
        
        \definecolor{tempcolor}{rgb}{1.0,0.0,0.0}
        \pronec{O}{6}{14}{tempcolor};
        
        \definecolor{tempcolor}{rgb}{0.0,0.0,1.0}
        \pronec{O}{7}{1}{tempcolor};
        \definecolor{tempcolor}{rgb}{0.0,0.0,1.0}
        \pronec{O}{7}{3}{tempcolor};
        \definecolor{tempcolor}{rgb}{0.0,0.0,1.0}
        \pronec{O}{7}{5}{tempcolor};
        \definecolor{tempcolor}{rgb}{0.0,0.0,1.0}
        \pronec{O}{7}{7}{tempcolor};
        \definecolor{tempcolor}{rgb}{0.0,0.0,1.0}
        \pronec{O}{7}{9}{tempcolor};
        \definecolor{tempcolor}{rgb}{0.0,0.0,1.0}
        \pronec{O}{7}{11}{tempcolor};
        \definecolor{tempcolor}{rgb}{1.0,0.0,0.0}
        \pronec{O}{7}{13}{tempcolor};
        \definecolor{tempcolor}{rgb}{1.0,0.0,0.0}
        \pronec{O}{7}{18}{tempcolor};
        
        \definecolor{tempcolor}{rgb}{1.0,0.0,0.0}
        \pronec{O}{8}{18}{tempcolor};
        
        \definecolor{tempcolor}{rgb}{0.0,0.0,1.0}
        \pronec{O}{9}{1}{tempcolor};
        \definecolor{tempcolor}{rgb}{0.0,0.0,1.0}
        \pronec{O}{9}{3}{tempcolor};
        \definecolor{tempcolor}{rgb}{0.0,0.0,1.0}
        \pronec{O}{9}{5}{tempcolor};
        \definecolor{tempcolor}{rgb}{0.0,0.0,1.0}
        \pronec{O}{9}{7}{tempcolor};
        \definecolor{tempcolor}{rgb}{0.0,0.0,1.0}
        \pronec{O}{9}{9}{tempcolor};
        \definecolor{tempcolor}{rgb}{0.0,0.0,1.0}
        \pronec{O}{9}{11}{tempcolor};
        \definecolor{tempcolor}{rgb}{1.0,0.0,0.0}
        \pronec{O}{9}{14}{tempcolor};
        \definecolor{tempcolor}{rgb}{1.0,0.0,0.0}
        \pronec{O}{9}{17}{tempcolor};
        
        \definecolor{tempcolor}{rgb}{1.0,0.0,0.0}
        \pronec{O}{10}{14}{tempcolor};
        
        \definecolor{tempcolor}{rgb}{0.0,0.0,1.0}
        \pronec{O}{11}{1}{tempcolor};
        \definecolor{tempcolor}{rgb}{0.0,0.0,1.0}
        \pronec{O}{11}{3}{tempcolor};
        \definecolor{tempcolor}{rgb}{0.0,0.0,1.0}
        \pronec{O}{11}{5}{tempcolor};
        \definecolor{tempcolor}{rgb}{0.0,0.0,1.0}
        \pronec{O}{11}{7}{tempcolor};
        \definecolor{tempcolor}{rgb}{0.0,0.0,1.0}
        \pronec{O}{11}{9}{tempcolor};
        \definecolor{tempcolor}{rgb}{0.0,0.0,1.0}
        \pronec{O}{11}{11}{tempcolor};
        \definecolor{tempcolor}{rgb}{1.0,0.0,0.0}
        \pronec{O}{11}{13}{tempcolor};
        \definecolor{tempcolor}{rgb}{0.0,0.0,0.0}        \pronec{O}{11}{22}{tempcolor};
        
        \definecolor{tempcolor}{rgb}{0.0,0.0,0.0}        \pronec{O}{12}{22}{tempcolor};
        
        \definecolor{tempcolor}{rgb}{1.0,0.0,0.0}
        \pronec{O}{13}{3}{tempcolor};
        \definecolor{tempcolor}{rgb}{1.0,0.0,0.0}
        \pronec{O}{13}{7}{tempcolor};
        \definecolor{tempcolor}{rgb}{1.0,0.0,0.0}
        \pronec{O}{13}{11}{tempcolor};
        \definecolor{tempcolor}{rgb}{0.0,0.3921568691730499,0.0}
        \pronec{O}{13}{22}{tempcolor};
        \definecolor{tempcolor}{rgb}{0.0,0.3921568691730499,0.0}
        \pronec{O}{13}{23}{tempcolor};
        \definecolor{tempcolor}{rgb}{0.0,0.3921568691730499,0.0}
        \pronec{O}{13}{24}{tempcolor};
        \definecolor{tempcolor}{rgb}{0.0,0.3921568691730499,0.0}
        \pronec{O}{13}{25}{tempcolor};
        \definecolor{tempcolor}{rgb}{0.0,0.3921568691730499,0.0}
        \pronec{O}{13}{26}{tempcolor};
        
        \definecolor{tempcolor}{rgb}{1.0,0.0,0.0}
        \pronec{O}{14}{1}{tempcolor};
        \definecolor{tempcolor}{rgb}{1.0,0.0,0.0}
        \pronec{O}{14}{2}{tempcolor};
        \definecolor{tempcolor}{rgb}{1.0,0.0,0.0}
        \pronec{O}{14}{5}{tempcolor};
        \definecolor{tempcolor}{rgb}{1.0,0.0,0.0}
        \pronec{O}{14}{6}{tempcolor};
        \definecolor{tempcolor}{rgb}{1.0,0.0,0.0}
        \pronec{O}{14}{9}{tempcolor};
        \definecolor{tempcolor}{rgb}{1.0,0.0,0.0}
        \pronec{O}{14}{10}{tempcolor};
        \definecolor{tempcolor}{rgb}{0.0,0.3921568691730499,0.0}
        \pronec{O}{14}{27}{tempcolor};
        
        \definecolor{tempcolor}{rgb}{0.0,0.3921568691730499,0.0}
        \pronec{O}{15}{22}{tempcolor};
        \definecolor{tempcolor}{rgb}{0.0,0.3921568691730499,0.0}
        \pronec{O}{15}{23}{tempcolor};
        \definecolor{tempcolor}{rgb}{0.0,0.3921568691730499,0.0}
        \pronec{O}{15}{24}{tempcolor};
        \definecolor{tempcolor}{rgb}{0.0,0.3921568691730499,0.0}
        \pronec{O}{15}{25}{tempcolor};
        \definecolor{tempcolor}{rgb}{0.0,0.3921568691730499,0.0}
        \pronec{O}{15}{26}{tempcolor};
        \definecolor{tempcolor}{rgb}{0.0,0.3921568691730499,0.0}
        \pronec{O}{15}{27}{tempcolor};
        
        \definecolor{tempcolor}{rgb}{0.0,0.0,0.0}        \pronec{O}{16}{22}{tempcolor};
        
        \definecolor{tempcolor}{rgb}{1.0,0.0,0.0}
        \pronec{O}{17}{5}{tempcolor};
        \definecolor{tempcolor}{rgb}{1.0,0.0,0.0}
        \pronec{O}{17}{9}{tempcolor};
        \definecolor{tempcolor}{rgb}{0.0,0.3921568691730499,0.0}
        \pronec{O}{17}{22}{tempcolor};
        \definecolor{tempcolor}{rgb}{0.0,0.3921568691730499,0.0}
        \pronec{O}{17}{23}{tempcolor};
        \definecolor{tempcolor}{rgb}{0.0,0.3921568691730499,0.0}
        \pronec{O}{17}{24}{tempcolor};
        \definecolor{tempcolor}{rgb}{0.0,0.3921568691730499,0.0}
        \pronec{O}{17}{25}{tempcolor};
        \definecolor{tempcolor}{rgb}{0.0,0.3921568691730499,0.0}
        \pronec{O}{17}{26}{tempcolor};
        
        \definecolor{tempcolor}{rgb}{1.0,0.0,0.0}
        \pronec{O}{18}{3}{tempcolor};
        \definecolor{tempcolor}{rgb}{1.0,0.0,0.0}
        \pronec{O}{18}{4}{tempcolor};
        \definecolor{tempcolor}{rgb}{1.0,0.0,0.0}
        \pronec{O}{18}{7}{tempcolor};
        \definecolor{tempcolor}{rgb}{1.0,0.0,0.0}
        \pronec{O}{18}{8}{tempcolor};
        \definecolor{tempcolor}{rgb}{0.0,0.3921568691730499,0.0}
        \pronec{O}{18}{27}{tempcolor};
        
        \definecolor{tempcolor}{rgb}{0.0,0.3921568691730499,0.0}
        \pronec{O}{19}{22}{tempcolor};
        \definecolor{tempcolor}{rgb}{0.0,0.3921568691730499,0.0}
        \pronec{O}{19}{23}{tempcolor};
        \definecolor{tempcolor}{rgb}{0.0,0.3921568691730499,0.0}
        \pronec{O}{19}{24}{tempcolor};
        \definecolor{tempcolor}{rgb}{0.0,0.3921568691730499,0.0}
        \pronec{O}{19}{25}{tempcolor};
        \definecolor{tempcolor}{rgb}{0.0,0.3921568691730499,0.0}
        \pronec{O}{19}{26}{tempcolor};
        \definecolor{tempcolor}{rgb}{0.0,0.3921568691730499,0.0}
        \pronec{O}{19}{27}{tempcolor};
        
        \definecolor{tempcolor}{rgb}{0.0,0.0,0.0}        \pronec{O}{20}{22}{tempcolor};
        
        \definecolor{tempcolor}{rgb}{0.0,0.0,0.0}        \pronec{O}{21}{22}{tempcolor};
        
        \definecolor{tempcolor}{rgb}{0.0,0.0,0.0}
        \pronec{O}{22}{11}{tempcolor};
        \definecolor{tempcolor}{rgb}{0.0,0.0,0.0}
        \pronec{O}{22}{12}{tempcolor};
        \definecolor{tempcolor}{rgb}{0.0,0.3921568691730499,0.0}
        \pronec{O}{22}{13}{tempcolor};
        \definecolor{tempcolor}{rgb}{0.0,0.3921568691730499,0.0}
        \pronec{O}{22}{15}{tempcolor};
        \definecolor{tempcolor}{rgb}{0.0,0.0,0.0}
        \pronec{O}{22}{16}{tempcolor};
        \definecolor{tempcolor}{rgb}{0.0,0.3921568691730499,0.0}
        \pronec{O}{22}{17}{tempcolor};
        \definecolor{tempcolor}{rgb}{0.0,0.3921568691730499,0.0}
        \pronec{O}{22}{19}{tempcolor};
        \definecolor{tempcolor}{rgb}{0.0,0.0,0.0}        \pronec{O}{22}{20}{tempcolor};        \definecolor{tempcolor}{rgb}{0.0,0.0,0.0}
        \pronec{O}{22}{21}{tempcolor}; \definecolor{tempcolor}{rgb}{0.0,0.0,0.0}
        \pronec{O}{22}{22}{tempcolor};
        
        \definecolor{tempcolor}{rgb}{0.0,0.3921568691730499,0.0}
        \pronec{O}{23}{13}{tempcolor};
        \definecolor{tempcolor}{rgb}{0.0,0.3921568691730499,0.0}
        \pronec{O}{23}{15}{tempcolor};
        \definecolor{tempcolor}{rgb}{0.0,0.3921568691730499,0.0}
        \pronec{O}{23}{17}{tempcolor};
        \definecolor{tempcolor}{rgb}{0.0,0.3921568691730499,0.0}
        \pronec{O}{23}{19}{tempcolor};
        
        \definecolor{tempcolor}{rgb}{0.0,0.3921568691730499,0.0}
        \pronec{O}{24}{13}{tempcolor};
        \definecolor{tempcolor}{rgb}{0.0,0.3921568691730499,0.0}
        \pronec{O}{24}{15}{tempcolor};
        \definecolor{tempcolor}{rgb}{0.0,0.3921568691730499,0.0}
        \pronec{O}{24}{17}{tempcolor};
        \definecolor{tempcolor}{rgb}{0.0,0.3921568691730499,0.0}
        \pronec{O}{24}{19}{tempcolor};
        
        \definecolor{tempcolor}{rgb}{0.0,0.3921568691730499,0.0}
        \pronec{O}{25}{13}{tempcolor};
        \definecolor{tempcolor}{rgb}{0.0,0.3921568691730499,0.0}
        \pronec{O}{25}{15}{tempcolor};
        \definecolor{tempcolor}{rgb}{0.0,0.3921568691730499,0.0}
        \pronec{O}{25}{17}{tempcolor};
        \definecolor{tempcolor}{rgb}{0.0,0.3921568691730499,0.0}
        \pronec{O}{25}{19}{tempcolor};
        
        \definecolor{tempcolor}{rgb}{0.0,0.3921568691730499,0.0}
        \pronec{O}{26}{13}{tempcolor};
        \definecolor{tempcolor}{rgb}{0.0,0.3921568691730499,0.0}
        \pronec{O}{26}{15}{tempcolor};
        \definecolor{tempcolor}{rgb}{0.0,0.3921568691730499,0.0}
        \pronec{O}{26}{17}{tempcolor};
        \definecolor{tempcolor}{rgb}{0.0,0.3921568691730499,0.0}
        \pronec{O}{26}{19}{tempcolor};
        
        \definecolor{tempcolor}{rgb}{0.0,0.3921568691730499,0.0}
        \pronec{O}{27}{14}{tempcolor};
        \definecolor{tempcolor}{rgb}{0.0,0.3921568691730499,0.0}
        \pronec{O}{27}{15}{tempcolor};
        \definecolor{tempcolor}{rgb}{0.0,0.3921568691730499,0.0}
        \pronec{O}{27}{18}{tempcolor};
        \definecolor{tempcolor}{rgb}{0.0,0.3921568691730499,0.0}
        \pronec{O}{27}{19}{tempcolor};
        
        \draw ($(O)+(0,2.25)$) node[anchor=east] {$5$};
        \draw ($(O)+(0,4.75)$) node[anchor=east] {$10$};
        \draw ($(O)+(0,7.25)$) node[anchor=east] {$15$};
        \draw ($(O)+(0,9.75)$) node[anchor=east] {$20$};
        \draw ($(O)+(0,12.25)$) node[anchor=east] {$25$};
        \draw ($(O)+(2.25,-0.3)$) node {$5$};
        \draw ($(O)+(4.75,-0.3)$) node {$10$};
        \draw ($(O)+(7.25,-0.3)$) node {$15$};
        \draw ($(O)+(9.75,-0.3)$) node {$20$};
        \draw ($(O)+(12.25,-0.3)$) node {$25$};
        
        \definecolor{tempcolor}{rgb}{0.0,0.0,1.0}
        \prvtdlinec{O}{2}{27}{tempcolor};
        \prvtdlinec{O}{4}{27}{tempcolor};
        \prvtdlinec{O}{6}{27}{tempcolor};
        \prvtdlinec{O}{8}{27}{tempcolor};
        \prvtdlinec{O}{10}{27}{tempcolor};
        
        \definecolor{tempcolor}{rgb}{0.0,0.3921568691730499,0.0}
        \prvtdlinec{O}{13}{27}{tempcolor};
        \prvtdlinec{O}{17}{27}{tempcolor};
        
        
        \definecolor{tempcolor}{rgb}{0.0,0.0,1.0}
        \prvtdcolumnc{O}{2}{27}{tempcolor};
        \prvtdcolumnc{O}{4}{27}{tempcolor};
        \prvtdcolumnc{O}{6}{27}{tempcolor};
        \prvtdcolumnc{O}{8}{27}{tempcolor};
        \prvtdcolumnc{O}{10}{27}{tempcolor};
        
        \prvtrectc{O}{0}{0}{11}{11}{tempcolor};
        
        \definecolor{tempcolor}{rgb}{0.0,0.3921568691730499,0.0}
        \prvtdcolumnc{O}{13}{27}{tempcolor};
        \prvtdcolumnc{O}{17}{27}{tempcolor};
        \prvtrectc{O}{10}{21}{21}{27}{tempcolor};
        \prvtrectc{O}{21}{10}{27}{21}{tempcolor};

    \end{tikzpicture}
    \caption{An example of instance for which an optimal density is $O(\sqrt{n})$ times the density of a solution returned by the Greedy algorithm.}
    \label{fig:greedyExample}
\end{figure}

\begin{figure}[ht!]
    \centering

    \begin{tikzpicture}
        \coordinate (O) at (0,0);
        \prgrid{O}{21}{20}

        \prone{O}{1}{1};
        \prone{O}{1}{2};
        \prone{O}{1}{3};
        \prone{O}{1}{4};
        \prone{O}{1}{5};
        \prone{O}{1}{6};
        \prone{O}{1}{9};

        \prone{O}{2}{1};
        \prone{O}{2}{2};
        \prone{O}{2}{3};
        \prone{O}{2}{4};
        \prone{O}{2}{5};
        \prone{O}{2}{6};
        \prone{O}{2}{8};
        \prone{O}{2}{9};
        \prone{O}{2}{12};

        \prone{O}{3}{1};
        \prone{O}{3}{2};
        \prone{O}{3}{3};
        \prone{O}{3}{4};
        \prone{O}{3}{5};
        \prone{O}{3}{6};
        \prone{O}{3}{9};
        \prone{O}{3}{11};
        \prone{O}{3}{12};

        \prone{O}{4}{1};
        \prone{O}{4}{2};
        \prone{O}{4}{3};
        \prone{O}{4}{4};
        \prone{O}{4}{5};
        \prone{O}{4}{6};
        \prone{O}{4}{8};
        \prone{O}{4}{9};
        \prone{O}{4}{12};

        \prone{O}{5}{1};
        \prone{O}{5}{2};
        \prone{O}{5}{3};
        \prone{O}{5}{4};
        \prone{O}{5}{5};
        \prone{O}{5}{6};
        \prone{O}{5}{9};
        \prone{O}{5}{11};
        \prone{O}{5}{12};

        \prone{O}{6}{1};
        \prone{O}{6}{2};
        \prone{O}{6}{3};
        \prone{O}{6}{4};
        \prone{O}{6}{5};
        \prone{O}{6}{6};
        \prone{O}{6}{8};
        \prone{O}{6}{9};
        \prone{O}{6}{15};

        \prone{O}{7}{15};

        \prone{O}{8}{2};
        \prone{O}{8}{4};
        \prone{O}{8}{6};
        \prone{O}{8}{15};
        \prone{O}{8}{16};
        \prone{O}{8}{17};
        \prone{O}{8}{18};
        \prone{O}{8}{19};

        \prone{O}{9}{1};
        \prone{O}{9}{2};
        \prone{O}{9}{3};
        \prone{O}{9}{4};
        \prone{O}{9}{5};
        \prone{O}{9}{6};
        \prone{O}{9}{20};

        \prone{O}{10}{15};
        \prone{O}{10}{16};
        \prone{O}{10}{17};
        \prone{O}{10}{18};
        \prone{O}{10}{19};
        \prone{O}{10}{20};

        \prone{O}{11}{3};
        \prone{O}{11}{5};
        \prone{O}{11}{15};
        \prone{O}{11}{16};
        \prone{O}{11}{17};
        \prone{O}{11}{18};
        \prone{O}{11}{19};

        \prone{O}{12}{2};
        \prone{O}{12}{3};
        \prone{O}{12}{4};
        \prone{O}{12}{5};
        \prone{O}{12}{20};

        \prone{O}{13}{15};
        \prone{O}{13}{16};
        \prone{O}{13}{17};
        \prone{O}{13}{18};
        \prone{O}{13}{19};
        \prone{O}{13}{20};

        \prone{O}{14}{15};

        \prone{O}{15}{15};

        \prone{O}{16}{6};
        \prone{O}{16}{7};
        \prone{O}{16}{8};
        \prone{O}{16}{10};
        \prone{O}{16}{11};
        \prone{O}{16}{13};
        \prone{O}{16}{14};
        \prone{O}{16}{15};

        \prone{O}{17}{8};
        \prone{O}{17}{10};
        \prone{O}{17}{11};
        \prone{O}{17}{13};

        \prone{O}{18}{8};
        \prone{O}{18}{10};
        \prone{O}{18}{11};
        \prone{O}{18}{13};

        \prone{O}{19}{8};
        \prone{O}{19}{10};
        \prone{O}{19}{11};
        \prone{O}{19}{13};

        \prone{O}{20}{8};
        \prone{O}{20}{10};
        \prone{O}{20}{11};
        \prone{O}{20}{13};

        \prone{O}{21}{9};
        \prone{O}{21}{10};
        \prone{O}{21}{12};
        \prone{O}{21}{13};


    \end{tikzpicture}
    \caption{An optimal solution of the instance given in Figure~\ref{fig:greedyExample}. The density of this solution is $O(n)$.}
    \label{fig:greedyExampleGood}
\end{figure}

\begin{figure}
    \centering

    \begin{tikzpicture}
        \coordinate (O) at (0,0);
        \prgrid{O}{25}{25}
        
        \definecolor{tempcolor}{rgb}{0.0,0.0,1.0}
        \pronec{O}{1}{1}{tempcolor};
        \definecolor{tempcolor}{rgb}{0.0,0.0,1.0}
        \pronec{O}{1}{3}{tempcolor};
        \definecolor{tempcolor}{rgb}{0.0,0.0,1.0}
        \pronec{O}{1}{5}{tempcolor};
        \definecolor{tempcolor}{rgb}{0.0,0.0,1.0}
        \pronec{O}{1}{7}{tempcolor};
        \definecolor{tempcolor}{rgb}{0.0,0.0,1.0}
        \pronec{O}{1}{9}{tempcolor};
        \definecolor{tempcolor}{rgb}{0.0,0.0,1.0}
        \pronec{O}{1}{11}{tempcolor};
        \definecolor{tempcolor}{rgb}{1.0,0.0,0.0}
        \pronec{O}{1}{13}{tempcolor};
        
        \definecolor{tempcolor}{rgb}{1.0,0.0,0.0}
        \pronec{O}{2}{13}{tempcolor};
        
        \definecolor{tempcolor}{rgb}{0.0,0.0,1.0}
        \pronec{O}{3}{1}{tempcolor};
        \definecolor{tempcolor}{rgb}{0.0,0.0,1.0}
        \pronec{O}{3}{3}{tempcolor};
        \definecolor{tempcolor}{rgb}{0.0,0.0,1.0}
        \pronec{O}{3}{5}{tempcolor};
        \definecolor{tempcolor}{rgb}{0.0,0.0,1.0}
        \pronec{O}{3}{7}{tempcolor};
        \definecolor{tempcolor}{rgb}{0.0,0.0,1.0}
        \pronec{O}{3}{9}{tempcolor};
        \definecolor{tempcolor}{rgb}{0.0,0.0,1.0}
        \pronec{O}{3}{11}{tempcolor};
        \definecolor{tempcolor}{rgb}{1.0,0.0,0.0}
        \pronec{O}{3}{13}{tempcolor};
        \definecolor{tempcolor}{rgb}{1.0,0.0,0.0}
        \pronec{O}{3}{16}{tempcolor};
        
        \definecolor{tempcolor}{rgb}{1.0,0.0,0.0}
        \pronec{O}{4}{16}{tempcolor};
        
        \definecolor{tempcolor}{rgb}{0.0,0.0,1.0}
        \pronec{O}{5}{1}{tempcolor};
        \definecolor{tempcolor}{rgb}{0.0,0.0,1.0}
        \pronec{O}{5}{3}{tempcolor};
        \definecolor{tempcolor}{rgb}{0.0,0.0,1.0}
        \pronec{O}{5}{5}{tempcolor};
        \definecolor{tempcolor}{rgb}{0.0,0.0,1.0}
        \pronec{O}{5}{7}{tempcolor};
        \definecolor{tempcolor}{rgb}{0.0,0.0,1.0}
        \pronec{O}{5}{9}{tempcolor};
        \definecolor{tempcolor}{rgb}{0.0,0.0,1.0}
        \pronec{O}{5}{11}{tempcolor};
        \definecolor{tempcolor}{rgb}{1.0,0.0,0.0}
        \pronec{O}{5}{13}{tempcolor};
        \definecolor{tempcolor}{rgb}{1.0,0.0,0.0}
        \pronec{O}{5}{16}{tempcolor};
        
        \definecolor{tempcolor}{rgb}{1.0,0.0,0.0}
        \pronec{O}{6}{13}{tempcolor};
        
        \definecolor{tempcolor}{rgb}{0.0,0.0,1.0}
        \pronec{O}{7}{1}{tempcolor};
        \definecolor{tempcolor}{rgb}{0.0,0.0,1.0}
        \pronec{O}{7}{3}{tempcolor};
        \definecolor{tempcolor}{rgb}{0.0,0.0,1.0}
        \pronec{O}{7}{5}{tempcolor};
        \definecolor{tempcolor}{rgb}{0.0,0.0,1.0}
        \pronec{O}{7}{7}{tempcolor};
        \definecolor{tempcolor}{rgb}{0.0,0.0,1.0}
        \pronec{O}{7}{9}{tempcolor};
        \definecolor{tempcolor}{rgb}{0.0,0.0,1.0}
        \pronec{O}{7}{11}{tempcolor};
        \definecolor{tempcolor}{rgb}{1.0,0.0,0.0}
        \pronec{O}{7}{13}{tempcolor};
        \definecolor{tempcolor}{rgb}{1.0,0.0,0.0}
        \pronec{O}{7}{16}{tempcolor};
        
        \definecolor{tempcolor}{rgb}{1.0,0.0,0.0}
        \pronec{O}{8}{16}{tempcolor};
        
        \definecolor{tempcolor}{rgb}{0.0,0.0,1.0}
        \pronec{O}{9}{1}{tempcolor};
        \definecolor{tempcolor}{rgb}{0.0,0.0,1.0}
        \pronec{O}{9}{3}{tempcolor};
        \definecolor{tempcolor}{rgb}{0.0,0.0,1.0}
        \pronec{O}{9}{5}{tempcolor};
        \definecolor{tempcolor}{rgb}{0.0,0.0,1.0}
        \pronec{O}{9}{7}{tempcolor};
        \definecolor{tempcolor}{rgb}{0.0,0.0,1.0}
        \pronec{O}{9}{9}{tempcolor};
        \definecolor{tempcolor}{rgb}{0.0,0.0,1.0}
        \pronec{O}{9}{11}{tempcolor};
        \definecolor{tempcolor}{rgb}{1.0,0.0,0.0}
        \pronec{O}{9}{13}{tempcolor};
        \definecolor{tempcolor}{rgb}{1.0,0.0,0.0}
        \pronec{O}{9}{16}{tempcolor};
        
        \definecolor{tempcolor}{rgb}{1.0,0.0,0.0}
        \pronec{O}{10}{13}{tempcolor};
        
        \definecolor{tempcolor}{rgb}{0.0,0.0,1.0}
        \pronec{O}{11}{1}{tempcolor};
        \definecolor{tempcolor}{rgb}{0.0,0.0,1.0}
        \pronec{O}{11}{3}{tempcolor};
        \definecolor{tempcolor}{rgb}{0.0,0.0,1.0}
        \pronec{O}{11}{5}{tempcolor};
        \definecolor{tempcolor}{rgb}{0.0,0.0,1.0}
        \pronec{O}{11}{7}{tempcolor};
        \definecolor{tempcolor}{rgb}{0.0,0.0,1.0}
        \pronec{O}{11}{9}{tempcolor};
        \definecolor{tempcolor}{rgb}{0.0,0.0,1.0}
        \pronec{O}{11}{11}{tempcolor};
        \definecolor{tempcolor}{rgb}{1.0,0.0,0.0}
        \pronec{O}{11}{13}{tempcolor};
        \definecolor{tempcolor}{rgb}{0.0,0.0,0.0}
        \pronec{O}{11}{20}{tempcolor};
        
        \definecolor{tempcolor}{rgb}{0.0,0.0,0.0}
        \pronec{O}{12}{20}{tempcolor};
        
        \definecolor{tempcolor}{rgb}{1.0,0.0,0.0}
        \pronec{O}{13}{1}{tempcolor};
        \definecolor{tempcolor}{rgb}{1.0,0.0,0.0}
        \pronec{O}{13}{2}{tempcolor};
        \definecolor{tempcolor}{rgb}{1.0,0.0,0.0}
        \pronec{O}{13}{3}{tempcolor};
        \definecolor{tempcolor}{rgb}{1.0,0.0,0.0}
        \pronec{O}{13}{5}{tempcolor};
        \definecolor{tempcolor}{rgb}{1.0,0.0,0.0}
        \pronec{O}{13}{6}{tempcolor};
        \definecolor{tempcolor}{rgb}{1.0,0.0,0.0}
        \pronec{O}{13}{7}{tempcolor};
        \definecolor{tempcolor}{rgb}{1.0,0.0,0.0}
        \pronec{O}{13}{9}{tempcolor};
        \definecolor{tempcolor}{rgb}{1.0,0.0,0.0}
        \pronec{O}{13}{10}{tempcolor};
        \definecolor{tempcolor}{rgb}{1.0,0.0,0.0}
        \pronec{O}{13}{11}{tempcolor};
        \definecolor{tempcolor}{rgb}{0.0,0.3921568691730499,0.0}
        \pronec{O}{13}{20}{tempcolor};
        \definecolor{tempcolor}{rgb}{0.0,0.3921568691730499,0.0}
        \pronec{O}{13}{21}{tempcolor};
        \definecolor{tempcolor}{rgb}{0.0,0.3921568691730499,0.0}
        \pronec{O}{13}{22}{tempcolor};
        \definecolor{tempcolor}{rgb}{0.0,0.3921568691730499,0.0}
        \pronec{O}{13}{23}{tempcolor};
        \definecolor{tempcolor}{rgb}{0.0,0.3921568691730499,0.0}
        \pronec{O}{13}{24}{tempcolor};
        \definecolor{tempcolor}{rgb}{0.0,0.3921568691730499,0.0}
        \pronec{O}{13}{25}{tempcolor};
        
        \definecolor{tempcolor}{rgb}{0.0,0.3921568691730499,0.0}
        \pronec{O}{14}{20}{tempcolor};
        \definecolor{tempcolor}{rgb}{0.0,0.3921568691730499,0.0}
        \pronec{O}{14}{21}{tempcolor};
        \definecolor{tempcolor}{rgb}{0.0,0.3921568691730499,0.0}
        \pronec{O}{14}{22}{tempcolor};
        \definecolor{tempcolor}{rgb}{0.0,0.3921568691730499,0.0}
        \pronec{O}{14}{23}{tempcolor};
        \definecolor{tempcolor}{rgb}{0.0,0.3921568691730499,0.0}
        \pronec{O}{14}{24}{tempcolor};
        \definecolor{tempcolor}{rgb}{0.0,0.3921568691730499,0.0}
        \pronec{O}{14}{25}{tempcolor};
        
        \definecolor{tempcolor}{rgb}{0.0,0.0,0.0}
        \pronec{O}{15}{20}{tempcolor};
        
        \definecolor{tempcolor}{rgb}{1.0,0.0,0.0}
        \pronec{O}{16}{3}{tempcolor};
        \definecolor{tempcolor}{rgb}{1.0,0.0,0.0}
        \pronec{O}{16}{4}{tempcolor};
        \definecolor{tempcolor}{rgb}{1.0,0.0,0.0}
        \pronec{O}{16}{5}{tempcolor};
        \definecolor{tempcolor}{rgb}{1.0,0.0,0.0}
        \pronec{O}{16}{7}{tempcolor};
        \definecolor{tempcolor}{rgb}{1.0,0.0,0.0}
        \pronec{O}{16}{8}{tempcolor};
        \definecolor{tempcolor}{rgb}{1.0,0.0,0.0}
        \pronec{O}{16}{9}{tempcolor};
        \definecolor{tempcolor}{rgb}{0.0,0.3921568691730499,0.0}
        \pronec{O}{16}{20}{tempcolor};
        \definecolor{tempcolor}{rgb}{0.0,0.3921568691730499,0.0}
        \pronec{O}{16}{21}{tempcolor};
        \definecolor{tempcolor}{rgb}{0.0,0.3921568691730499,0.0}
        \pronec{O}{16}{22}{tempcolor};
        \definecolor{tempcolor}{rgb}{0.0,0.3921568691730499,0.0}
        \pronec{O}{16}{23}{tempcolor};
        \definecolor{tempcolor}{rgb}{0.0,0.3921568691730499,0.0}
        \pronec{O}{16}{24}{tempcolor};
        \definecolor{tempcolor}{rgb}{0.0,0.3921568691730499,0.0}
        \pronec{O}{16}{25}{tempcolor};
        
        \definecolor{tempcolor}{rgb}{0.0,0.3921568691730499,0.0}
        \pronec{O}{17}{20}{tempcolor};
        \definecolor{tempcolor}{rgb}{0.0,0.3921568691730499,0.0}
        \pronec{O}{17}{21}{tempcolor};
        \definecolor{tempcolor}{rgb}{0.0,0.3921568691730499,0.0}
        \pronec{O}{17}{22}{tempcolor};
        \definecolor{tempcolor}{rgb}{0.0,0.3921568691730499,0.0}
        \pronec{O}{17}{23}{tempcolor};
        \definecolor{tempcolor}{rgb}{0.0,0.3921568691730499,0.0}
        \pronec{O}{17}{24}{tempcolor};
        \definecolor{tempcolor}{rgb}{0.0,0.3921568691730499,0.0}
        \pronec{O}{17}{25}{tempcolor};
        
        \definecolor{tempcolor}{rgb}{0.0,0.0,0.0}
        \pronec{O}{18}{20}{tempcolor};
        
        \definecolor{tempcolor}{rgb}{0.0,0.0,0.0}
        \pronec{O}{19}{20}{tempcolor};
        
        \definecolor{tempcolor}{rgb}{0.0,0.0,0.0}
        \pronec{O}{20}{11}{tempcolor};
        \definecolor{tempcolor}{rgb}{0.0,0.0,0.0}
        \pronec{O}{20}{12}{tempcolor};
        \definecolor{tempcolor}{rgb}{0.0,0.3921568691730499,0.0}
        \pronec{O}{20}{13}{tempcolor};
        \definecolor{tempcolor}{rgb}{0.0,0.3921568691730499,0.0}
        \pronec{O}{20}{14}{tempcolor};
        \definecolor{tempcolor}{rgb}{0.0,0.0,0.0}
        \pronec{O}{20}{15}{tempcolor};
        \definecolor{tempcolor}{rgb}{0.0,0.3921568691730499,0.0}
        \pronec{O}{20}{16}{tempcolor};
        \definecolor{tempcolor}{rgb}{0.0,0.3921568691730499,0.0}
        \pronec{O}{20}{17}{tempcolor};
        \definecolor{tempcolor}{rgb}{0.0,0.0,0.0}
        \pronec{O}{20}{18}{tempcolor};
        \definecolor{tempcolor}{rgb}{0.0,0.0,0.0}
        \pronec{O}{20}{19}{tempcolor};
        \definecolor{tempcolor}{rgb}{0.0,0.0,0.0}
        \pronec{O}{20}{20}{tempcolor};
        
        \definecolor{tempcolor}{rgb}{0.0,0.3921568691730499,0.0}
        \pronec{O}{21}{13}{tempcolor};
        \definecolor{tempcolor}{rgb}{0.0,0.3921568691730499,0.0}
        \pronec{O}{21}{14}{tempcolor};
        \definecolor{tempcolor}{rgb}{0.0,0.3921568691730499,0.0}
        \pronec{O}{21}{16}{tempcolor};
        \definecolor{tempcolor}{rgb}{0.0,0.3921568691730499,0.0}
        \pronec{O}{21}{17}{tempcolor};
        
        \definecolor{tempcolor}{rgb}{0.0,0.3921568691730499,0.0}
        \pronec{O}{22}{13}{tempcolor};
        \definecolor{tempcolor}{rgb}{0.0,0.3921568691730499,0.0}
        \pronec{O}{22}{14}{tempcolor};
        \definecolor{tempcolor}{rgb}{0.0,0.3921568691730499,0.0}
        \pronec{O}{22}{16}{tempcolor};
        \definecolor{tempcolor}{rgb}{0.0,0.3921568691730499,0.0}
        \pronec{O}{22}{17}{tempcolor};
        
        \definecolor{tempcolor}{rgb}{0.0,0.3921568691730499,0.0}
        \pronec{O}{23}{13}{tempcolor};
        \definecolor{tempcolor}{rgb}{0.0,0.3921568691730499,0.0}
        \pronec{O}{23}{14}{tempcolor};
        \definecolor{tempcolor}{rgb}{0.0,0.3921568691730499,0.0}
        \pronec{O}{23}{16}{tempcolor};
        \definecolor{tempcolor}{rgb}{0.0,0.3921568691730499,0.0}
        \pronec{O}{23}{17}{tempcolor};
        
        \definecolor{tempcolor}{rgb}{0.0,0.3921568691730499,0.0}
        \pronec{O}{24}{13}{tempcolor};
        \definecolor{tempcolor}{rgb}{0.0,0.3921568691730499,0.0}
        \pronec{O}{24}{14}{tempcolor};
        \definecolor{tempcolor}{rgb}{0.0,0.3921568691730499,0.0}
        \pronec{O}{24}{16}{tempcolor};
        \definecolor{tempcolor}{rgb}{0.0,0.3921568691730499,0.0}
        \pronec{O}{24}{17}{tempcolor};
        
        \definecolor{tempcolor}{rgb}{0.0,0.3921568691730499,0.0}
        \pronec{O}{25}{13}{tempcolor};
        \definecolor{tempcolor}{rgb}{0.0,0.3921568691730499,0.0}
        \pronec{O}{25}{14}{tempcolor};
        \definecolor{tempcolor}{rgb}{0.0,0.3921568691730499,0.0}
        \pronec{O}{25}{16}{tempcolor};
        \definecolor{tempcolor}{rgb}{0.0,0.3921568691730499,0.0}
        \pronec{O}{25}{17}{tempcolor};
        
        \draw ($(O)+(0,2.25)$) node[anchor=east] {$5$};
        \draw ($(O)+(0,4.75)$) node[anchor=east] {$10$};
        \draw ($(O)+(0,7.25)$) node[anchor=east] {$15$};
        \draw ($(O)+(0,9.75)$) node[anchor=east] {$20$};
        \draw ($(O)+(0,12.25)$) node[anchor=east] {$25$};
        \draw ($(O)+(2.25,-0.3)$) node {$5$};
        \draw ($(O)+(4.75,-0.3)$) node {$10$};
        \draw ($(O)+(7.25,-0.3)$) node {$15$};
        \draw ($(O)+(9.75,-0.3)$) node {$20$};
        \draw ($(O)+(12.25,-0.3)$) node {$25$};
    
    \end{tikzpicture}
	   \caption{A solution of the instance given in Figure~\ref{fig:greedyExample} that is returned by the greedy algorithm. The density of this solution is $O(\sqrt{n})$.}
	   \label{fig:greedyExampleBad}
\end{figure}


\subsection{Adaptation to the neighborization algorithm}

In Figure~\ref{fig:neighborizationExample}, we give an instance adapted from the instance of Figure~\ref{fig:badExample}. This instance contains, from coordinates (1,1) to (11,11), is the square of blue 1. An optimal solution is obtained by fully contracting this square. To do so, we have to contract lines and columns 2, 4, 6, 8 and 10 (the blue lines and columns). This is illustrated with Figure~\ref{fig:neighborizationExampleGood}. The worst maximal solution is obtained by not contracting it. This instance contains also 4 groups of ones in the last 6 columns and the last 6 lines (the green 1). We can increase the density of the matrix by contracting lines 13,14,15 and 18,19 and by contracting columns 13,14,15 and 18,19 (the green lines and columns). However, due to the red 1, it is not allowed to contract all the green lines and columns \textbf{and} all the blue lines and columns. A consequence is that we can either contract all the green 1 or contract all the red 1 (or a part of each). Finally, note that it in not allowed to contract any other line or column.

What columns or lines does the neighborization algorithm contract in this instance? The algorithm search for the line or the column such that the contraction of that line or column minimally decreases the number of pairs of 1 that can become neighbors. It appears that contracting any green line or column decreases this number by 17 or 18. On the other hand, contracting a blue line or column decreases this number by 20. Thus, the neighborization algorithm chooses one of the green lines or columns and starts again. The same phenomenon occurs in the resulting matrix : an decrease of 20 with the blue lines and columns and an increase of 17 or 18 with the left green lines and columns. The algorithm then always choose a green line or column until the solution is maximal. The returned solution is given in Figure~\ref{fig:neighborizationExampleBad}.


\renewcommand{\gridsize}{0.5}
\begin{figure}[ht!]
    \centering

    \begin{tikzpicture}
        \coordinate (O) at (0,0);
        \prgrid{O}{30}{30}

        \prone{O}{1}{1};
        \prone{O}{1}{3};
        \prone{O}{1}{5};
        \prone{O}{1}{7};
        \prone{O}{1}{9};
        \prone{O}{1}{11};
        \prone{O}{1}{14};

        \prone{O}{2}{14};

        \prone{O}{3}{1};
        \prone{O}{3}{3};
        \prone{O}{3}{5};
        \prone{O}{3}{7};
        \prone{O}{3}{9};
        \prone{O}{3}{11};
        \prone{O}{3}{13};
        \prone{O}{3}{19};

        \prone{O}{4}{19};

        \prone{O}{5}{1};
        \prone{O}{5}{3};
        \prone{O}{5}{5};
        \prone{O}{5}{7};
        \prone{O}{5}{9};
        \prone{O}{5}{11};
        \prone{O}{5}{15};
        \prone{O}{5}{18};

        \prone{O}{6}{15};

        \prone{O}{7}{1};
        \prone{O}{7}{3};
        \prone{O}{7}{5};
        \prone{O}{7}{7};
        \prone{O}{7}{9};
        \prone{O}{7}{11};
        \prone{O}{7}{14};
        \prone{O}{7}{20};

        \prone{O}{8}{20};

        \prone{O}{9}{1};
        \prone{O}{9}{3};
        \prone{O}{9}{5};
        \prone{O}{9}{7};
        \prone{O}{9}{9};
        \prone{O}{9}{11};
        \prone{O}{9}{16};
        \prone{O}{9}{19};

        \prone{O}{10}{16};

        \prone{O}{11}{1};
        \prone{O}{11}{3};
        \prone{O}{11}{5};
        \prone{O}{11}{7};
        \prone{O}{11}{9};
        \prone{O}{11}{11};
        \prone{O}{11}{15};
        \prone{O}{11}{23};

        \prone{O}{12}{23};

        \prone{O}{13}{3};
        \prone{O}{13}{23};
        \prone{O}{13}{24};
        \prone{O}{13}{25};
        \prone{O}{13}{26};
        \prone{O}{13}{27};
        \prone{O}{13}{28};
        \prone{O}{13}{29};

        \prone{O}{14}{1};
        \prone{O}{14}{2};
        \prone{O}{14}{7};

        \prone{O}{15}{5};
        \prone{O}{15}{6};
        \prone{O}{15}{11};

        \prone{O}{16}{9};
        \prone{O}{16}{10};
        \prone{O}{16}{30};

        \prone{O}{17}{23};
        \prone{O}{17}{24};
        \prone{O}{17}{25};
        \prone{O}{17}{26};
        \prone{O}{17}{27};
        \prone{O}{17}{28};
        \prone{O}{17}{29};
        \prone{O}{17}{30};

        \prone{O}{18}{5};
        \prone{O}{18}{23};
        \prone{O}{18}{24};
        \prone{O}{18}{25};
        \prone{O}{18}{26};
        \prone{O}{18}{27};
        \prone{O}{18}{28};
        \prone{O}{18}{29};

        \prone{O}{19}{3};
        \prone{O}{19}{4};
        \prone{O}{19}{9};

        \prone{O}{20}{7};
        \prone{O}{20}{8};
        \prone{O}{20}{30};

        \prone{O}{21}{23};
        \prone{O}{21}{24};
        \prone{O}{21}{25};
        \prone{O}{21}{26};
        \prone{O}{21}{27};
        \prone{O}{21}{28};
        \prone{O}{21}{29};
        \prone{O}{21}{30};

        \prone{O}{22}{23};

        \prone{O}{23}{11};
        \prone{O}{23}{12};
        \prone{O}{23}{13};
        \prone{O}{23}{17};
        \prone{O}{23}{18};
        \prone{O}{23}{21};
        \prone{O}{23}{22};
        \prone{O}{23}{23};

        \prone{O}{24}{13};
        \prone{O}{24}{17};
        \prone{O}{24}{18};
        \prone{O}{24}{21};

        \prone{O}{25}{13};
        \prone{O}{25}{17};
        \prone{O}{25}{18};
        \prone{O}{25}{21};

        \prone{O}{26}{13};
        \prone{O}{26}{17};
        \prone{O}{26}{18};
        \prone{O}{26}{21};

        \prone{O}{27}{13};
        \prone{O}{27}{17};
        \prone{O}{27}{18};
        \prone{O}{27}{21};

        \prone{O}{28}{13};
        \prone{O}{28}{17};
        \prone{O}{28}{18};
        \prone{O}{28}{21};

        \prone{O}{29}{13};
        \prone{O}{29}{17};
        \prone{O}{29}{18};
        \prone{O}{29}{21};

        \prone{O}{30}{16};
        \prone{O}{30}{17};
        \prone{O}{30}{20};
        \prone{O}{30}{21};


    \end{tikzpicture}
    \caption{An example of instance for which an optimal density is $O(\sqrt{n})$ times the density of a solution returned by the Neighborization algorithm.}
    \label{fig:neighborizationExample}
\end{figure}

\begin{figure}
    \centering

    \begin{tikzpicture}
        \coordinate (O) at (0,0);
        \prgrid{O}{24}{24}

\definecolor{tempcolor}{rgb}{0.0,0.0,1.0}
        \pronec{O}{1}{1}{tempcolor};
\definecolor{tempcolor}{rgb}{0.0,0.0,1.0}
        \pronec{O}{1}{2}{tempcolor};
\definecolor{tempcolor}{rgb}{0.0,0.0,1.0}
        \pronec{O}{1}{3}{tempcolor};
\definecolor{tempcolor}{rgb}{0.0,0.0,1.0}
        \pronec{O}{1}{4}{tempcolor};
\definecolor{tempcolor}{rgb}{0.0,0.0,1.0}
        \pronec{O}{1}{5}{tempcolor};
\definecolor{tempcolor}{rgb}{0.0,0.0,1.0}
        \pronec{O}{1}{6}{tempcolor};
\definecolor{tempcolor}{rgb}{1.0,0.0,0.0}
        \pronec{O}{1}{9}{tempcolor};

\definecolor{tempcolor}{rgb}{0.0,0.0,1.0}
        \pronec{O}{2}{1}{tempcolor};
\definecolor{tempcolor}{rgb}{0.0,0.0,1.0}
        \pronec{O}{2}{2}{tempcolor};
\definecolor{tempcolor}{rgb}{0.0,0.0,1.0}
        \pronec{O}{2}{3}{tempcolor};
\definecolor{tempcolor}{rgb}{0.0,0.0,1.0}
        \pronec{O}{2}{4}{tempcolor};
\definecolor{tempcolor}{rgb}{0.0,0.0,1.0}
        \pronec{O}{2}{5}{tempcolor};
\definecolor{tempcolor}{rgb}{0.0,0.0,1.0}
        \pronec{O}{2}{6}{tempcolor};
\definecolor{tempcolor}{rgb}{1.0,0.0,0.0}
        \pronec{O}{2}{8}{tempcolor};
\definecolor{tempcolor}{rgb}{1.0,0.0,0.0}
        \pronec{O}{2}{9}{tempcolor};
\definecolor{tempcolor}{rgb}{1.0,0.0,0.0}
        \pronec{O}{2}{14}{tempcolor};

\definecolor{tempcolor}{rgb}{0.0,0.0,1.0}
        \pronec{O}{3}{1}{tempcolor};
\definecolor{tempcolor}{rgb}{0.0,0.0,1.0}
        \pronec{O}{3}{2}{tempcolor};
\definecolor{tempcolor}{rgb}{0.0,0.0,1.0}
        \pronec{O}{3}{3}{tempcolor};
\definecolor{tempcolor}{rgb}{0.0,0.0,1.0}
        \pronec{O}{3}{4}{tempcolor};
\definecolor{tempcolor}{rgb}{0.0,0.0,1.0}
        \pronec{O}{3}{5}{tempcolor};
\definecolor{tempcolor}{rgb}{0.0,0.0,1.0}
        \pronec{O}{3}{6}{tempcolor};
\definecolor{tempcolor}{rgb}{1.0,0.0,0.0}
        \pronec{O}{3}{10}{tempcolor};
\definecolor{tempcolor}{rgb}{1.0,0.0,0.0}
        \pronec{O}{3}{13}{tempcolor};
\definecolor{tempcolor}{rgb}{1.0,0.0,0.0}
        \pronec{O}{3}{14}{tempcolor};

\definecolor{tempcolor}{rgb}{0.0,0.0,1.0}
        \pronec{O}{4}{1}{tempcolor};
\definecolor{tempcolor}{rgb}{0.0,0.0,1.0}
        \pronec{O}{4}{2}{tempcolor};
\definecolor{tempcolor}{rgb}{0.0,0.0,1.0}
        \pronec{O}{4}{3}{tempcolor};
\definecolor{tempcolor}{rgb}{0.0,0.0,1.0}
        \pronec{O}{4}{4}{tempcolor};
\definecolor{tempcolor}{rgb}{0.0,0.0,1.0}
        \pronec{O}{4}{5}{tempcolor};
\definecolor{tempcolor}{rgb}{0.0,0.0,1.0}
        \pronec{O}{4}{6}{tempcolor};
\definecolor{tempcolor}{rgb}{1.0,0.0,0.0}
        \pronec{O}{4}{9}{tempcolor};
\definecolor{tempcolor}{rgb}{1.0,0.0,0.0}
        \pronec{O}{4}{10}{tempcolor};
\definecolor{tempcolor}{rgb}{1.0,0.0,0.0}
        \pronec{O}{4}{15}{tempcolor};

\definecolor{tempcolor}{rgb}{0.0,0.0,1.0}
        \pronec{O}{5}{1}{tempcolor};
\definecolor{tempcolor}{rgb}{0.0,0.0,1.0}
        \pronec{O}{5}{2}{tempcolor};
\definecolor{tempcolor}{rgb}{0.0,0.0,1.0}
        \pronec{O}{5}{3}{tempcolor};
\definecolor{tempcolor}{rgb}{0.0,0.0,1.0}
        \pronec{O}{5}{4}{tempcolor};
\definecolor{tempcolor}{rgb}{0.0,0.0,1.0}
        \pronec{O}{5}{5}{tempcolor};
\definecolor{tempcolor}{rgb}{0.0,0.0,1.0}
        \pronec{O}{5}{6}{tempcolor};
\definecolor{tempcolor}{rgb}{1.0,0.0,0.0}
        \pronec{O}{5}{11}{tempcolor};
\definecolor{tempcolor}{rgb}{1.0,0.0,0.0}
        \pronec{O}{5}{14}{tempcolor};
\definecolor{tempcolor}{rgb}{1.0,0.0,0.0}
        \pronec{O}{5}{15}{tempcolor};

\definecolor{tempcolor}{rgb}{0.0,0.0,1.0}
        \pronec{O}{6}{1}{tempcolor};
\definecolor{tempcolor}{rgb}{0.0,0.0,1.0}
        \pronec{O}{6}{2}{tempcolor};
\definecolor{tempcolor}{rgb}{0.0,0.0,1.0}
        \pronec{O}{6}{3}{tempcolor};
\definecolor{tempcolor}{rgb}{0.0,0.0,1.0}
        \pronec{O}{6}{4}{tempcolor};
\definecolor{tempcolor}{rgb}{0.0,0.0,1.0}
        \pronec{O}{6}{5}{tempcolor};
\definecolor{tempcolor}{rgb}{0.0,0.0,1.0}
        \pronec{O}{6}{6}{tempcolor};
\definecolor{tempcolor}{rgb}{1.0,0.0,0.0}
        \pronec{O}{6}{10}{tempcolor};
\definecolor{tempcolor}{rgb}{1.0,0.0,0.0}
        \pronec{O}{6}{11}{tempcolor};
\definecolor{tempcolor}{rgb}{0.0,0.3921568691730499,0.0}
        \pronec{O}{6}{18}{tempcolor};

\definecolor{tempcolor}{rgb}{0.0,0.3921568691730499,0.0}
        \pronec{O}{7}{18}{tempcolor};
\definecolor{tempcolor}{rgb}{0.0,0.3921568691730499,0.0}
        \pronec{O}{7}{23}{tempcolor};
\definecolor{tempcolor}{rgb}{0.0,0.3921568691730499,0.0}
        \pronec{O}{7}{24}{tempcolor};

\definecolor{tempcolor}{rgb}{1.0,0.0,0.0}
        \pronec{O}{8}{2}{tempcolor};
\definecolor{tempcolor}{rgb}{0.0,0.3921568691730499,0.0}
        \pronec{O}{8}{18}{tempcolor};
\definecolor{tempcolor}{rgb}{0.0,0.3921568691730499,0.0}
        \pronec{O}{8}{19}{tempcolor};
\definecolor{tempcolor}{rgb}{0.0,0.3921568691730499,0.0}
        \pronec{O}{8}{20}{tempcolor};
\definecolor{tempcolor}{rgb}{0.0,0.3921568691730499,0.0}
        \pronec{O}{8}{21}{tempcolor};
\definecolor{tempcolor}{rgb}{0.0,0.3921568691730499,0.0}
        \pronec{O}{8}{22}{tempcolor};
\definecolor{tempcolor}{rgb}{0.0,0.3921568691730499,0.0}
        \pronec{O}{8}{23}{tempcolor};

\definecolor{tempcolor}{rgb}{1.0,0.0,0.0}
        \pronec{O}{9}{1}{tempcolor};
\definecolor{tempcolor}{rgb}{1.0,0.0,0.0}
        \pronec{O}{9}{2}{tempcolor};
\definecolor{tempcolor}{rgb}{1.0,0.0,0.0}
        \pronec{O}{9}{4}{tempcolor};

\definecolor{tempcolor}{rgb}{1.0,0.0,0.0}
        \pronec{O}{10}{3}{tempcolor};
\definecolor{tempcolor}{rgb}{1.0,0.0,0.0}
        \pronec{O}{10}{4}{tempcolor};
\definecolor{tempcolor}{rgb}{1.0,0.0,0.0}
        \pronec{O}{10}{6}{tempcolor};

\definecolor{tempcolor}{rgb}{1.0,0.0,0.0}
        \pronec{O}{11}{5}{tempcolor};
\definecolor{tempcolor}{rgb}{1.0,0.0,0.0}
        \pronec{O}{11}{6}{tempcolor};
\definecolor{tempcolor}{rgb}{0.0,0.3921568691730499,0.0}
        \pronec{O}{11}{24}{tempcolor};

\definecolor{tempcolor}{rgb}{0.0,0.3921568691730499,0.0}
        \pronec{O}{12}{18}{tempcolor};
\definecolor{tempcolor}{rgb}{0.0,0.3921568691730499,0.0}
        \pronec{O}{12}{19}{tempcolor};
\definecolor{tempcolor}{rgb}{0.0,0.3921568691730499,0.0}
        \pronec{O}{12}{20}{tempcolor};
\definecolor{tempcolor}{rgb}{0.0,0.3921568691730499,0.0}
        \pronec{O}{12}{21}{tempcolor};
\definecolor{tempcolor}{rgb}{0.0,0.3921568691730499,0.0}
        \pronec{O}{12}{22}{tempcolor};
\definecolor{tempcolor}{rgb}{0.0,0.3921568691730499,0.0}
        \pronec{O}{12}{23}{tempcolor};
\definecolor{tempcolor}{rgb}{0.0,0.3921568691730499,0.0}
        \pronec{O}{12}{24}{tempcolor};

\definecolor{tempcolor}{rgb}{1.0,0.0,0.0}
        \pronec{O}{13}{3}{tempcolor};
\definecolor{tempcolor}{rgb}{0.0,0.3921568691730499,0.0}
        \pronec{O}{13}{18}{tempcolor};
\definecolor{tempcolor}{rgb}{0.0,0.3921568691730499,0.0}
        \pronec{O}{13}{19}{tempcolor};
\definecolor{tempcolor}{rgb}{0.0,0.3921568691730499,0.0}
        \pronec{O}{13}{20}{tempcolor};
\definecolor{tempcolor}{rgb}{0.0,0.3921568691730499,0.0}
        \pronec{O}{13}{21}{tempcolor};
\definecolor{tempcolor}{rgb}{0.0,0.3921568691730499,0.0}
        \pronec{O}{13}{22}{tempcolor};
\definecolor{tempcolor}{rgb}{0.0,0.3921568691730499,0.0}
        \pronec{O}{13}{23}{tempcolor};

\definecolor{tempcolor}{rgb}{1.0,0.0,0.0}
        \pronec{O}{14}{2}{tempcolor};
\definecolor{tempcolor}{rgb}{1.0,0.0,0.0}
        \pronec{O}{14}{3}{tempcolor};
\definecolor{tempcolor}{rgb}{1.0,0.0,0.0}
        \pronec{O}{14}{5}{tempcolor};

\definecolor{tempcolor}{rgb}{1.0,0.0,0.0}
        \pronec{O}{15}{4}{tempcolor};
\definecolor{tempcolor}{rgb}{1.0,0.0,0.0}
        \pronec{O}{15}{5}{tempcolor};
\definecolor{tempcolor}{rgb}{0.0,0.3921568691730499,0.0}
        \pronec{O}{15}{24}{tempcolor};

\definecolor{tempcolor}{rgb}{0.0,0.3921568691730499,0.0}
        \pronec{O}{16}{18}{tempcolor};
\definecolor{tempcolor}{rgb}{0.0,0.3921568691730499,0.0}
        \pronec{O}{16}{19}{tempcolor};
\definecolor{tempcolor}{rgb}{0.0,0.3921568691730499,0.0}
        \pronec{O}{16}{20}{tempcolor};
\definecolor{tempcolor}{rgb}{0.0,0.3921568691730499,0.0}
        \pronec{O}{16}{21}{tempcolor};
\definecolor{tempcolor}{rgb}{0.0,0.3921568691730499,0.0}
        \pronec{O}{16}{22}{tempcolor};
\definecolor{tempcolor}{rgb}{0.0,0.3921568691730499,0.0}
        \pronec{O}{16}{23}{tempcolor};
\definecolor{tempcolor}{rgb}{0.0,0.3921568691730499,0.0}
        \pronec{O}{16}{24}{tempcolor};

\definecolor{tempcolor}{rgb}{0.0,0.3921568691730499,0.0}
        \pronec{O}{17}{18}{tempcolor};

\definecolor{tempcolor}{rgb}{0.0,0.3921568691730499,0.0}
        \pronec{O}{18}{6}{tempcolor};
\definecolor{tempcolor}{rgb}{0.0,0.3921568691730499,0.0}
        \pronec{O}{18}{7}{tempcolor};
\definecolor{tempcolor}{rgb}{0.0,0.3921568691730499,0.0}
        \pronec{O}{18}{8}{tempcolor};
\definecolor{tempcolor}{rgb}{0.0,0.3921568691730499,0.0}
        \pronec{O}{18}{12}{tempcolor};
\definecolor{tempcolor}{rgb}{0.0,0.3921568691730499,0.0}
        \pronec{O}{18}{13}{tempcolor};
\definecolor{tempcolor}{rgb}{0.0,0.3921568691730499,0.0}
        \pronec{O}{18}{16}{tempcolor};
\definecolor{tempcolor}{rgb}{0.0,0.3921568691730499,0.0}
        \pronec{O}{18}{17}{tempcolor};
\definecolor{tempcolor}{rgb}{0.0,0.3921568691730499,0.0}
        \pronec{O}{18}{18}{tempcolor};

\definecolor{tempcolor}{rgb}{0.0,0.3921568691730499,0.0}
        \pronec{O}{19}{8}{tempcolor};
\definecolor{tempcolor}{rgb}{0.0,0.3921568691730499,0.0}
        \pronec{O}{19}{12}{tempcolor};
\definecolor{tempcolor}{rgb}{0.0,0.3921568691730499,0.0}
        \pronec{O}{19}{13}{tempcolor};
\definecolor{tempcolor}{rgb}{0.0,0.3921568691730499,0.0}
        \pronec{O}{19}{16}{tempcolor};

\definecolor{tempcolor}{rgb}{0.0,0.3921568691730499,0.0}
        \pronec{O}{20}{8}{tempcolor};
\definecolor{tempcolor}{rgb}{0.0,0.3921568691730499,0.0}
        \pronec{O}{20}{12}{tempcolor};
\definecolor{tempcolor}{rgb}{0.0,0.3921568691730499,0.0}
        \pronec{O}{20}{13}{tempcolor};
\definecolor{tempcolor}{rgb}{0.0,0.3921568691730499,0.0}
        \pronec{O}{20}{16}{tempcolor};

\definecolor{tempcolor}{rgb}{0.0,0.3921568691730499,0.0}
        \pronec{O}{21}{8}{tempcolor};
\definecolor{tempcolor}{rgb}{0.0,0.3921568691730499,0.0}
        \pronec{O}{21}{12}{tempcolor};
\definecolor{tempcolor}{rgb}{0.0,0.3921568691730499,0.0}
        \pronec{O}{21}{13}{tempcolor};
\definecolor{tempcolor}{rgb}{0.0,0.3921568691730499,0.0}
        \pronec{O}{21}{16}{tempcolor};

\definecolor{tempcolor}{rgb}{0.0,0.3921568691730499,0.0}
        \pronec{O}{22}{8}{tempcolor};
\definecolor{tempcolor}{rgb}{0.0,0.3921568691730499,0.0}
        \pronec{O}{22}{12}{tempcolor};
\definecolor{tempcolor}{rgb}{0.0,0.3921568691730499,0.0}
        \pronec{O}{22}{13}{tempcolor};
\definecolor{tempcolor}{rgb}{0.0,0.3921568691730499,0.0}
        \pronec{O}{22}{16}{tempcolor};

\definecolor{tempcolor}{rgb}{0.0,0.3921568691730499,0.0}
        \pronec{O}{23}{7}{tempcolor};
\definecolor{tempcolor}{rgb}{0.0,0.3921568691730499,0.0}
        \pronec{O}{23}{8}{tempcolor};
\definecolor{tempcolor}{rgb}{0.0,0.3921568691730499,0.0}
        \pronec{O}{23}{12}{tempcolor};
\definecolor{tempcolor}{rgb}{0.0,0.3921568691730499,0.0}
        \pronec{O}{23}{13}{tempcolor};
\definecolor{tempcolor}{rgb}{0.0,0.3921568691730499,0.0}
        \pronec{O}{23}{16}{tempcolor};

\definecolor{tempcolor}{rgb}{0.0,0.3921568691730499,0.0}
        \pronec{O}{24}{7}{tempcolor};
\definecolor{tempcolor}{rgb}{0.0,0.3921568691730499,0.0}
        \pronec{O}{24}{11}{tempcolor};
\definecolor{tempcolor}{rgb}{0.0,0.3921568691730499,0.0}
        \pronec{O}{24}{12}{tempcolor};
\definecolor{tempcolor}{rgb}{0.0,0.3921568691730499,0.0}
        \pronec{O}{24}{15}{tempcolor};
\definecolor{tempcolor}{rgb}{0.0,0.3921568691730499,0.0}
        \pronec{O}{24}{16}{tempcolor};

\draw ($(O)+(0,2.25)$) node[anchor=east] {$5$};
\draw ($(O)+(0,4.75)$) node[anchor=east] {$10$};
\draw ($(O)+(0,7.25)$) node[anchor=east] {$15$};
\draw ($(O)+(0,9.75)$) node[anchor=east] {$20$};
\draw ($(O)+(2.25,-0.3)$) node {$5$};
\draw ($(O)+(4.75,-0.3)$) node {$10$};
\draw ($(O)+(7.25,-0.3)$) node {$15$};
\draw ($(O)+(9.75,-0.3)$) node {$20$};

    \end{tikzpicture}
     \caption{An optimal solution of the instance given in Figure~\ref{fig:neighborizationExample}. The density of this solution is $O(n)$.}
     \label{fig:neighborizationExampleGood}
\end{figure}

\begin{figure}[ht!]
    \centering

    \begin{tikzpicture}
        \coordinate (O) at (0,0);
        \prgrid{O}{25}{25}

        \prone{O}{1}{1};
        \prone{O}{1}{3};
        \prone{O}{1}{5};
        \prone{O}{1}{7};
        \prone{O}{1}{9};
        \prone{O}{1}{11};
        \prone{O}{1}{13};

        \prone{O}{2}{13};

        \prone{O}{3}{1};
        \prone{O}{3}{3};
        \prone{O}{3}{5};
        \prone{O}{3}{7};
        \prone{O}{3}{9};
        \prone{O}{3}{11};
        \prone{O}{3}{13};
        \prone{O}{3}{15};

        \prone{O}{4}{15};

        \prone{O}{5}{1};
        \prone{O}{5}{3};
        \prone{O}{5}{5};
        \prone{O}{5}{7};
        \prone{O}{5}{9};
        \prone{O}{5}{11};
        \prone{O}{5}{13};
        \prone{O}{5}{15};

        \prone{O}{6}{13};

        \prone{O}{7}{1};
        \prone{O}{7}{3};
        \prone{O}{7}{5};
        \prone{O}{7}{7};
        \prone{O}{7}{9};
        \prone{O}{7}{11};
        \prone{O}{7}{13};
        \prone{O}{7}{15};

        \prone{O}{8}{15};

        \prone{O}{9}{1};
        \prone{O}{9}{3};
        \prone{O}{9}{5};
        \prone{O}{9}{7};
        \prone{O}{9}{9};
        \prone{O}{9}{11};
        \prone{O}{9}{13};
        \prone{O}{9}{15};

        \prone{O}{10}{13};

        \prone{O}{11}{1};
        \prone{O}{11}{3};
        \prone{O}{11}{5};
        \prone{O}{11}{7};
        \prone{O}{11}{9};
        \prone{O}{11}{11};
        \prone{O}{11}{13};
        \prone{O}{11}{18};

        \prone{O}{12}{18};

        \prone{O}{13}{1};
        \prone{O}{13}{2};
        \prone{O}{13}{3};
        \prone{O}{13}{5};
        \prone{O}{13}{6};
        \prone{O}{13}{7};
        \prone{O}{13}{9};
        \prone{O}{13}{10};
        \prone{O}{13}{11};
        \prone{O}{13}{18};
        \prone{O}{13}{19};
        \prone{O}{13}{20};
        \prone{O}{13}{21};
        \prone{O}{13}{22};
        \prone{O}{13}{23};
        \prone{O}{13}{24};
        \prone{O}{13}{25};

        \prone{O}{14}{18};
        \prone{O}{14}{19};
        \prone{O}{14}{20};
        \prone{O}{14}{21};
        \prone{O}{14}{22};
        \prone{O}{14}{23};
        \prone{O}{14}{24};
        \prone{O}{14}{25};

        \prone{O}{15}{3};
        \prone{O}{15}{4};
        \prone{O}{15}{5};
        \prone{O}{15}{7};
        \prone{O}{15}{8};
        \prone{O}{15}{9};
        \prone{O}{15}{18};
        \prone{O}{15}{19};
        \prone{O}{15}{20};
        \prone{O}{15}{21};
        \prone{O}{15}{22};
        \prone{O}{15}{23};
        \prone{O}{15}{24};
        \prone{O}{15}{25};

        \prone{O}{16}{18};
        \prone{O}{16}{19};
        \prone{O}{16}{20};
        \prone{O}{16}{21};
        \prone{O}{16}{22};
        \prone{O}{16}{23};
        \prone{O}{16}{24};
        \prone{O}{16}{25};

        \prone{O}{17}{18};

        \prone{O}{18}{11};
        \prone{O}{18}{12};
        \prone{O}{18}{13};
        \prone{O}{18}{14};
        \prone{O}{18}{15};
        \prone{O}{18}{16};
        \prone{O}{18}{17};
        \prone{O}{18}{18};

        \prone{O}{19}{13};
        \prone{O}{19}{14};
        \prone{O}{19}{15};
        \prone{O}{19}{16};

        \prone{O}{20}{13};
        \prone{O}{20}{14};
        \prone{O}{20}{15};
        \prone{O}{20}{16};

        \prone{O}{21}{13};
        \prone{O}{21}{14};
        \prone{O}{21}{15};
        \prone{O}{21}{16};

        \prone{O}{22}{13};
        \prone{O}{22}{14};
        \prone{O}{22}{15};
        \prone{O}{22}{16};

        \prone{O}{23}{13};
        \prone{O}{23}{14};
        \prone{O}{23}{15};
        \prone{O}{23}{16};

        \prone{O}{24}{13};
        \prone{O}{24}{14};
        \prone{O}{24}{15};
        \prone{O}{24}{16};

        \prone{O}{25}{13};
        \prone{O}{25}{14};
        \prone{O}{25}{15};
        \prone{O}{25}{16};


    \end{tikzpicture}
    \caption{A solution of the instance given in Figure~\ref{fig:neighborizationExample} that is returned by the neighborization algorithm. The density of this solution is $O(n)$.}
    \label{fig:neighborizationExampleBad}
\end{figure}


\subsection{Adaptation to the LCL algorithm}
\renewcommand{\gridsize}{0.35}

In Figure~\ref{fig:lclExample}, we give an instance adapted from the instance of Figure~\ref{fig:badExample}. This instance contains, from coordinates (1,1) to (22,22), is the square of black 1. An optimal solution is obtained by fully contracting this square. To do so, we have to contract lines and columns 1 to 21 except 5,11 and 17. This is illustrated with Figure~\ref{fig:lclExampleGood}. The worst maximal solution is obtained by not contracting it. This instance contains also six groups of four 1: two green groups, two blue groups and two red groups. 

What columns or lines does the LCL algorithm contract in this instance? The algorithm computes two maximal feasible solutions : the LC solution, in which we first contracts a maximal set of lines and then contracts a maximal set of columns; and the CL solution, in which we start with the columns and end with the lines. A problem appears in the instance of Figure~\ref{fig:lclExample} due to the green, blue and red groups. For instance, there are two 1 at coordinates (1,35) and (6,35). Consequently, it is not allowed to contract lines 1,2,3,4 and 5: the algorithm must choose which line not to contract. The same phenomenon occurs for every pair of colored 1 that are on the same line or on the same column. If the algorithm chooses not to contract the columns and lines 4,6,10,12,16 and 18, the LC solution and the CL solution are the same and the algorithm returns the matrix given in Figure~\ref{fig:lclExampleBad}.

\begin{remark}
	Currently, the way the algorithm chooses which line of column not to contract is not specified. Note that if the LCL algorithm always choose the first line or the first column that can be contracted, it returns an optimal solution on the instance of Figure~\ref{fig:lclExample}. It is not know if there is an instance for which that algorithm returns $O(\sqrt{n})$ times an optimal density. 
\end{remark}

\begin{figure}[ht!]
    \centering

	
    \begin{tikzpicture}
        \coordinate (O) at (0,0);
        \prgrid{O}{41}{41}

        \prone{O}{1}{40};

        \prone{O}{2}{2};

        \prone{O}{3}{9};

        \prone{O}{4}{17};

        \prone{O}{5}{24};

        \prone{O}{6}{38};

        \prone{O}{7}{40};

        \prone{O}{8}{36};

        \prone{O}{9}{3};

        \prone{O}{10}{10};

        \prone{O}{11}{18};

        \prone{O}{12}{25};

        \prone{O}{13}{38};

        \prone{O}{14}{34};

        \prone{O}{15}{36};

        \prone{O}{16}{32};

        \prone{O}{17}{4};

        \prone{O}{18}{11};

        \prone{O}{19}{19};

        \prone{O}{20}{26};

        \prone{O}{21}{34};

        \prone{O}{22}{30};

        \prone{O}{23}{32};

        \prone{O}{24}{5};

        \prone{O}{25}{12};

        \prone{O}{26}{20};

        \prone{O}{27}{27};

        \prone{O}{28}{30};
        \prone{O}{28}{41};

        \prone{O}{29}{41};

        \prone{O}{30}{22};
        \prone{O}{30}{28};
        \prone{O}{30}{41};

        \prone{O}{31}{41};

        \prone{O}{32}{16};
        \prone{O}{32}{23};
        \prone{O}{32}{41};

        \prone{O}{33}{41};

        \prone{O}{34}{14};
        \prone{O}{34}{21};
        \prone{O}{34}{41};

        \prone{O}{35}{41};

        \prone{O}{36}{8};
        \prone{O}{36}{15};
        \prone{O}{36}{41};

        \prone{O}{37}{41};

        \prone{O}{38}{6};
        \prone{O}{38}{13};
        \prone{O}{38}{41};

        \prone{O}{39}{41};

        \prone{O}{40}{1};
        \prone{O}{40}{7};
        \prone{O}{40}{41};

        \prone{O}{41}{28};
        \prone{O}{41}{29};
        \prone{O}{41}{30};
        \prone{O}{41}{31};
        \prone{O}{41}{32};
        \prone{O}{41}{33};
        \prone{O}{41}{34};
        \prone{O}{41}{35};
        \prone{O}{41}{36};
        \prone{O}{41}{37};
        \prone{O}{41}{38};
        \prone{O}{41}{39};
        \prone{O}{41}{40};
        \prone{O}{41}{41};


    \end{tikzpicture}
    \caption{An example of instance for which an optimal density is $O(\sqrt{n})$ times the density of a solution returned by the LCL algorithm.}
    \label{fig:lclExample}
\end{figure}

\renewcommand{\gridsize}{0.5}
\begin{figure}
    \centering

    \begin{tikzpicture}
        \coordinate (O) at (0,0);
        \prgrid{O}{17}{17}

\definecolor{tempcolor}{rgb}{0.0,0.0,0.0}
        \pronec{O}{1}{1}{tempcolor};
\definecolor{tempcolor}{rgb}{0.0,0.0,0.0}
        \pronec{O}{1}{2}{tempcolor};
\definecolor{tempcolor}{rgb}{0.0,0.0,0.0}
        \pronec{O}{1}{3}{tempcolor};
\definecolor{tempcolor}{rgb}{0.0,0.0,0.0}
        \pronec{O}{1}{4}{tempcolor};
\definecolor{tempcolor}{rgb}{0.0,0.3921568691730499,0.0}
        \pronec{O}{1}{14}{tempcolor};
\definecolor{tempcolor}{rgb}{0.0,0.3921568691730499,0.0}
        \pronec{O}{1}{16}{tempcolor};

\definecolor{tempcolor}{rgb}{0.0,0.0,0.0}
        \pronec{O}{2}{1}{tempcolor};
\definecolor{tempcolor}{rgb}{0.0,0.0,0.0}
        \pronec{O}{2}{2}{tempcolor};
\definecolor{tempcolor}{rgb}{0.0,0.0,0.0}
        \pronec{O}{2}{3}{tempcolor};
\definecolor{tempcolor}{rgb}{0.0,0.0,0.0}
        \pronec{O}{2}{4}{tempcolor};
\definecolor{tempcolor}{rgb}{0.0,0.0,1.0}
        \pronec{O}{2}{10}{tempcolor};
\definecolor{tempcolor}{rgb}{0.0,0.0,1.0}
        \pronec{O}{2}{12}{tempcolor};
\definecolor{tempcolor}{rgb}{0.0,0.3921568691730499,0.0}
        \pronec{O}{2}{14}{tempcolor};
\definecolor{tempcolor}{rgb}{0.0,0.3921568691730499,0.0}
        \pronec{O}{2}{16}{tempcolor};

\definecolor{tempcolor}{rgb}{0.0,0.0,0.0}
        \pronec{O}{3}{1}{tempcolor};
\definecolor{tempcolor}{rgb}{0.0,0.0,0.0}
        \pronec{O}{3}{2}{tempcolor};
\definecolor{tempcolor}{rgb}{0.0,0.0,0.0}
        \pronec{O}{3}{3}{tempcolor};
\definecolor{tempcolor}{rgb}{0.0,0.0,0.0}
        \pronec{O}{3}{4}{tempcolor};
\definecolor{tempcolor}{rgb}{1.0,0.0,0.0}
        \pronec{O}{3}{6}{tempcolor};
\definecolor{tempcolor}{rgb}{1.0,0.0,0.0}
        \pronec{O}{3}{8}{tempcolor};
\definecolor{tempcolor}{rgb}{0.0,0.0,1.0}
        \pronec{O}{3}{10}{tempcolor};
\definecolor{tempcolor}{rgb}{0.0,0.0,1.0}
        \pronec{O}{3}{12}{tempcolor};

\definecolor{tempcolor}{rgb}{0.0,0.0,0.0}
        \pronec{O}{4}{1}{tempcolor};
\definecolor{tempcolor}{rgb}{0.0,0.0,0.0}
        \pronec{O}{4}{2}{tempcolor};
\definecolor{tempcolor}{rgb}{0.0,0.0,0.0}
        \pronec{O}{4}{3}{tempcolor};
\definecolor{tempcolor}{rgb}{0.0,0.0,0.0}
        \pronec{O}{4}{4}{tempcolor};
\definecolor{tempcolor}{rgb}{1.0,0.0,0.0}
        \pronec{O}{4}{6}{tempcolor};
\definecolor{tempcolor}{rgb}{1.0,0.0,0.0}
        \pronec{O}{4}{8}{tempcolor};
\definecolor{tempcolor}{rgb}{0.0,0.0,0.0}
        \pronec{O}{4}{17}{tempcolor};

\definecolor{tempcolor}{rgb}{0.0,0.0,0.0}
        \pronec{O}{5}{17}{tempcolor};

\definecolor{tempcolor}{rgb}{1.0,0.0,0.0}
        \pronec{O}{6}{3}{tempcolor};
\definecolor{tempcolor}{rgb}{1.0,0.0,0.0}
        \pronec{O}{6}{4}{tempcolor};
\definecolor{tempcolor}{rgb}{0.0,0.0,0.0}
        \pronec{O}{6}{17}{tempcolor};

\definecolor{tempcolor}{rgb}{0.0,0.0,0.0}
        \pronec{O}{7}{17}{tempcolor};

\definecolor{tempcolor}{rgb}{1.0,0.0,0.0}
        \pronec{O}{8}{3}{tempcolor};
\definecolor{tempcolor}{rgb}{1.0,0.0,0.0}
        \pronec{O}{8}{4}{tempcolor};
\definecolor{tempcolor}{rgb}{0.0,0.0,0.0}
        \pronec{O}{8}{17}{tempcolor};

\definecolor{tempcolor}{rgb}{0.0,0.0,0.0}
        \pronec{O}{9}{17}{tempcolor};

\definecolor{tempcolor}{rgb}{0.0,0.0,1.0}
        \pronec{O}{10}{2}{tempcolor};
\definecolor{tempcolor}{rgb}{0.0,0.0,1.0}
        \pronec{O}{10}{3}{tempcolor};
\definecolor{tempcolor}{rgb}{0.0,0.0,0.0}
        \pronec{O}{10}{17}{tempcolor};

\definecolor{tempcolor}{rgb}{0.0,0.0,0.0}
        \pronec{O}{11}{17}{tempcolor};

\definecolor{tempcolor}{rgb}{0.0,0.0,1.0}
        \pronec{O}{12}{2}{tempcolor};
\definecolor{tempcolor}{rgb}{0.0,0.0,1.0}
        \pronec{O}{12}{3}{tempcolor};
\definecolor{tempcolor}{rgb}{0.0,0.0,0.0}
        \pronec{O}{12}{17}{tempcolor};

\definecolor{tempcolor}{rgb}{0.0,0.0,0.0}
        \pronec{O}{13}{17}{tempcolor};

\definecolor{tempcolor}{rgb}{0.0,0.3921568691730499,0.0}
        \pronec{O}{14}{1}{tempcolor};
\definecolor{tempcolor}{rgb}{0.0,0.3921568691730499,0.0}
        \pronec{O}{14}{2}{tempcolor};
\definecolor{tempcolor}{rgb}{0.0,0.0,0.0}
        \pronec{O}{14}{17}{tempcolor};

\definecolor{tempcolor}{rgb}{0.0,0.0,0.0}
        \pronec{O}{15}{17}{tempcolor};

\definecolor{tempcolor}{rgb}{0.0,0.3921568691730499,0.0}
        \pronec{O}{16}{1}{tempcolor};
\definecolor{tempcolor}{rgb}{0.0,0.3921568691730499,0.0}
        \pronec{O}{16}{2}{tempcolor};
\definecolor{tempcolor}{rgb}{0.0,0.0,0.0}
        \pronec{O}{16}{17}{tempcolor};

\definecolor{tempcolor}{rgb}{0.0,0.0,0.0}
        \pronec{O}{17}{4}{tempcolor};
\definecolor{tempcolor}{rgb}{0.0,0.0,0.0}
        \pronec{O}{17}{5}{tempcolor};
\definecolor{tempcolor}{rgb}{0.0,0.0,0.0}
        \pronec{O}{17}{6}{tempcolor};
\definecolor{tempcolor}{rgb}{0.0,0.0,0.0}
        \pronec{O}{17}{7}{tempcolor};
\definecolor{tempcolor}{rgb}{0.0,0.0,0.0}
        \pronec{O}{17}{8}{tempcolor};
\definecolor{tempcolor}{rgb}{0.0,0.0,0.0}
        \pronec{O}{17}{9}{tempcolor};
\definecolor{tempcolor}{rgb}{0.0,0.0,0.0}
        \pronec{O}{17}{10}{tempcolor};
\definecolor{tempcolor}{rgb}{0.0,0.0,0.0}
        \pronec{O}{17}{11}{tempcolor};
\definecolor{tempcolor}{rgb}{0.0,0.0,0.0}
        \pronec{O}{17}{12}{tempcolor};
\definecolor{tempcolor}{rgb}{0.0,0.0,0.0}
        \pronec{O}{17}{13}{tempcolor};
\definecolor{tempcolor}{rgb}{0.0,0.0,0.0}
        \pronec{O}{17}{14}{tempcolor};
\definecolor{tempcolor}{rgb}{0.0,0.0,0.0}
        \pronec{O}{17}{15}{tempcolor};
\definecolor{tempcolor}{rgb}{0.0,0.0,0.0}
        \pronec{O}{17}{16}{tempcolor};
\definecolor{tempcolor}{rgb}{0.0,0.0,0.0}
        \pronec{O}{17}{17}{tempcolor};

\draw ($(O)+(0,2.25)$) node[anchor=east] {$5$};
\draw ($(O)+(0,4.75)$) node[anchor=east] {$10$};
\draw ($(O)+(0,7.25)$) node[anchor=east] {$15$};
\draw ($(O)+(2.25,-0.3)$) node {$5$};
\draw ($(O)+(4.75,-0.3)$) node {$10$};
\draw ($(O)+(7.25,-0.3)$) node {$15$};

    \end{tikzpicture}
   \caption{An optimal solution of the instance given in Figure~\ref{fig:lclExample}. The density of this solution is $O(n)$.}
   \label{fig:lclExampleGood}
\end{figure}

\renewcommand{\gridsize}{0.5}
\begin{figure}
    \centering

    \begin{tikzpicture}
        \coordinate (O) at (0,0);
        \prgrid{O}{20}{20}

\definecolor{tempcolor}{rgb}{0.0,0.0,0.0}
        \pronec{O}{1}{1}{tempcolor};
\definecolor{tempcolor}{rgb}{0.0,0.0,0.0}
        \pronec{O}{1}{3}{tempcolor};
\definecolor{tempcolor}{rgb}{0.0,0.0,0.0}
        \pronec{O}{1}{5}{tempcolor};
\definecolor{tempcolor}{rgb}{0.0,0.0,0.0}
        \pronec{O}{1}{7}{tempcolor};
\definecolor{tempcolor}{rgb}{0.0,0.3921568691730499,0.0}
        \pronec{O}{1}{19}{tempcolor};

\definecolor{tempcolor}{rgb}{0.0,0.3921568691730499,0.0}
        \pronec{O}{2}{17}{tempcolor};
\definecolor{tempcolor}{rgb}{0.0,0.3921568691730499,0.0}
        \pronec{O}{2}{19}{tempcolor};

\definecolor{tempcolor}{rgb}{0.0,0.0,0.0}
        \pronec{O}{3}{1}{tempcolor};
\definecolor{tempcolor}{rgb}{0.0,0.0,0.0}
        \pronec{O}{3}{3}{tempcolor};
\definecolor{tempcolor}{rgb}{0.0,0.0,0.0}
        \pronec{O}{3}{5}{tempcolor};
\definecolor{tempcolor}{rgb}{0.0,0.0,0.0}
        \pronec{O}{3}{7}{tempcolor};
\definecolor{tempcolor}{rgb}{0.0,0.0,1.0}
        \pronec{O}{3}{15}{tempcolor};
\definecolor{tempcolor}{rgb}{0.0,0.3921568691730499,0.0}
        \pronec{O}{3}{17}{tempcolor};

\definecolor{tempcolor}{rgb}{0.0,0.0,1.0}
        \pronec{O}{4}{13}{tempcolor};
\definecolor{tempcolor}{rgb}{0.0,0.0,1.0}
        \pronec{O}{4}{15}{tempcolor};

\definecolor{tempcolor}{rgb}{0.0,0.0,0.0}
        \pronec{O}{5}{1}{tempcolor};
\definecolor{tempcolor}{rgb}{0.0,0.0,0.0}
        \pronec{O}{5}{3}{tempcolor};
\definecolor{tempcolor}{rgb}{0.0,0.0,0.0}
        \pronec{O}{5}{5}{tempcolor};
\definecolor{tempcolor}{rgb}{0.0,0.0,0.0}
        \pronec{O}{5}{7}{tempcolor};
\definecolor{tempcolor}{rgb}{1.0,0.0,0.0}
        \pronec{O}{5}{11}{tempcolor};
\definecolor{tempcolor}{rgb}{0.0,0.0,1.0}
        \pronec{O}{5}{13}{tempcolor};

\definecolor{tempcolor}{rgb}{1.0,0.0,0.0}
        \pronec{O}{6}{9}{tempcolor};
\definecolor{tempcolor}{rgb}{1.0,0.0,0.0}
        \pronec{O}{6}{11}{tempcolor};

\definecolor{tempcolor}{rgb}{0.0,0.0,0.0}
        \pronec{O}{7}{1}{tempcolor};
\definecolor{tempcolor}{rgb}{0.0,0.0,0.0}
        \pronec{O}{7}{3}{tempcolor};
\definecolor{tempcolor}{rgb}{0.0,0.0,0.0}
        \pronec{O}{7}{5}{tempcolor};
\definecolor{tempcolor}{rgb}{0.0,0.0,0.0}
        \pronec{O}{7}{7}{tempcolor};
\definecolor{tempcolor}{rgb}{1.0,0.0,0.0}
        \pronec{O}{7}{9}{tempcolor};
\definecolor{tempcolor}{rgb}{0.0,0.0,0.0}
        \pronec{O}{7}{20}{tempcolor};

\definecolor{tempcolor}{rgb}{0.0,0.0,0.0}
        \pronec{O}{8}{20}{tempcolor};

\definecolor{tempcolor}{rgb}{1.0,0.0,0.0}
        \pronec{O}{9}{6}{tempcolor};
\definecolor{tempcolor}{rgb}{1.0,0.0,0.0}
        \pronec{O}{9}{7}{tempcolor};
\definecolor{tempcolor}{rgb}{0.0,0.0,0.0}
        \pronec{O}{9}{20}{tempcolor};

\definecolor{tempcolor}{rgb}{0.0,0.0,0.0}
        \pronec{O}{10}{20}{tempcolor};

\definecolor{tempcolor}{rgb}{1.0,0.0,0.0}
        \pronec{O}{11}{5}{tempcolor};
\definecolor{tempcolor}{rgb}{1.0,0.0,0.0}
        \pronec{O}{11}{6}{tempcolor};
\definecolor{tempcolor}{rgb}{0.0,0.0,0.0}
        \pronec{O}{11}{20}{tempcolor};

\definecolor{tempcolor}{rgb}{0.0,0.0,0.0}
        \pronec{O}{12}{20}{tempcolor};

\definecolor{tempcolor}{rgb}{0.0,0.0,1.0}
        \pronec{O}{13}{4}{tempcolor};
\definecolor{tempcolor}{rgb}{0.0,0.0,1.0}
        \pronec{O}{13}{5}{tempcolor};
\definecolor{tempcolor}{rgb}{0.0,0.0,0.0}
        \pronec{O}{13}{20}{tempcolor};

\definecolor{tempcolor}{rgb}{0.0,0.0,0.0}
        \pronec{O}{14}{20}{tempcolor};

\definecolor{tempcolor}{rgb}{0.0,0.0,1.0}
        \pronec{O}{15}{3}{tempcolor};
\definecolor{tempcolor}{rgb}{0.0,0.0,1.0}
        \pronec{O}{15}{4}{tempcolor};
\definecolor{tempcolor}{rgb}{0.0,0.0,0.0}
        \pronec{O}{15}{20}{tempcolor};

\definecolor{tempcolor}{rgb}{0.0,0.0,0.0}
        \pronec{O}{16}{20}{tempcolor};

\definecolor{tempcolor}{rgb}{0.0,0.3921568691730499,0.0}
        \pronec{O}{17}{2}{tempcolor};
\definecolor{tempcolor}{rgb}{0.0,0.3921568691730499,0.0}
        \pronec{O}{17}{3}{tempcolor};
\definecolor{tempcolor}{rgb}{0.0,0.0,0.0}
        \pronec{O}{17}{20}{tempcolor};

\definecolor{tempcolor}{rgb}{0.0,0.0,0.0}
        \pronec{O}{18}{20}{tempcolor};

\definecolor{tempcolor}{rgb}{0.0,0.3921568691730499,0.0}
        \pronec{O}{19}{1}{tempcolor};
\definecolor{tempcolor}{rgb}{0.0,0.3921568691730499,0.0}
        \pronec{O}{19}{2}{tempcolor};
\definecolor{tempcolor}{rgb}{0.0,0.0,0.0}
        \pronec{O}{19}{20}{tempcolor};

\definecolor{tempcolor}{rgb}{0.0,0.0,0.0}
        \pronec{O}{20}{7}{tempcolor};
\definecolor{tempcolor}{rgb}{0.0,0.0,0.0}
        \pronec{O}{20}{8}{tempcolor};
\definecolor{tempcolor}{rgb}{0.0,0.0,0.0}
        \pronec{O}{20}{9}{tempcolor};
\definecolor{tempcolor}{rgb}{0.0,0.0,0.0}
        \pronec{O}{20}{10}{tempcolor};
\definecolor{tempcolor}{rgb}{0.0,0.0,0.0}
        \pronec{O}{20}{11}{tempcolor};
\definecolor{tempcolor}{rgb}{0.0,0.0,0.0}
        \pronec{O}{20}{12}{tempcolor};
\definecolor{tempcolor}{rgb}{0.0,0.0,0.0}
        \pronec{O}{20}{13}{tempcolor};
\definecolor{tempcolor}{rgb}{0.0,0.0,0.0}
        \pronec{O}{20}{14}{tempcolor};
\definecolor{tempcolor}{rgb}{0.0,0.0,0.0}
        \pronec{O}{20}{15}{tempcolor};
\definecolor{tempcolor}{rgb}{0.0,0.0,0.0}
        \pronec{O}{20}{16}{tempcolor};
\definecolor{tempcolor}{rgb}{0.0,0.0,0.0}
        \pronec{O}{20}{17}{tempcolor};
\definecolor{tempcolor}{rgb}{0.0,0.0,0.0}
        \pronec{O}{20}{18}{tempcolor};
\definecolor{tempcolor}{rgb}{0.0,0.0,0.0}
        \pronec{O}{20}{19}{tempcolor};
\definecolor{tempcolor}{rgb}{0.0,0.0,0.0}
        \pronec{O}{20}{20}{tempcolor};

\draw ($(O)+(0,2.25)$) node[anchor=east] {$5$};
\draw ($(O)+(0,4.75)$) node[anchor=east] {$10$};
\draw ($(O)+(0,7.25)$) node[anchor=east] {$15$};
\draw ($(O)+(0,9.75)$) node[anchor=east] {$20$};
\draw ($(O)+(2.25,-0.3)$) node {$5$};
\draw ($(O)+(4.75,-0.3)$) node {$10$};
\draw ($(O)+(7.25,-0.3)$) node {$15$};
\draw ($(O)+(9.75,-0.3)$) node {$20$};

    \end{tikzpicture}
    \caption{A solution of the instance given in Figure~\ref{fig:lclExample} that may be returned by the LCL algorithm. The density of this solution is $O(n)$.}
    \label{fig:lclExampleBad}
\end{figure}


\section{Discussion}

We highly believe this instance is the worst case that can happen and that the density of a maximal solution is always higher than $4\sqrt{n}$. The result of Theorem~\ref{theo:sqrtnapprox} may then possibly be updated to the following conjecture. 

\begin{conjecture}
	An algorithm returning any maximal solution of an instance of MMC is a $\sqrt{n}$-approximation.
\end{conjecture}