\section{An instance with an $O(\sqrt{n})$ gap between the worst and the best solution.}

\label{apx:badinstance}


Theorem~\ref{theo:sqrtnapprox} of Section~\ref{sect:approx} proves a default $2\sqrt{n}$ upper bound of the approximation ratio for every algorithm returning a maximal solution.

We give, in this appendix, in Figure~\ref{fig:badinstance}, an instance in which the ratio between an optimal density and the lowest density of a maximal solution is $O(\sqrt{n})$.

\begin{figure}
    \centering

    \begin{tikzpicture}
        \coordinate (O) at (0,0);
        \prgrid{O}{16}{16}

\definecolor{tempcolor}{rgb}{0.0,0.0,1.0}
        \pronec{O}{1}{1}{tempcolor};
\definecolor{tempcolor}{rgb}{0.0,0.0,1.0}
        \pronec{O}{1}{3}{tempcolor};
\definecolor{tempcolor}{rgb}{0.0,0.0,1.0}
        \pronec{O}{1}{5}{tempcolor};
\definecolor{tempcolor}{rgb}{0.0,0.0,1.0}
        \pronec{O}{1}{7}{tempcolor};
\definecolor{tempcolor}{rgb}{0.0,0.0,0.0}
        \pronec{O}{1}{8}{tempcolor};
\definecolor{tempcolor}{rgb}{0.0,0.0,0.0}
        \pronec{O}{1}{9}{tempcolor};
\definecolor{tempcolor}{rgb}{1.0,0.0,0.0}
        \pronec{O}{1}{16}{tempcolor};

\definecolor{tempcolor}{rgb}{1.0,0.0,0.0}
        \pronec{O}{2}{15}{tempcolor};

\definecolor{tempcolor}{rgb}{0.0,0.0,1.0}
        \pronec{O}{3}{1}{tempcolor};
\definecolor{tempcolor}{rgb}{0.0,0.0,1.0}
        \pronec{O}{3}{3}{tempcolor};
\definecolor{tempcolor}{rgb}{0.0,0.0,1.0}
        \pronec{O}{3}{5}{tempcolor};
\definecolor{tempcolor}{rgb}{0.0,0.0,1.0}
        \pronec{O}{3}{7}{tempcolor};
\definecolor{tempcolor}{rgb}{1.0,0.0,0.0}
        \pronec{O}{3}{14}{tempcolor};

\definecolor{tempcolor}{rgb}{1.0,0.0,0.0}
        \pronec{O}{4}{13}{tempcolor};

\definecolor{tempcolor}{rgb}{0.0,0.0,1.0}
        \pronec{O}{5}{1}{tempcolor};
\definecolor{tempcolor}{rgb}{0.0,0.0,1.0}
        \pronec{O}{5}{3}{tempcolor};
\definecolor{tempcolor}{rgb}{0.0,0.0,1.0}
        \pronec{O}{5}{5}{tempcolor};
\definecolor{tempcolor}{rgb}{0.0,0.0,1.0}
        \pronec{O}{5}{7}{tempcolor};
\definecolor{tempcolor}{rgb}{1.0,0.0,0.0}
        \pronec{O}{5}{12}{tempcolor};

\definecolor{tempcolor}{rgb}{1.0,0.0,0.0}
        \pronec{O}{6}{11}{tempcolor};

\definecolor{tempcolor}{rgb}{0.0,0.0,1.0}
        \pronec{O}{7}{1}{tempcolor};
\definecolor{tempcolor}{rgb}{0.0,0.0,1.0}
        \pronec{O}{7}{3}{tempcolor};
\definecolor{tempcolor}{rgb}{0.0,0.0,1.0}
        \pronec{O}{7}{5}{tempcolor};
\definecolor{tempcolor}{rgb}{0.0,0.0,1.0}
        \pronec{O}{7}{7}{tempcolor};
\definecolor{tempcolor}{rgb}{1.0,0.0,0.0}
        \pronec{O}{7}{10}{tempcolor};

\definecolor{tempcolor}{rgb}{0.0,0.0,0.0}
        \pronec{O}{8}{1}{tempcolor};

\definecolor{tempcolor}{rgb}{0.0,0.0,0.0}
        \pronec{O}{9}{1}{tempcolor};

\definecolor{tempcolor}{rgb}{1.0,0.0,0.0}
        \pronec{O}{10}{7}{tempcolor};

\definecolor{tempcolor}{rgb}{1.0,0.0,0.0}
        \pronec{O}{11}{6}{tempcolor};

\definecolor{tempcolor}{rgb}{1.0,0.0,0.0}
        \pronec{O}{12}{5}{tempcolor};

\definecolor{tempcolor}{rgb}{1.0,0.0,0.0}
        \pronec{O}{13}{4}{tempcolor};

\definecolor{tempcolor}{rgb}{1.0,0.0,0.0}
        \pronec{O}{14}{3}{tempcolor};

\definecolor{tempcolor}{rgb}{1.0,0.0,0.0}
        \pronec{O}{15}{2}{tempcolor};

\definecolor{tempcolor}{rgb}{1.0,0.0,0.0}
        \pronec{O}{16}{1}{tempcolor};

\draw ($(O)+(0,2.25)$) node[anchor=east] {$5$};
\draw ($(O)+(0,4.75)$) node[anchor=east] {$10$};
\draw ($(O)+(0,7.25)$) node[anchor=east] {$15$};
\draw ($(O)+(2.25,-0.3)$) node {$5$};
\draw ($(O)+(4.75,-0.3)$) node {$10$};
\draw ($(O)+(7.25,-0.3)$) node {$15$};

    \end{tikzpicture}
    \caption{An example of grid in which the gap between an optimal solution and the worst maximal solution is $O(\sqrt{n})$.}
    \label{fig:badExample}
\end{figure}

\begin{figure}
    \centering

    \begin{tikzpicture}
        \coordinate (O) at (0,0);
        \prgrid{O}{9}{9}

\definecolor{tempcolor}{rgb}{0.0,0.0,1.0}
        \pronec{O}{1}{1}{tempcolor};
\definecolor{tempcolor}{rgb}{0.0,0.0,1.0}
        \pronec{O}{1}{2}{tempcolor};
\definecolor{tempcolor}{rgb}{0.0,0.0,1.0}
        \pronec{O}{1}{3}{tempcolor};
\definecolor{tempcolor}{rgb}{0.0,0.0,1.0}
        \pronec{O}{1}{4}{tempcolor};
\definecolor{tempcolor}{rgb}{0.0,0.0,0.0}
        \pronec{O}{1}{5}{tempcolor};
\definecolor{tempcolor}{rgb}{0.0,0.0,0.0}
        \pronec{O}{1}{6}{tempcolor};
\definecolor{tempcolor}{rgb}{1.0,0.0,0.0}
        \pronec{O}{1}{8}{tempcolor};
\definecolor{tempcolor}{rgb}{1.0,0.0,0.0}
        \pronec{O}{1}{9}{tempcolor};

\definecolor{tempcolor}{rgb}{0.0,0.0,1.0}
        \pronec{O}{2}{1}{tempcolor};
\definecolor{tempcolor}{rgb}{0.0,0.0,1.0}
        \pronec{O}{2}{2}{tempcolor};
\definecolor{tempcolor}{rgb}{0.0,0.0,1.0}
        \pronec{O}{2}{3}{tempcolor};
\definecolor{tempcolor}{rgb}{0.0,0.0,1.0}
        \pronec{O}{2}{4}{tempcolor};
\definecolor{tempcolor}{rgb}{1.0,0.0,0.0}
        \pronec{O}{2}{7}{tempcolor};
\definecolor{tempcolor}{rgb}{1.0,0.0,0.0}
        \pronec{O}{2}{8}{tempcolor};

\definecolor{tempcolor}{rgb}{0.0,0.0,1.0}
        \pronec{O}{3}{1}{tempcolor};
\definecolor{tempcolor}{rgb}{0.0,0.0,1.0}
        \pronec{O}{3}{2}{tempcolor};
\definecolor{tempcolor}{rgb}{0.0,0.0,1.0}
        \pronec{O}{3}{3}{tempcolor};
\definecolor{tempcolor}{rgb}{0.0,0.0,1.0}
        \pronec{O}{3}{4}{tempcolor};
\definecolor{tempcolor}{rgb}{1.0,0.0,0.0}
        \pronec{O}{3}{6}{tempcolor};
\definecolor{tempcolor}{rgb}{1.0,0.0,0.0}
        \pronec{O}{3}{7}{tempcolor};

\definecolor{tempcolor}{rgb}{0.0,0.0,1.0}
        \pronec{O}{4}{1}{tempcolor};
\definecolor{tempcolor}{rgb}{0.0,0.0,1.0}
        \pronec{O}{4}{2}{tempcolor};
\definecolor{tempcolor}{rgb}{0.0,0.0,1.0}
        \pronec{O}{4}{3}{tempcolor};
\definecolor{tempcolor}{rgb}{0.0,0.0,1.0}
        \pronec{O}{4}{4}{tempcolor};
\definecolor{tempcolor}{rgb}{1.0,0.0,0.0}
        \pronec{O}{4}{6}{tempcolor};

\definecolor{tempcolor}{rgb}{0.0,0.0,0.0}
        \pronec{O}{5}{1}{tempcolor};

\definecolor{tempcolor}{rgb}{0.0,0.0,0.0}
        \pronec{O}{6}{1}{tempcolor};
\definecolor{tempcolor}{rgb}{1.0,0.0,0.0}
        \pronec{O}{6}{3}{tempcolor};
\definecolor{tempcolor}{rgb}{1.0,0.0,0.0}
        \pronec{O}{6}{4}{tempcolor};

\definecolor{tempcolor}{rgb}{1.0,0.0,0.0}
        \pronec{O}{7}{2}{tempcolor};
\definecolor{tempcolor}{rgb}{1.0,0.0,0.0}
        \pronec{O}{7}{3}{tempcolor};

\definecolor{tempcolor}{rgb}{1.0,0.0,0.0}
        \pronec{O}{8}{1}{tempcolor};
\definecolor{tempcolor}{rgb}{1.0,0.0,0.0}
        \pronec{O}{8}{2}{tempcolor};

\definecolor{tempcolor}{rgb}{1.0,0.0,0.0}
        \pronec{O}{9}{1}{tempcolor};

\draw ($(O)+(0,2.25)$) node[anchor=east] {$5$};
\draw ($(O)+(2.25,-0.3)$) node {$5$};

    \end{tikzpicture}
    \caption{An optimal solution of the instance given in Figure~\ref{fig:badExample}. The density of this solution is $O(n^2)$.}
    \label{fig:badexampleGood}
\end{figure}

\begin{figure}
    \centering

 \begin{tikzpicture}
 \coordinate (O) at (0,0);
 \prgrid{O}{10}{10}
 
 \definecolor{tempcolor}{rgb}{0.0,0.0,1.0}
 \pronec{O}{1}{1}{tempcolor};
 \definecolor{tempcolor}{rgb}{0.0,0.0,1.0}
 \pronec{O}{1}{3}{tempcolor};
 \definecolor{tempcolor}{rgb}{0.0,0.0,1.0}
 \pronec{O}{1}{5}{tempcolor};
 \definecolor{tempcolor}{rgb}{0.0,0.0,1.0}
 \pronec{O}{1}{7}{tempcolor};
 \definecolor{tempcolor}{rgb}{0.0,0.0,0.0}
 \pronec{O}{1}{8}{tempcolor};
 \definecolor{tempcolor}{rgb}{0.0,0.0,0.0}
 \pronec{O}{1}{9}{tempcolor};
 \definecolor{tempcolor}{rgb}{1.0,0.0,0.0}
 \pronec{O}{1}{10}{tempcolor};
 
 \definecolor{tempcolor}{rgb}{1.0,0.0,0.0}
 \pronec{O}{2}{10}{tempcolor};
 
 \definecolor{tempcolor}{rgb}{0.0,0.0,1.0}
 \pronec{O}{3}{1}{tempcolor};
 \definecolor{tempcolor}{rgb}{0.0,0.0,1.0}
 \pronec{O}{3}{3}{tempcolor};
 \definecolor{tempcolor}{rgb}{0.0,0.0,1.0}
 \pronec{O}{3}{5}{tempcolor};
 \definecolor{tempcolor}{rgb}{0.0,0.0,1.0}
 \pronec{O}{3}{7}{tempcolor};
 \definecolor{tempcolor}{rgb}{1.0,0.0,0.0}
 \pronec{O}{3}{10}{tempcolor};
 
 \definecolor{tempcolor}{rgb}{1.0,0.0,0.0}
 \pronec{O}{4}{10}{tempcolor};
 
 \definecolor{tempcolor}{rgb}{0.0,0.0,1.0}
 \pronec{O}{5}{1}{tempcolor};
 \definecolor{tempcolor}{rgb}{0.0,0.0,1.0}
 \pronec{O}{5}{3}{tempcolor};
 \definecolor{tempcolor}{rgb}{0.0,0.0,1.0}
 \pronec{O}{5}{5}{tempcolor};
 \definecolor{tempcolor}{rgb}{0.0,0.0,1.0}
 \pronec{O}{5}{7}{tempcolor};
 \definecolor{tempcolor}{rgb}{1.0,0.0,0.0}
 \pronec{O}{5}{10}{tempcolor};
 
 \definecolor{tempcolor}{rgb}{1.0,0.0,0.0}
 \pronec{O}{6}{10}{tempcolor};
 
 \definecolor{tempcolor}{rgb}{0.0,0.0,1.0}
 \pronec{O}{7}{1}{tempcolor};
 \definecolor{tempcolor}{rgb}{0.0,0.0,1.0}
 \pronec{O}{7}{3}{tempcolor};
 \definecolor{tempcolor}{rgb}{0.0,0.0,1.0}
 \pronec{O}{7}{5}{tempcolor};
 \definecolor{tempcolor}{rgb}{0.0,0.0,1.0}
 \pronec{O}{7}{7}{tempcolor};
 \definecolor{tempcolor}{rgb}{1.0,0.0,0.0}
 \pronec{O}{7}{10}{tempcolor};
 
 \definecolor{tempcolor}{rgb}{0.0,0.0,0.0}
 \pronec{O}{8}{1}{tempcolor};
 
 \definecolor{tempcolor}{rgb}{0.0,0.0,0.0}
 \pronec{O}{9}{1}{tempcolor};
 
 \definecolor{tempcolor}{rgb}{1.0,0.0,0.0}
 \pronec{O}{10}{1}{tempcolor};
 \definecolor{tempcolor}{rgb}{1.0,0.0,0.0}
 \pronec{O}{10}{2}{tempcolor};
 \definecolor{tempcolor}{rgb}{1.0,0.0,0.0}
 \pronec{O}{10}{3}{tempcolor};
 \definecolor{tempcolor}{rgb}{1.0,0.0,0.0}
 \pronec{O}{10}{4}{tempcolor};
 \definecolor{tempcolor}{rgb}{1.0,0.0,0.0}
 \pronec{O}{10}{5}{tempcolor};
 \definecolor{tempcolor}{rgb}{1.0,0.0,0.0}
 \pronec{O}{10}{6}{tempcolor};
 \definecolor{tempcolor}{rgb}{1.0,0.0,0.0}
 \pronec{O}{10}{7}{tempcolor};
 
 \draw ($(O)+(0,2.25)$) node[anchor=east] {$5$};
 \draw ($(O)+(0,4.75)$) node[anchor=east] {$10$};
 \draw ($(O)+(2.25,-0.3)$) node {$5$};
 \draw ($(O)+(4.75,-0.3)$) node {$10$};
 
 \end{tikzpicture}
    \caption{A feasible solution of the instance given in Figure~\ref{fig:badExample} for which the density is $O(\sqrt{n})$.}
    \label{fig:badExampleBad}
\end{figure}


\subsection{Adaptation to the LCL algorithm}
\renewcommand{\gridsize}{0.35}
\begin{figure}[ht!]
    \centering

	
    \begin{tikzpicture}
        \coordinate (O) at (0,0);
        \prgrid{O}{41}{41}

        \prone{O}{1}{40};

        \prone{O}{2}{2};

        \prone{O}{3}{9};

        \prone{O}{4}{17};

        \prone{O}{5}{24};

        \prone{O}{6}{38};

        \prone{O}{7}{40};

        \prone{O}{8}{36};

        \prone{O}{9}{3};

        \prone{O}{10}{10};

        \prone{O}{11}{18};

        \prone{O}{12}{25};

        \prone{O}{13}{38};

        \prone{O}{14}{34};

        \prone{O}{15}{36};

        \prone{O}{16}{32};

        \prone{O}{17}{4};

        \prone{O}{18}{11};

        \prone{O}{19}{19};

        \prone{O}{20}{26};

        \prone{O}{21}{34};

        \prone{O}{22}{30};

        \prone{O}{23}{32};

        \prone{O}{24}{5};

        \prone{O}{25}{12};

        \prone{O}{26}{20};

        \prone{O}{27}{27};

        \prone{O}{28}{30};
        \prone{O}{28}{41};

        \prone{O}{29}{41};

        \prone{O}{30}{22};
        \prone{O}{30}{28};
        \prone{O}{30}{41};

        \prone{O}{31}{41};

        \prone{O}{32}{16};
        \prone{O}{32}{23};
        \prone{O}{32}{41};

        \prone{O}{33}{41};

        \prone{O}{34}{14};
        \prone{O}{34}{21};
        \prone{O}{34}{41};

        \prone{O}{35}{41};

        \prone{O}{36}{8};
        \prone{O}{36}{15};
        \prone{O}{36}{41};

        \prone{O}{37}{41};

        \prone{O}{38}{6};
        \prone{O}{38}{13};
        \prone{O}{38}{41};

        \prone{O}{39}{41};

        \prone{O}{40}{1};
        \prone{O}{40}{7};
        \prone{O}{40}{41};

        \prone{O}{41}{28};
        \prone{O}{41}{29};
        \prone{O}{41}{30};
        \prone{O}{41}{31};
        \prone{O}{41}{32};
        \prone{O}{41}{33};
        \prone{O}{41}{34};
        \prone{O}{41}{35};
        \prone{O}{41}{36};
        \prone{O}{41}{37};
        \prone{O}{41}{38};
        \prone{O}{41}{39};
        \prone{O}{41}{40};
        \prone{O}{41}{41};


    \end{tikzpicture}
    \caption{An example of instance for which an optimal density is $O(\sqrt{n})$ times the density of a solution returned by the LCL algorithm.}
    \label{fig:lclExample}
\end{figure}

\renewcommand{\gridsize}{0.5}
\begin{figure}
    \centering

    \begin{tikzpicture}
        \coordinate (O) at (0,0);
        \prgrid{O}{17}{17}

\definecolor{tempcolor}{rgb}{0.0,0.0,0.0}
        \pronec{O}{1}{1}{tempcolor};
\definecolor{tempcolor}{rgb}{0.0,0.0,0.0}
        \pronec{O}{1}{2}{tempcolor};
\definecolor{tempcolor}{rgb}{0.0,0.0,0.0}
        \pronec{O}{1}{3}{tempcolor};
\definecolor{tempcolor}{rgb}{0.0,0.0,0.0}
        \pronec{O}{1}{4}{tempcolor};
\definecolor{tempcolor}{rgb}{0.0,0.3921568691730499,0.0}
        \pronec{O}{1}{14}{tempcolor};
\definecolor{tempcolor}{rgb}{0.0,0.3921568691730499,0.0}
        \pronec{O}{1}{16}{tempcolor};

\definecolor{tempcolor}{rgb}{0.0,0.0,0.0}
        \pronec{O}{2}{1}{tempcolor};
\definecolor{tempcolor}{rgb}{0.0,0.0,0.0}
        \pronec{O}{2}{2}{tempcolor};
\definecolor{tempcolor}{rgb}{0.0,0.0,0.0}
        \pronec{O}{2}{3}{tempcolor};
\definecolor{tempcolor}{rgb}{0.0,0.0,0.0}
        \pronec{O}{2}{4}{tempcolor};
\definecolor{tempcolor}{rgb}{0.0,0.0,1.0}
        \pronec{O}{2}{10}{tempcolor};
\definecolor{tempcolor}{rgb}{0.0,0.0,1.0}
        \pronec{O}{2}{12}{tempcolor};
\definecolor{tempcolor}{rgb}{0.0,0.3921568691730499,0.0}
        \pronec{O}{2}{14}{tempcolor};
\definecolor{tempcolor}{rgb}{0.0,0.3921568691730499,0.0}
        \pronec{O}{2}{16}{tempcolor};

\definecolor{tempcolor}{rgb}{0.0,0.0,0.0}
        \pronec{O}{3}{1}{tempcolor};
\definecolor{tempcolor}{rgb}{0.0,0.0,0.0}
        \pronec{O}{3}{2}{tempcolor};
\definecolor{tempcolor}{rgb}{0.0,0.0,0.0}
        \pronec{O}{3}{3}{tempcolor};
\definecolor{tempcolor}{rgb}{0.0,0.0,0.0}
        \pronec{O}{3}{4}{tempcolor};
\definecolor{tempcolor}{rgb}{1.0,0.0,0.0}
        \pronec{O}{3}{6}{tempcolor};
\definecolor{tempcolor}{rgb}{1.0,0.0,0.0}
        \pronec{O}{3}{8}{tempcolor};
\definecolor{tempcolor}{rgb}{0.0,0.0,1.0}
        \pronec{O}{3}{10}{tempcolor};
\definecolor{tempcolor}{rgb}{0.0,0.0,1.0}
        \pronec{O}{3}{12}{tempcolor};

\definecolor{tempcolor}{rgb}{0.0,0.0,0.0}
        \pronec{O}{4}{1}{tempcolor};
\definecolor{tempcolor}{rgb}{0.0,0.0,0.0}
        \pronec{O}{4}{2}{tempcolor};
\definecolor{tempcolor}{rgb}{0.0,0.0,0.0}
        \pronec{O}{4}{3}{tempcolor};
\definecolor{tempcolor}{rgb}{0.0,0.0,0.0}
        \pronec{O}{4}{4}{tempcolor};
\definecolor{tempcolor}{rgb}{1.0,0.0,0.0}
        \pronec{O}{4}{6}{tempcolor};
\definecolor{tempcolor}{rgb}{1.0,0.0,0.0}
        \pronec{O}{4}{8}{tempcolor};
\definecolor{tempcolor}{rgb}{0.0,0.0,0.0}
        \pronec{O}{4}{17}{tempcolor};

\definecolor{tempcolor}{rgb}{0.0,0.0,0.0}
        \pronec{O}{5}{17}{tempcolor};

\definecolor{tempcolor}{rgb}{1.0,0.0,0.0}
        \pronec{O}{6}{3}{tempcolor};
\definecolor{tempcolor}{rgb}{1.0,0.0,0.0}
        \pronec{O}{6}{4}{tempcolor};
\definecolor{tempcolor}{rgb}{0.0,0.0,0.0}
        \pronec{O}{6}{17}{tempcolor};

\definecolor{tempcolor}{rgb}{0.0,0.0,0.0}
        \pronec{O}{7}{17}{tempcolor};

\definecolor{tempcolor}{rgb}{1.0,0.0,0.0}
        \pronec{O}{8}{3}{tempcolor};
\definecolor{tempcolor}{rgb}{1.0,0.0,0.0}
        \pronec{O}{8}{4}{tempcolor};
\definecolor{tempcolor}{rgb}{0.0,0.0,0.0}
        \pronec{O}{8}{17}{tempcolor};

\definecolor{tempcolor}{rgb}{0.0,0.0,0.0}
        \pronec{O}{9}{17}{tempcolor};

\definecolor{tempcolor}{rgb}{0.0,0.0,1.0}
        \pronec{O}{10}{2}{tempcolor};
\definecolor{tempcolor}{rgb}{0.0,0.0,1.0}
        \pronec{O}{10}{3}{tempcolor};
\definecolor{tempcolor}{rgb}{0.0,0.0,0.0}
        \pronec{O}{10}{17}{tempcolor};

\definecolor{tempcolor}{rgb}{0.0,0.0,0.0}
        \pronec{O}{11}{17}{tempcolor};

\definecolor{tempcolor}{rgb}{0.0,0.0,1.0}
        \pronec{O}{12}{2}{tempcolor};
\definecolor{tempcolor}{rgb}{0.0,0.0,1.0}
        \pronec{O}{12}{3}{tempcolor};
\definecolor{tempcolor}{rgb}{0.0,0.0,0.0}
        \pronec{O}{12}{17}{tempcolor};

\definecolor{tempcolor}{rgb}{0.0,0.0,0.0}
        \pronec{O}{13}{17}{tempcolor};

\definecolor{tempcolor}{rgb}{0.0,0.3921568691730499,0.0}
        \pronec{O}{14}{1}{tempcolor};
\definecolor{tempcolor}{rgb}{0.0,0.3921568691730499,0.0}
        \pronec{O}{14}{2}{tempcolor};
\definecolor{tempcolor}{rgb}{0.0,0.0,0.0}
        \pronec{O}{14}{17}{tempcolor};

\definecolor{tempcolor}{rgb}{0.0,0.0,0.0}
        \pronec{O}{15}{17}{tempcolor};

\definecolor{tempcolor}{rgb}{0.0,0.3921568691730499,0.0}
        \pronec{O}{16}{1}{tempcolor};
\definecolor{tempcolor}{rgb}{0.0,0.3921568691730499,0.0}
        \pronec{O}{16}{2}{tempcolor};
\definecolor{tempcolor}{rgb}{0.0,0.0,0.0}
        \pronec{O}{16}{17}{tempcolor};

\definecolor{tempcolor}{rgb}{0.0,0.0,0.0}
        \pronec{O}{17}{4}{tempcolor};
\definecolor{tempcolor}{rgb}{0.0,0.0,0.0}
        \pronec{O}{17}{5}{tempcolor};
\definecolor{tempcolor}{rgb}{0.0,0.0,0.0}
        \pronec{O}{17}{6}{tempcolor};
\definecolor{tempcolor}{rgb}{0.0,0.0,0.0}
        \pronec{O}{17}{7}{tempcolor};
\definecolor{tempcolor}{rgb}{0.0,0.0,0.0}
        \pronec{O}{17}{8}{tempcolor};
\definecolor{tempcolor}{rgb}{0.0,0.0,0.0}
        \pronec{O}{17}{9}{tempcolor};
\definecolor{tempcolor}{rgb}{0.0,0.0,0.0}
        \pronec{O}{17}{10}{tempcolor};
\definecolor{tempcolor}{rgb}{0.0,0.0,0.0}
        \pronec{O}{17}{11}{tempcolor};
\definecolor{tempcolor}{rgb}{0.0,0.0,0.0}
        \pronec{O}{17}{12}{tempcolor};
\definecolor{tempcolor}{rgb}{0.0,0.0,0.0}
        \pronec{O}{17}{13}{tempcolor};
\definecolor{tempcolor}{rgb}{0.0,0.0,0.0}
        \pronec{O}{17}{14}{tempcolor};
\definecolor{tempcolor}{rgb}{0.0,0.0,0.0}
        \pronec{O}{17}{15}{tempcolor};
\definecolor{tempcolor}{rgb}{0.0,0.0,0.0}
        \pronec{O}{17}{16}{tempcolor};
\definecolor{tempcolor}{rgb}{0.0,0.0,0.0}
        \pronec{O}{17}{17}{tempcolor};

\draw ($(O)+(0,2.25)$) node[anchor=east] {$5$};
\draw ($(O)+(0,4.75)$) node[anchor=east] {$10$};
\draw ($(O)+(0,7.25)$) node[anchor=east] {$15$};
\draw ($(O)+(2.25,-0.3)$) node {$5$};
\draw ($(O)+(4.75,-0.3)$) node {$10$};
\draw ($(O)+(7.25,-0.3)$) node {$15$};

    \end{tikzpicture}
   \caption{An optimal solution of the instance given in Figure~\ref{fig:lclExample}. The density of this solution is $O(n)$.}
   \label{fig:lclExampleGood}
\end{figure}

\renewcommand{\gridsize}{0.5}
\begin{figure}
    \centering

    \begin{tikzpicture}
        \coordinate (O) at (0,0);
        \prgrid{O}{20}{20}

\definecolor{tempcolor}{rgb}{0.0,0.0,0.0}
        \pronec{O}{1}{1}{tempcolor};
\definecolor{tempcolor}{rgb}{0.0,0.0,0.0}
        \pronec{O}{1}{3}{tempcolor};
\definecolor{tempcolor}{rgb}{0.0,0.0,0.0}
        \pronec{O}{1}{5}{tempcolor};
\definecolor{tempcolor}{rgb}{0.0,0.0,0.0}
        \pronec{O}{1}{7}{tempcolor};
\definecolor{tempcolor}{rgb}{0.0,0.3921568691730499,0.0}
        \pronec{O}{1}{19}{tempcolor};

\definecolor{tempcolor}{rgb}{0.0,0.3921568691730499,0.0}
        \pronec{O}{2}{17}{tempcolor};
\definecolor{tempcolor}{rgb}{0.0,0.3921568691730499,0.0}
        \pronec{O}{2}{19}{tempcolor};

\definecolor{tempcolor}{rgb}{0.0,0.0,0.0}
        \pronec{O}{3}{1}{tempcolor};
\definecolor{tempcolor}{rgb}{0.0,0.0,0.0}
        \pronec{O}{3}{3}{tempcolor};
\definecolor{tempcolor}{rgb}{0.0,0.0,0.0}
        \pronec{O}{3}{5}{tempcolor};
\definecolor{tempcolor}{rgb}{0.0,0.0,0.0}
        \pronec{O}{3}{7}{tempcolor};
\definecolor{tempcolor}{rgb}{0.0,0.0,1.0}
        \pronec{O}{3}{15}{tempcolor};
\definecolor{tempcolor}{rgb}{0.0,0.3921568691730499,0.0}
        \pronec{O}{3}{17}{tempcolor};

\definecolor{tempcolor}{rgb}{0.0,0.0,1.0}
        \pronec{O}{4}{13}{tempcolor};
\definecolor{tempcolor}{rgb}{0.0,0.0,1.0}
        \pronec{O}{4}{15}{tempcolor};

\definecolor{tempcolor}{rgb}{0.0,0.0,0.0}
        \pronec{O}{5}{1}{tempcolor};
\definecolor{tempcolor}{rgb}{0.0,0.0,0.0}
        \pronec{O}{5}{3}{tempcolor};
\definecolor{tempcolor}{rgb}{0.0,0.0,0.0}
        \pronec{O}{5}{5}{tempcolor};
\definecolor{tempcolor}{rgb}{0.0,0.0,0.0}
        \pronec{O}{5}{7}{tempcolor};
\definecolor{tempcolor}{rgb}{1.0,0.0,0.0}
        \pronec{O}{5}{11}{tempcolor};
\definecolor{tempcolor}{rgb}{0.0,0.0,1.0}
        \pronec{O}{5}{13}{tempcolor};

\definecolor{tempcolor}{rgb}{1.0,0.0,0.0}
        \pronec{O}{6}{9}{tempcolor};
\definecolor{tempcolor}{rgb}{1.0,0.0,0.0}
        \pronec{O}{6}{11}{tempcolor};

\definecolor{tempcolor}{rgb}{0.0,0.0,0.0}
        \pronec{O}{7}{1}{tempcolor};
\definecolor{tempcolor}{rgb}{0.0,0.0,0.0}
        \pronec{O}{7}{3}{tempcolor};
\definecolor{tempcolor}{rgb}{0.0,0.0,0.0}
        \pronec{O}{7}{5}{tempcolor};
\definecolor{tempcolor}{rgb}{0.0,0.0,0.0}
        \pronec{O}{7}{7}{tempcolor};
\definecolor{tempcolor}{rgb}{1.0,0.0,0.0}
        \pronec{O}{7}{9}{tempcolor};
\definecolor{tempcolor}{rgb}{0.0,0.0,0.0}
        \pronec{O}{7}{20}{tempcolor};

\definecolor{tempcolor}{rgb}{0.0,0.0,0.0}
        \pronec{O}{8}{20}{tempcolor};

\definecolor{tempcolor}{rgb}{1.0,0.0,0.0}
        \pronec{O}{9}{6}{tempcolor};
\definecolor{tempcolor}{rgb}{1.0,0.0,0.0}
        \pronec{O}{9}{7}{tempcolor};
\definecolor{tempcolor}{rgb}{0.0,0.0,0.0}
        \pronec{O}{9}{20}{tempcolor};

\definecolor{tempcolor}{rgb}{0.0,0.0,0.0}
        \pronec{O}{10}{20}{tempcolor};

\definecolor{tempcolor}{rgb}{1.0,0.0,0.0}
        \pronec{O}{11}{5}{tempcolor};
\definecolor{tempcolor}{rgb}{1.0,0.0,0.0}
        \pronec{O}{11}{6}{tempcolor};
\definecolor{tempcolor}{rgb}{0.0,0.0,0.0}
        \pronec{O}{11}{20}{tempcolor};

\definecolor{tempcolor}{rgb}{0.0,0.0,0.0}
        \pronec{O}{12}{20}{tempcolor};

\definecolor{tempcolor}{rgb}{0.0,0.0,1.0}
        \pronec{O}{13}{4}{tempcolor};
\definecolor{tempcolor}{rgb}{0.0,0.0,1.0}
        \pronec{O}{13}{5}{tempcolor};
\definecolor{tempcolor}{rgb}{0.0,0.0,0.0}
        \pronec{O}{13}{20}{tempcolor};

\definecolor{tempcolor}{rgb}{0.0,0.0,0.0}
        \pronec{O}{14}{20}{tempcolor};

\definecolor{tempcolor}{rgb}{0.0,0.0,1.0}
        \pronec{O}{15}{3}{tempcolor};
\definecolor{tempcolor}{rgb}{0.0,0.0,1.0}
        \pronec{O}{15}{4}{tempcolor};
\definecolor{tempcolor}{rgb}{0.0,0.0,0.0}
        \pronec{O}{15}{20}{tempcolor};

\definecolor{tempcolor}{rgb}{0.0,0.0,0.0}
        \pronec{O}{16}{20}{tempcolor};

\definecolor{tempcolor}{rgb}{0.0,0.3921568691730499,0.0}
        \pronec{O}{17}{2}{tempcolor};
\definecolor{tempcolor}{rgb}{0.0,0.3921568691730499,0.0}
        \pronec{O}{17}{3}{tempcolor};
\definecolor{tempcolor}{rgb}{0.0,0.0,0.0}
        \pronec{O}{17}{20}{tempcolor};

\definecolor{tempcolor}{rgb}{0.0,0.0,0.0}
        \pronec{O}{18}{20}{tempcolor};

\definecolor{tempcolor}{rgb}{0.0,0.3921568691730499,0.0}
        \pronec{O}{19}{1}{tempcolor};
\definecolor{tempcolor}{rgb}{0.0,0.3921568691730499,0.0}
        \pronec{O}{19}{2}{tempcolor};
\definecolor{tempcolor}{rgb}{0.0,0.0,0.0}
        \pronec{O}{19}{20}{tempcolor};

\definecolor{tempcolor}{rgb}{0.0,0.0,0.0}
        \pronec{O}{20}{7}{tempcolor};
\definecolor{tempcolor}{rgb}{0.0,0.0,0.0}
        \pronec{O}{20}{8}{tempcolor};
\definecolor{tempcolor}{rgb}{0.0,0.0,0.0}
        \pronec{O}{20}{9}{tempcolor};
\definecolor{tempcolor}{rgb}{0.0,0.0,0.0}
        \pronec{O}{20}{10}{tempcolor};
\definecolor{tempcolor}{rgb}{0.0,0.0,0.0}
        \pronec{O}{20}{11}{tempcolor};
\definecolor{tempcolor}{rgb}{0.0,0.0,0.0}
        \pronec{O}{20}{12}{tempcolor};
\definecolor{tempcolor}{rgb}{0.0,0.0,0.0}
        \pronec{O}{20}{13}{tempcolor};
\definecolor{tempcolor}{rgb}{0.0,0.0,0.0}
        \pronec{O}{20}{14}{tempcolor};
\definecolor{tempcolor}{rgb}{0.0,0.0,0.0}
        \pronec{O}{20}{15}{tempcolor};
\definecolor{tempcolor}{rgb}{0.0,0.0,0.0}
        \pronec{O}{20}{16}{tempcolor};
\definecolor{tempcolor}{rgb}{0.0,0.0,0.0}
        \pronec{O}{20}{17}{tempcolor};
\definecolor{tempcolor}{rgb}{0.0,0.0,0.0}
        \pronec{O}{20}{18}{tempcolor};
\definecolor{tempcolor}{rgb}{0.0,0.0,0.0}
        \pronec{O}{20}{19}{tempcolor};
\definecolor{tempcolor}{rgb}{0.0,0.0,0.0}
        \pronec{O}{20}{20}{tempcolor};

\draw ($(O)+(0,2.25)$) node[anchor=east] {$5$};
\draw ($(O)+(0,4.75)$) node[anchor=east] {$10$};
\draw ($(O)+(0,7.25)$) node[anchor=east] {$15$};
\draw ($(O)+(0,9.75)$) node[anchor=east] {$20$};
\draw ($(O)+(2.25,-0.3)$) node {$5$};
\draw ($(O)+(4.75,-0.3)$) node {$10$};
\draw ($(O)+(7.25,-0.3)$) node {$15$};
\draw ($(O)+(9.75,-0.3)$) node {$20$};

    \end{tikzpicture}
    \caption{A solution of the instance given in Figure~\ref{fig:lclExample} that may be returned by the LCL algorithm. The density of this solution is $O(\sqrt{n})$.}
    \label{fig:lclExampleBad}
\end{figure}


\subsection{Adaptation to the greedy algorithm}
\renewcommand{\gridsize}{0.5}
\begin{figure}
    \centering

    \begin{tikzpicture}
        \coordinate (O) at (0,0);
        \prgrid{O}{25}{25}

\definecolor{tempcolor}{rgb}{0.0,0.0,1.0}
        \pronec{O}{1}{1}{tempcolor};
\definecolor{tempcolor}{rgb}{0.0,0.0,1.0}
        \pronec{O}{1}{3}{tempcolor};
\definecolor{tempcolor}{rgb}{0.0,0.0,1.0}
        \pronec{O}{1}{5}{tempcolor};
\definecolor{tempcolor}{rgb}{0.0,0.0,1.0}
        \pronec{O}{1}{7}{tempcolor};
\definecolor{tempcolor}{rgb}{0.0,0.0,1.0}
        \pronec{O}{1}{9}{tempcolor};
\definecolor{tempcolor}{rgb}{0.0,0.0,1.0}
        \pronec{O}{1}{11}{tempcolor};
\definecolor{tempcolor}{rgb}{1.0,0.0,0.0}
        \pronec{O}{1}{14}{tempcolor};

\definecolor{tempcolor}{rgb}{1.0,0.0,0.0}
        \pronec{O}{2}{14}{tempcolor};

\definecolor{tempcolor}{rgb}{0.0,0.0,1.0}
        \pronec{O}{3}{1}{tempcolor};
\definecolor{tempcolor}{rgb}{0.0,0.0,1.0}
        \pronec{O}{3}{3}{tempcolor};
\definecolor{tempcolor}{rgb}{0.0,0.0,1.0}
        \pronec{O}{3}{5}{tempcolor};
\definecolor{tempcolor}{rgb}{0.0,0.0,1.0}
        \pronec{O}{3}{7}{tempcolor};
\definecolor{tempcolor}{rgb}{0.0,0.0,1.0}
        \pronec{O}{3}{9}{tempcolor};
\definecolor{tempcolor}{rgb}{0.0,0.0,1.0}
        \pronec{O}{3}{11}{tempcolor};
\definecolor{tempcolor}{rgb}{1.0,0.0,0.0}
        \pronec{O}{3}{13}{tempcolor};
\definecolor{tempcolor}{rgb}{1.0,0.0,0.0}
        \pronec{O}{3}{17}{tempcolor};

\definecolor{tempcolor}{rgb}{1.0,0.0,0.0}
        \pronec{O}{4}{17}{tempcolor};

\definecolor{tempcolor}{rgb}{0.0,0.0,1.0}
        \pronec{O}{5}{1}{tempcolor};
\definecolor{tempcolor}{rgb}{0.0,0.0,1.0}
        \pronec{O}{5}{3}{tempcolor};
\definecolor{tempcolor}{rgb}{0.0,0.0,1.0}
        \pronec{O}{5}{5}{tempcolor};
\definecolor{tempcolor}{rgb}{0.0,0.0,1.0}
        \pronec{O}{5}{7}{tempcolor};
\definecolor{tempcolor}{rgb}{0.0,0.0,1.0}
        \pronec{O}{5}{9}{tempcolor};
\definecolor{tempcolor}{rgb}{0.0,0.0,1.0}
        \pronec{O}{5}{11}{tempcolor};
\definecolor{tempcolor}{rgb}{1.0,0.0,0.0}
        \pronec{O}{5}{14}{tempcolor};
\definecolor{tempcolor}{rgb}{1.0,0.0,0.0}
        \pronec{O}{5}{16}{tempcolor};

\definecolor{tempcolor}{rgb}{1.0,0.0,0.0}
        \pronec{O}{6}{14}{tempcolor};

\definecolor{tempcolor}{rgb}{0.0,0.0,1.0}
        \pronec{O}{7}{1}{tempcolor};
\definecolor{tempcolor}{rgb}{0.0,0.0,1.0}
        \pronec{O}{7}{3}{tempcolor};
\definecolor{tempcolor}{rgb}{0.0,0.0,1.0}
        \pronec{O}{7}{5}{tempcolor};
\definecolor{tempcolor}{rgb}{0.0,0.0,1.0}
        \pronec{O}{7}{7}{tempcolor};
\definecolor{tempcolor}{rgb}{0.0,0.0,1.0}
        \pronec{O}{7}{9}{tempcolor};
\definecolor{tempcolor}{rgb}{0.0,0.0,1.0}
        \pronec{O}{7}{11}{tempcolor};
\definecolor{tempcolor}{rgb}{1.0,0.0,0.0}
        \pronec{O}{7}{13}{tempcolor};
\definecolor{tempcolor}{rgb}{1.0,0.0,0.0}
        \pronec{O}{7}{17}{tempcolor};

\definecolor{tempcolor}{rgb}{1.0,0.0,0.0}
        \pronec{O}{8}{17}{tempcolor};

\definecolor{tempcolor}{rgb}{0.0,0.0,1.0}
        \pronec{O}{9}{1}{tempcolor};
\definecolor{tempcolor}{rgb}{0.0,0.0,1.0}
        \pronec{O}{9}{3}{tempcolor};
\definecolor{tempcolor}{rgb}{0.0,0.0,1.0}
        \pronec{O}{9}{5}{tempcolor};
\definecolor{tempcolor}{rgb}{0.0,0.0,1.0}
        \pronec{O}{9}{7}{tempcolor};
\definecolor{tempcolor}{rgb}{0.0,0.0,1.0}
        \pronec{O}{9}{9}{tempcolor};
\definecolor{tempcolor}{rgb}{0.0,0.0,1.0}
        \pronec{O}{9}{11}{tempcolor};
\definecolor{tempcolor}{rgb}{1.0,0.0,0.0}
        \pronec{O}{9}{14}{tempcolor};
\definecolor{tempcolor}{rgb}{1.0,0.0,0.0}
        \pronec{O}{9}{16}{tempcolor};

\definecolor{tempcolor}{rgb}{1.0,0.0,0.0}
        \pronec{O}{10}{14}{tempcolor};

\definecolor{tempcolor}{rgb}{0.0,0.0,1.0}
        \pronec{O}{11}{1}{tempcolor};
\definecolor{tempcolor}{rgb}{0.0,0.0,1.0}
        \pronec{O}{11}{3}{tempcolor};
\definecolor{tempcolor}{rgb}{0.0,0.0,1.0}
        \pronec{O}{11}{5}{tempcolor};
\definecolor{tempcolor}{rgb}{0.0,0.0,1.0}
        \pronec{O}{11}{7}{tempcolor};
\definecolor{tempcolor}{rgb}{0.0,0.0,1.0}
        \pronec{O}{11}{9}{tempcolor};
\definecolor{tempcolor}{rgb}{0.0,0.0,1.0}
        \pronec{O}{11}{11}{tempcolor};
\definecolor{tempcolor}{rgb}{1.0,0.0,0.0}
        \pronec{O}{11}{13}{tempcolor};
\definecolor{tempcolor}{rgb}{0.0,0.3921568691730499,0.0}
        \pronec{O}{11}{20}{tempcolor};

\definecolor{tempcolor}{rgb}{0.0,0.3921568691730499,0.0}
        \pronec{O}{12}{20}{tempcolor};
\definecolor{tempcolor}{rgb}{0.0,0.3921568691730499,0.0}
        \pronec{O}{12}{24}{tempcolor};
\definecolor{tempcolor}{rgb}{0.0,0.3921568691730499,0.0}
        \pronec{O}{12}{25}{tempcolor};

\definecolor{tempcolor}{rgb}{1.0,0.0,0.0}
        \pronec{O}{13}{3}{tempcolor};
\definecolor{tempcolor}{rgb}{1.0,0.0,0.0}
        \pronec{O}{13}{7}{tempcolor};
\definecolor{tempcolor}{rgb}{1.0,0.0,0.0}
        \pronec{O}{13}{11}{tempcolor};
\definecolor{tempcolor}{rgb}{0.0,0.3921568691730499,0.0}
        \pronec{O}{13}{20}{tempcolor};
\definecolor{tempcolor}{rgb}{0.0,0.3921568691730499,0.0}
        \pronec{O}{13}{21}{tempcolor};
\definecolor{tempcolor}{rgb}{0.0,0.3921568691730499,0.0}
        \pronec{O}{13}{22}{tempcolor};
\definecolor{tempcolor}{rgb}{0.0,0.3921568691730499,0.0}
        \pronec{O}{13}{23}{tempcolor};
\definecolor{tempcolor}{rgb}{0.0,0.3921568691730499,0.0}
        \pronec{O}{13}{24}{tempcolor};

\definecolor{tempcolor}{rgb}{1.0,0.0,0.0}
        \pronec{O}{14}{1}{tempcolor};
\definecolor{tempcolor}{rgb}{1.0,0.0,0.0}
        \pronec{O}{14}{2}{tempcolor};
\definecolor{tempcolor}{rgb}{1.0,0.0,0.0}
        \pronec{O}{14}{5}{tempcolor};
\definecolor{tempcolor}{rgb}{1.0,0.0,0.0}
        \pronec{O}{14}{6}{tempcolor};
\definecolor{tempcolor}{rgb}{1.0,0.0,0.0}
        \pronec{O}{14}{9}{tempcolor};
\definecolor{tempcolor}{rgb}{1.0,0.0,0.0}
        \pronec{O}{14}{10}{tempcolor};
\definecolor{tempcolor}{rgb}{0.0,0.3921568691730499,0.0}
        \pronec{O}{14}{25}{tempcolor};

\definecolor{tempcolor}{rgb}{0.0,0.3921568691730499,0.0}
        \pronec{O}{15}{20}{tempcolor};
\definecolor{tempcolor}{rgb}{0.0,0.3921568691730499,0.0}
        \pronec{O}{15}{21}{tempcolor};
\definecolor{tempcolor}{rgb}{0.0,0.3921568691730499,0.0}
        \pronec{O}{15}{22}{tempcolor};
\definecolor{tempcolor}{rgb}{0.0,0.3921568691730499,0.0}
        \pronec{O}{15}{23}{tempcolor};
\definecolor{tempcolor}{rgb}{0.0,0.3921568691730499,0.0}
        \pronec{O}{15}{24}{tempcolor};
\definecolor{tempcolor}{rgb}{0.0,0.3921568691730499,0.0}
        \pronec{O}{15}{25}{tempcolor};

\definecolor{tempcolor}{rgb}{1.0,0.0,0.0}
        \pronec{O}{16}{5}{tempcolor};
\definecolor{tempcolor}{rgb}{1.0,0.0,0.0}
        \pronec{O}{16}{9}{tempcolor};
\definecolor{tempcolor}{rgb}{0.0,0.3921568691730499,0.0}
        \pronec{O}{16}{20}{tempcolor};
\definecolor{tempcolor}{rgb}{0.0,0.3921568691730499,0.0}
        \pronec{O}{16}{21}{tempcolor};
\definecolor{tempcolor}{rgb}{0.0,0.3921568691730499,0.0}
        \pronec{O}{16}{22}{tempcolor};
\definecolor{tempcolor}{rgb}{0.0,0.3921568691730499,0.0}
        \pronec{O}{16}{23}{tempcolor};
\definecolor{tempcolor}{rgb}{0.0,0.3921568691730499,0.0}
        \pronec{O}{16}{24}{tempcolor};

\definecolor{tempcolor}{rgb}{1.0,0.0,0.0}
        \pronec{O}{17}{3}{tempcolor};
\definecolor{tempcolor}{rgb}{1.0,0.0,0.0}
        \pronec{O}{17}{4}{tempcolor};
\definecolor{tempcolor}{rgb}{1.0,0.0,0.0}
        \pronec{O}{17}{7}{tempcolor};
\definecolor{tempcolor}{rgb}{1.0,0.0,0.0}
        \pronec{O}{17}{8}{tempcolor};
\definecolor{tempcolor}{rgb}{0.0,0.3921568691730499,0.0}
        \pronec{O}{17}{25}{tempcolor};

\definecolor{tempcolor}{rgb}{0.0,0.3921568691730499,0.0}
        \pronec{O}{18}{20}{tempcolor};
\definecolor{tempcolor}{rgb}{0.0,0.3921568691730499,0.0}
        \pronec{O}{18}{21}{tempcolor};
\definecolor{tempcolor}{rgb}{0.0,0.3921568691730499,0.0}
        \pronec{O}{18}{22}{tempcolor};
\definecolor{tempcolor}{rgb}{0.0,0.3921568691730499,0.0}
        \pronec{O}{18}{23}{tempcolor};
\definecolor{tempcolor}{rgb}{0.0,0.3921568691730499,0.0}
        \pronec{O}{18}{24}{tempcolor};
\definecolor{tempcolor}{rgb}{0.0,0.3921568691730499,0.0}
        \pronec{O}{18}{25}{tempcolor};

\definecolor{tempcolor}{rgb}{0.0,0.3921568691730499,0.0}
        \pronec{O}{19}{20}{tempcolor};

\definecolor{tempcolor}{rgb}{0.0,0.3921568691730499,0.0}
        \pronec{O}{20}{11}{tempcolor};
\definecolor{tempcolor}{rgb}{0.0,0.3921568691730499,0.0}
        \pronec{O}{20}{12}{tempcolor};
\definecolor{tempcolor}{rgb}{0.0,0.3921568691730499,0.0}
        \pronec{O}{20}{13}{tempcolor};
\definecolor{tempcolor}{rgb}{0.0,0.3921568691730499,0.0}
        \pronec{O}{20}{15}{tempcolor};
\definecolor{tempcolor}{rgb}{0.0,0.3921568691730499,0.0}
        \pronec{O}{20}{16}{tempcolor};
\definecolor{tempcolor}{rgb}{0.0,0.3921568691730499,0.0}
        \pronec{O}{20}{18}{tempcolor};
\definecolor{tempcolor}{rgb}{0.0,0.3921568691730499,0.0}
        \pronec{O}{20}{19}{tempcolor};
\definecolor{tempcolor}{rgb}{0.0,0.3921568691730499,0.0}
        \pronec{O}{20}{20}{tempcolor};

\definecolor{tempcolor}{rgb}{0.0,0.3921568691730499,0.0}
        \pronec{O}{21}{13}{tempcolor};
\definecolor{tempcolor}{rgb}{0.0,0.3921568691730499,0.0}
        \pronec{O}{21}{15}{tempcolor};
\definecolor{tempcolor}{rgb}{0.0,0.3921568691730499,0.0}
        \pronec{O}{21}{16}{tempcolor};
\definecolor{tempcolor}{rgb}{0.0,0.3921568691730499,0.0}
        \pronec{O}{21}{18}{tempcolor};

\definecolor{tempcolor}{rgb}{0.0,0.3921568691730499,0.0}
        \pronec{O}{22}{13}{tempcolor};
\definecolor{tempcolor}{rgb}{0.0,0.3921568691730499,0.0}
        \pronec{O}{22}{15}{tempcolor};
\definecolor{tempcolor}{rgb}{0.0,0.3921568691730499,0.0}
        \pronec{O}{22}{16}{tempcolor};
\definecolor{tempcolor}{rgb}{0.0,0.3921568691730499,0.0}
        \pronec{O}{22}{18}{tempcolor};

\definecolor{tempcolor}{rgb}{0.0,0.3921568691730499,0.0}
        \pronec{O}{23}{13}{tempcolor};
\definecolor{tempcolor}{rgb}{0.0,0.3921568691730499,0.0}
        \pronec{O}{23}{15}{tempcolor};
\definecolor{tempcolor}{rgb}{0.0,0.3921568691730499,0.0}
        \pronec{O}{23}{16}{tempcolor};
\definecolor{tempcolor}{rgb}{0.0,0.3921568691730499,0.0}
        \pronec{O}{23}{18}{tempcolor};

\definecolor{tempcolor}{rgb}{0.0,0.3921568691730499,0.0}
        \pronec{O}{24}{12}{tempcolor};
\definecolor{tempcolor}{rgb}{0.0,0.3921568691730499,0.0}
        \pronec{O}{24}{13}{tempcolor};
\definecolor{tempcolor}{rgb}{0.0,0.3921568691730499,0.0}
        \pronec{O}{24}{15}{tempcolor};
\definecolor{tempcolor}{rgb}{0.0,0.3921568691730499,0.0}
        \pronec{O}{24}{16}{tempcolor};
\definecolor{tempcolor}{rgb}{0.0,0.3921568691730499,0.0}
        \pronec{O}{24}{18}{tempcolor};

\definecolor{tempcolor}{rgb}{0.0,0.3921568691730499,0.0}
        \pronec{O}{25}{12}{tempcolor};
\definecolor{tempcolor}{rgb}{0.0,0.3921568691730499,0.0}
        \pronec{O}{25}{14}{tempcolor};
\definecolor{tempcolor}{rgb}{0.0,0.3921568691730499,0.0}
        \pronec{O}{25}{15}{tempcolor};
\definecolor{tempcolor}{rgb}{0.0,0.3921568691730499,0.0}
        \pronec{O}{25}{17}{tempcolor};
\definecolor{tempcolor}{rgb}{0.0,0.3921568691730499,0.0}
        \pronec{O}{25}{18}{tempcolor};

\draw ($(O)+(0,2.25)$) node[anchor=east] {$5$};
\draw ($(O)+(0,4.75)$) node[anchor=east] {$10$};
\draw ($(O)+(0,7.25)$) node[anchor=east] {$15$};
\draw ($(O)+(0,9.75)$) node[anchor=east] {$20$};
\draw ($(O)+(0,12.25)$) node[anchor=east] {$25$};
\draw ($(O)+(2.25,-0.3)$) node {$5$};
\draw ($(O)+(4.75,-0.3)$) node {$10$};
\draw ($(O)+(7.25,-0.3)$) node {$15$};
\draw ($(O)+(9.75,-0.3)$) node {$20$};
\draw ($(O)+(12.25,-0.3)$) node {$25$};

    \end{tikzpicture}
    \caption{An example of instance for which an optimal density is $O(\sqrt{n})$ times the density of a solution returned by the Greedy algorithm.}
    \label{fig:greedyExample}
\end{figure}

\begin{figure}
    \centering

    \begin{tikzpicture}
        \coordinate (O) at (0,0);
        \prgrid{O}{20}{20}

\definecolor{tempcolor}{rgb}{0.0,0.0,1.0}
        \pronec{O}{1}{1}{tempcolor};
\definecolor{tempcolor}{rgb}{0.0,0.0,1.0}
        \pronec{O}{1}{2}{tempcolor};
\definecolor{tempcolor}{rgb}{0.0,0.0,1.0}
        \pronec{O}{1}{3}{tempcolor};
\definecolor{tempcolor}{rgb}{0.0,0.0,1.0}
        \pronec{O}{1}{4}{tempcolor};
\definecolor{tempcolor}{rgb}{0.0,0.0,1.0}
        \pronec{O}{1}{5}{tempcolor};
\definecolor{tempcolor}{rgb}{0.0,0.0,1.0}
        \pronec{O}{1}{6}{tempcolor};
\definecolor{tempcolor}{rgb}{1.0,0.0,0.0}
        \pronec{O}{1}{9}{tempcolor};

\definecolor{tempcolor}{rgb}{0.0,0.0,1.0}
        \pronec{O}{2}{1}{tempcolor};
\definecolor{tempcolor}{rgb}{0.0,0.0,1.0}
        \pronec{O}{2}{2}{tempcolor};
\definecolor{tempcolor}{rgb}{0.0,0.0,1.0}
        \pronec{O}{2}{3}{tempcolor};
\definecolor{tempcolor}{rgb}{0.0,0.0,1.0}
        \pronec{O}{2}{4}{tempcolor};
\definecolor{tempcolor}{rgb}{0.0,0.0,1.0}
        \pronec{O}{2}{5}{tempcolor};
\definecolor{tempcolor}{rgb}{0.0,0.0,1.0}
        \pronec{O}{2}{6}{tempcolor};
\definecolor{tempcolor}{rgb}{1.0,0.0,0.0}
        \pronec{O}{2}{8}{tempcolor};
\definecolor{tempcolor}{rgb}{1.0,0.0,0.0}
        \pronec{O}{2}{9}{tempcolor};
\definecolor{tempcolor}{rgb}{1.0,0.0,0.0}
        \pronec{O}{2}{12}{tempcolor};

\definecolor{tempcolor}{rgb}{0.0,0.0,1.0}
        \pronec{O}{3}{1}{tempcolor};
\definecolor{tempcolor}{rgb}{0.0,0.0,1.0}
        \pronec{O}{3}{2}{tempcolor};
\definecolor{tempcolor}{rgb}{0.0,0.0,1.0}
        \pronec{O}{3}{3}{tempcolor};
\definecolor{tempcolor}{rgb}{0.0,0.0,1.0}
        \pronec{O}{3}{4}{tempcolor};
\definecolor{tempcolor}{rgb}{0.0,0.0,1.0}
        \pronec{O}{3}{5}{tempcolor};
\definecolor{tempcolor}{rgb}{0.0,0.0,1.0}
        \pronec{O}{3}{6}{tempcolor};
\definecolor{tempcolor}{rgb}{1.0,0.0,0.0}
        \pronec{O}{3}{9}{tempcolor};
\definecolor{tempcolor}{rgb}{1.0,0.0,0.0}
        \pronec{O}{3}{11}{tempcolor};
\definecolor{tempcolor}{rgb}{1.0,0.0,0.0}
        \pronec{O}{3}{12}{tempcolor};

\definecolor{tempcolor}{rgb}{0.0,0.0,1.0}
        \pronec{O}{4}{1}{tempcolor};
\definecolor{tempcolor}{rgb}{0.0,0.0,1.0}
        \pronec{O}{4}{2}{tempcolor};
\definecolor{tempcolor}{rgb}{0.0,0.0,1.0}
        \pronec{O}{4}{3}{tempcolor};
\definecolor{tempcolor}{rgb}{0.0,0.0,1.0}
        \pronec{O}{4}{4}{tempcolor};
\definecolor{tempcolor}{rgb}{0.0,0.0,1.0}
        \pronec{O}{4}{5}{tempcolor};
\definecolor{tempcolor}{rgb}{0.0,0.0,1.0}
        \pronec{O}{4}{6}{tempcolor};
\definecolor{tempcolor}{rgb}{1.0,0.0,0.0}
        \pronec{O}{4}{8}{tempcolor};
\definecolor{tempcolor}{rgb}{1.0,0.0,0.0}
        \pronec{O}{4}{9}{tempcolor};
\definecolor{tempcolor}{rgb}{1.0,0.0,0.0}
        \pronec{O}{4}{12}{tempcolor};

\definecolor{tempcolor}{rgb}{0.0,0.0,1.0}
        \pronec{O}{5}{1}{tempcolor};
\definecolor{tempcolor}{rgb}{0.0,0.0,1.0}
        \pronec{O}{5}{2}{tempcolor};
\definecolor{tempcolor}{rgb}{0.0,0.0,1.0}
        \pronec{O}{5}{3}{tempcolor};
\definecolor{tempcolor}{rgb}{0.0,0.0,1.0}
        \pronec{O}{5}{4}{tempcolor};
\definecolor{tempcolor}{rgb}{0.0,0.0,1.0}
        \pronec{O}{5}{5}{tempcolor};
\definecolor{tempcolor}{rgb}{0.0,0.0,1.0}
        \pronec{O}{5}{6}{tempcolor};
\definecolor{tempcolor}{rgb}{1.0,0.0,0.0}
        \pronec{O}{5}{9}{tempcolor};
\definecolor{tempcolor}{rgb}{1.0,0.0,0.0}
        \pronec{O}{5}{11}{tempcolor};
\definecolor{tempcolor}{rgb}{1.0,0.0,0.0}
        \pronec{O}{5}{12}{tempcolor};

\definecolor{tempcolor}{rgb}{0.0,0.0,1.0}
        \pronec{O}{6}{1}{tempcolor};
\definecolor{tempcolor}{rgb}{0.0,0.0,1.0}
        \pronec{O}{6}{2}{tempcolor};
\definecolor{tempcolor}{rgb}{0.0,0.0,1.0}
        \pronec{O}{6}{3}{tempcolor};
\definecolor{tempcolor}{rgb}{0.0,0.0,1.0}
        \pronec{O}{6}{4}{tempcolor};
\definecolor{tempcolor}{rgb}{0.0,0.0,1.0}
        \pronec{O}{6}{5}{tempcolor};
\definecolor{tempcolor}{rgb}{0.0,0.0,1.0}
        \pronec{O}{6}{6}{tempcolor};
\definecolor{tempcolor}{rgb}{1.0,0.0,0.0}
        \pronec{O}{6}{8}{tempcolor};
\definecolor{tempcolor}{rgb}{1.0,0.0,0.0}
        \pronec{O}{6}{9}{tempcolor};
\definecolor{tempcolor}{rgb}{0.0,0.3921568691730499,0.0}
        \pronec{O}{6}{15}{tempcolor};

\definecolor{tempcolor}{rgb}{0.0,0.3921568691730499,0.0}
        \pronec{O}{7}{15}{tempcolor};
\definecolor{tempcolor}{rgb}{0.0,0.3921568691730499,0.0}
        \pronec{O}{7}{19}{tempcolor};
\definecolor{tempcolor}{rgb}{0.0,0.3921568691730499,0.0}
        \pronec{O}{7}{20}{tempcolor};

\definecolor{tempcolor}{rgb}{1.0,0.0,0.0}
        \pronec{O}{8}{2}{tempcolor};
\definecolor{tempcolor}{rgb}{1.0,0.0,0.0}
        \pronec{O}{8}{4}{tempcolor};
\definecolor{tempcolor}{rgb}{1.0,0.0,0.0}
        \pronec{O}{8}{6}{tempcolor};
\definecolor{tempcolor}{rgb}{0.0,0.3921568691730499,0.0}
        \pronec{O}{8}{15}{tempcolor};
\definecolor{tempcolor}{rgb}{0.0,0.3921568691730499,0.0}
        \pronec{O}{8}{16}{tempcolor};
\definecolor{tempcolor}{rgb}{0.0,0.3921568691730499,0.0}
        \pronec{O}{8}{17}{tempcolor};
\definecolor{tempcolor}{rgb}{0.0,0.3921568691730499,0.0}
        \pronec{O}{8}{18}{tempcolor};
\definecolor{tempcolor}{rgb}{0.0,0.3921568691730499,0.0}
        \pronec{O}{8}{19}{tempcolor};

\definecolor{tempcolor}{rgb}{1.0,0.0,0.0}
        \pronec{O}{9}{1}{tempcolor};
\definecolor{tempcolor}{rgb}{1.0,0.0,0.0}
        \pronec{O}{9}{2}{tempcolor};
\definecolor{tempcolor}{rgb}{1.0,0.0,0.0}
        \pronec{O}{9}{3}{tempcolor};
\definecolor{tempcolor}{rgb}{1.0,0.0,0.0}
        \pronec{O}{9}{4}{tempcolor};
\definecolor{tempcolor}{rgb}{1.0,0.0,0.0}
        \pronec{O}{9}{5}{tempcolor};
\definecolor{tempcolor}{rgb}{1.0,0.0,0.0}
        \pronec{O}{9}{6}{tempcolor};
\definecolor{tempcolor}{rgb}{0.0,0.3921568691730499,0.0}
        \pronec{O}{9}{20}{tempcolor};

\definecolor{tempcolor}{rgb}{0.0,0.3921568691730499,0.0}
        \pronec{O}{10}{15}{tempcolor};
\definecolor{tempcolor}{rgb}{0.0,0.3921568691730499,0.0}
        \pronec{O}{10}{16}{tempcolor};
\definecolor{tempcolor}{rgb}{0.0,0.3921568691730499,0.0}
        \pronec{O}{10}{17}{tempcolor};
\definecolor{tempcolor}{rgb}{0.0,0.3921568691730499,0.0}
        \pronec{O}{10}{18}{tempcolor};
\definecolor{tempcolor}{rgb}{0.0,0.3921568691730499,0.0}
        \pronec{O}{10}{19}{tempcolor};
\definecolor{tempcolor}{rgb}{0.0,0.3921568691730499,0.0}
        \pronec{O}{10}{20}{tempcolor};

\definecolor{tempcolor}{rgb}{1.0,0.0,0.0}
        \pronec{O}{11}{3}{tempcolor};
\definecolor{tempcolor}{rgb}{1.0,0.0,0.0}
        \pronec{O}{11}{5}{tempcolor};
\definecolor{tempcolor}{rgb}{0.0,0.3921568691730499,0.0}
        \pronec{O}{11}{15}{tempcolor};
\definecolor{tempcolor}{rgb}{0.0,0.3921568691730499,0.0}
        \pronec{O}{11}{16}{tempcolor};
\definecolor{tempcolor}{rgb}{0.0,0.3921568691730499,0.0}
        \pronec{O}{11}{17}{tempcolor};
\definecolor{tempcolor}{rgb}{0.0,0.3921568691730499,0.0}
        \pronec{O}{11}{18}{tempcolor};
\definecolor{tempcolor}{rgb}{0.0,0.3921568691730499,0.0}
        \pronec{O}{11}{19}{tempcolor};

\definecolor{tempcolor}{rgb}{1.0,0.0,0.0}
        \pronec{O}{12}{2}{tempcolor};
\definecolor{tempcolor}{rgb}{1.0,0.0,0.0}
        \pronec{O}{12}{3}{tempcolor};
\definecolor{tempcolor}{rgb}{1.0,0.0,0.0}
        \pronec{O}{12}{4}{tempcolor};
\definecolor{tempcolor}{rgb}{1.0,0.0,0.0}
        \pronec{O}{12}{5}{tempcolor};
\definecolor{tempcolor}{rgb}{0.0,0.3921568691730499,0.0}
        \pronec{O}{12}{20}{tempcolor};

\definecolor{tempcolor}{rgb}{0.0,0.3921568691730499,0.0}
        \pronec{O}{13}{15}{tempcolor};
\definecolor{tempcolor}{rgb}{0.0,0.3921568691730499,0.0}
        \pronec{O}{13}{16}{tempcolor};
\definecolor{tempcolor}{rgb}{0.0,0.3921568691730499,0.0}
        \pronec{O}{13}{17}{tempcolor};
\definecolor{tempcolor}{rgb}{0.0,0.3921568691730499,0.0}
        \pronec{O}{13}{18}{tempcolor};
\definecolor{tempcolor}{rgb}{0.0,0.3921568691730499,0.0}
        \pronec{O}{13}{19}{tempcolor};
\definecolor{tempcolor}{rgb}{0.0,0.3921568691730499,0.0}
        \pronec{O}{13}{20}{tempcolor};

\definecolor{tempcolor}{rgb}{0.0,0.3921568691730499,0.0}
        \pronec{O}{14}{15}{tempcolor};

\definecolor{tempcolor}{rgb}{0.0,0.3921568691730499,0.0}
        \pronec{O}{15}{6}{tempcolor};
\definecolor{tempcolor}{rgb}{0.0,0.3921568691730499,0.0}
        \pronec{O}{15}{7}{tempcolor};
\definecolor{tempcolor}{rgb}{0.0,0.3921568691730499,0.0}
        \pronec{O}{15}{8}{tempcolor};
\definecolor{tempcolor}{rgb}{0.0,0.3921568691730499,0.0}
        \pronec{O}{15}{10}{tempcolor};
\definecolor{tempcolor}{rgb}{0.0,0.3921568691730499,0.0}
        \pronec{O}{15}{11}{tempcolor};
\definecolor{tempcolor}{rgb}{0.0,0.3921568691730499,0.0}
        \pronec{O}{15}{13}{tempcolor};
\definecolor{tempcolor}{rgb}{0.0,0.3921568691730499,0.0}
        \pronec{O}{15}{14}{tempcolor};
\definecolor{tempcolor}{rgb}{0.0,0.3921568691730499,0.0}
        \pronec{O}{15}{15}{tempcolor};

\definecolor{tempcolor}{rgb}{0.0,0.3921568691730499,0.0}
        \pronec{O}{16}{8}{tempcolor};
\definecolor{tempcolor}{rgb}{0.0,0.3921568691730499,0.0}
        \pronec{O}{16}{10}{tempcolor};
\definecolor{tempcolor}{rgb}{0.0,0.3921568691730499,0.0}
        \pronec{O}{16}{11}{tempcolor};
\definecolor{tempcolor}{rgb}{0.0,0.3921568691730499,0.0}
        \pronec{O}{16}{13}{tempcolor};

\definecolor{tempcolor}{rgb}{0.0,0.3921568691730499,0.0}
        \pronec{O}{17}{8}{tempcolor};
\definecolor{tempcolor}{rgb}{0.0,0.3921568691730499,0.0}
        \pronec{O}{17}{10}{tempcolor};
\definecolor{tempcolor}{rgb}{0.0,0.3921568691730499,0.0}
        \pronec{O}{17}{11}{tempcolor};
\definecolor{tempcolor}{rgb}{0.0,0.3921568691730499,0.0}
        \pronec{O}{17}{13}{tempcolor};

\definecolor{tempcolor}{rgb}{0.0,0.3921568691730499,0.0}
        \pronec{O}{18}{8}{tempcolor};
\definecolor{tempcolor}{rgb}{0.0,0.3921568691730499,0.0}
        \pronec{O}{18}{10}{tempcolor};
\definecolor{tempcolor}{rgb}{0.0,0.3921568691730499,0.0}
        \pronec{O}{18}{11}{tempcolor};
\definecolor{tempcolor}{rgb}{0.0,0.3921568691730499,0.0}
        \pronec{O}{18}{13}{tempcolor};

\definecolor{tempcolor}{rgb}{0.0,0.3921568691730499,0.0}
        \pronec{O}{19}{7}{tempcolor};
\definecolor{tempcolor}{rgb}{0.0,0.3921568691730499,0.0}
        \pronec{O}{19}{8}{tempcolor};
\definecolor{tempcolor}{rgb}{0.0,0.3921568691730499,0.0}
        \pronec{O}{19}{10}{tempcolor};
\definecolor{tempcolor}{rgb}{0.0,0.3921568691730499,0.0}
        \pronec{O}{19}{11}{tempcolor};
\definecolor{tempcolor}{rgb}{0.0,0.3921568691730499,0.0}
        \pronec{O}{19}{13}{tempcolor};

\definecolor{tempcolor}{rgb}{0.0,0.3921568691730499,0.0}
        \pronec{O}{20}{7}{tempcolor};
\definecolor{tempcolor}{rgb}{0.0,0.3921568691730499,0.0}
        \pronec{O}{20}{9}{tempcolor};
\definecolor{tempcolor}{rgb}{0.0,0.3921568691730499,0.0}
        \pronec{O}{20}{10}{tempcolor};
\definecolor{tempcolor}{rgb}{0.0,0.3921568691730499,0.0}
        \pronec{O}{20}{12}{tempcolor};
\definecolor{tempcolor}{rgb}{0.0,0.3921568691730499,0.0}
        \pronec{O}{20}{13}{tempcolor};

\draw ($(O)+(0,2.25)$) node[anchor=east] {$5$};
\draw ($(O)+(0,4.75)$) node[anchor=east] {$10$};
\draw ($(O)+(0,7.25)$) node[anchor=east] {$15$};
\draw ($(O)+(0,9.75)$) node[anchor=east] {$20$};
\draw ($(O)+(2.25,-0.3)$) node {$5$};
\draw ($(O)+(4.75,-0.3)$) node {$10$};
\draw ($(O)+(7.25,-0.3)$) node {$15$};
\draw ($(O)+(9.75,-0.3)$) node {$20$};

    \end{tikzpicture}
    \caption{An optimal solution of the instance given in Figure~\ref{fig:greedyExample}. The density of this solution is $O(n)$.}
    \label{fig:greedyExampleGood}
\end{figure}

\begin{figure}[ht!]
    \centering

    \begin{tikzpicture}
        \coordinate (O) at (0,0);
        \prgrid{O}{23}{24}

        \prone{O}{1}{1};
        \prone{O}{1}{3};
        \prone{O}{1}{5};
        \prone{O}{1}{7};
        \prone{O}{1}{9};
        \prone{O}{1}{11};
        \prone{O}{1}{13};

        \prone{O}{2}{13};

        \prone{O}{3}{1};
        \prone{O}{3}{3};
        \prone{O}{3}{5};
        \prone{O}{3}{7};
        \prone{O}{3}{9};
        \prone{O}{3}{11};
        \prone{O}{3}{13};
        \prone{O}{3}{15};

        \prone{O}{4}{15};

        \prone{O}{5}{1};
        \prone{O}{5}{3};
        \prone{O}{5}{5};
        \prone{O}{5}{7};
        \prone{O}{5}{9};
        \prone{O}{5}{11};
        \prone{O}{5}{13};
        \prone{O}{5}{15};

        \prone{O}{6}{13};

        \prone{O}{7}{1};
        \prone{O}{7}{3};
        \prone{O}{7}{5};
        \prone{O}{7}{7};
        \prone{O}{7}{9};
        \prone{O}{7}{11};
        \prone{O}{7}{13};
        \prone{O}{7}{15};

        \prone{O}{8}{15};

        \prone{O}{9}{1};
        \prone{O}{9}{3};
        \prone{O}{9}{5};
        \prone{O}{9}{7};
        \prone{O}{9}{9};
        \prone{O}{9}{11};
        \prone{O}{9}{13};
        \prone{O}{9}{15};

        \prone{O}{10}{13};

        \prone{O}{11}{1};
        \prone{O}{11}{3};
        \prone{O}{11}{5};
        \prone{O}{11}{7};
        \prone{O}{11}{9};
        \prone{O}{11}{11};
        \prone{O}{11}{13};
        \prone{O}{11}{19};

        \prone{O}{12}{19};

        \prone{O}{13}{1};
        \prone{O}{13}{2};
        \prone{O}{13}{3};
        \prone{O}{13}{5};
        \prone{O}{13}{6};
        \prone{O}{13}{7};
        \prone{O}{13}{9};
        \prone{O}{13}{10};
        \prone{O}{13}{11};
        \prone{O}{13}{19};
        \prone{O}{13}{20};
        \prone{O}{13}{21};
        \prone{O}{13}{22};
        \prone{O}{13}{23};
        \prone{O}{13}{24};

        \prone{O}{14}{19};
        \prone{O}{14}{20};
        \prone{O}{14}{21};
        \prone{O}{14}{22};
        \prone{O}{14}{23};
        \prone{O}{14}{24};

        \prone{O}{15}{3};
        \prone{O}{15}{4};
        \prone{O}{15}{5};
        \prone{O}{15}{7};
        \prone{O}{15}{8};
        \prone{O}{15}{9};
        \prone{O}{15}{19};
        \prone{O}{15}{20};
        \prone{O}{15}{21};
        \prone{O}{15}{22};
        \prone{O}{15}{23};
        \prone{O}{15}{24};

        \prone{O}{16}{19};
        \prone{O}{16}{20};
        \prone{O}{16}{21};
        \prone{O}{16}{22};
        \prone{O}{16}{23};
        \prone{O}{16}{24};

        \prone{O}{17}{19};

        \prone{O}{18}{11};
        \prone{O}{18}{12};
        \prone{O}{18}{13};
        \prone{O}{18}{14};
        \prone{O}{18}{15};
        \prone{O}{18}{16};
        \prone{O}{18}{17};
        \prone{O}{18}{18};
        \prone{O}{18}{19};

        \prone{O}{19}{13};
        \prone{O}{19}{14};
        \prone{O}{19}{15};
        \prone{O}{19}{16};

        \prone{O}{20}{13};
        \prone{O}{20}{14};
        \prone{O}{20}{15};
        \prone{O}{20}{16};

        \prone{O}{21}{13};
        \prone{O}{21}{14};
        \prone{O}{21}{15};
        \prone{O}{21}{16};

        \prone{O}{22}{13};
        \prone{O}{22}{14};
        \prone{O}{22}{15};
        \prone{O}{22}{16};

        \prone{O}{23}{13};
        \prone{O}{23}{14};
        \prone{O}{23}{15};
        \prone{O}{23}{16};


    \end{tikzpicture}
    \caption{A solution of the instance given in Figure~\ref{fig:greedyExample} that is returned by the greedy algorithm. The density of this solution is $O(n)$.}
    \label{fig:greedyAlgorithmBad}
\end{figure}


\subsection{Adaptation to the neighborization algorithm}
\begin{figure}
    \centering

    \begin{tikzpicture}
        \coordinate (O) at (0,0);
        \prgrid{O}{29}{29}

\definecolor{tempcolor}{rgb}{0.0,0.0,1.0}
        \pronec{O}{1}{1}{tempcolor};
\definecolor{tempcolor}{rgb}{0.0,0.0,1.0}
        \pronec{O}{1}{3}{tempcolor};
\definecolor{tempcolor}{rgb}{0.0,0.0,1.0}
        \pronec{O}{1}{5}{tempcolor};
\definecolor{tempcolor}{rgb}{0.0,0.0,1.0}
        \pronec{O}{1}{7}{tempcolor};
\definecolor{tempcolor}{rgb}{0.0,0.0,1.0}
        \pronec{O}{1}{9}{tempcolor};
\definecolor{tempcolor}{rgb}{0.0,0.0,1.0}
        \pronec{O}{1}{11}{tempcolor};
\definecolor{tempcolor}{rgb}{1.0,0.0,0.0}
        \pronec{O}{1}{14}{tempcolor};

\definecolor{tempcolor}{rgb}{1.0,0.0,0.0}
        \pronec{O}{2}{14}{tempcolor};

\definecolor{tempcolor}{rgb}{0.0,0.0,1.0}
        \pronec{O}{3}{1}{tempcolor};
\definecolor{tempcolor}{rgb}{0.0,0.0,1.0}
        \pronec{O}{3}{3}{tempcolor};
\definecolor{tempcolor}{rgb}{0.0,0.0,1.0}
        \pronec{O}{3}{5}{tempcolor};
\definecolor{tempcolor}{rgb}{0.0,0.0,1.0}
        \pronec{O}{3}{7}{tempcolor};
\definecolor{tempcolor}{rgb}{0.0,0.0,1.0}
        \pronec{O}{3}{9}{tempcolor};
\definecolor{tempcolor}{rgb}{0.0,0.0,1.0}
        \pronec{O}{3}{11}{tempcolor};
\definecolor{tempcolor}{rgb}{1.0,0.0,0.0}
        \pronec{O}{3}{13}{tempcolor};
\definecolor{tempcolor}{rgb}{1.0,0.0,0.0}
        \pronec{O}{3}{19}{tempcolor};

\definecolor{tempcolor}{rgb}{1.0,0.0,0.0}
        \pronec{O}{4}{19}{tempcolor};

\definecolor{tempcolor}{rgb}{0.0,0.0,1.0}
        \pronec{O}{5}{1}{tempcolor};
\definecolor{tempcolor}{rgb}{0.0,0.0,1.0}
        \pronec{O}{5}{3}{tempcolor};
\definecolor{tempcolor}{rgb}{0.0,0.0,1.0}
        \pronec{O}{5}{5}{tempcolor};
\definecolor{tempcolor}{rgb}{0.0,0.0,1.0}
        \pronec{O}{5}{7}{tempcolor};
\definecolor{tempcolor}{rgb}{0.0,0.0,1.0}
        \pronec{O}{5}{9}{tempcolor};
\definecolor{tempcolor}{rgb}{0.0,0.0,1.0}
        \pronec{O}{5}{11}{tempcolor};
\definecolor{tempcolor}{rgb}{1.0,0.0,0.0}
        \pronec{O}{5}{15}{tempcolor};
\definecolor{tempcolor}{rgb}{1.0,0.0,0.0}
        \pronec{O}{5}{18}{tempcolor};

\definecolor{tempcolor}{rgb}{1.0,0.0,0.0}
        \pronec{O}{6}{15}{tempcolor};

\definecolor{tempcolor}{rgb}{0.0,0.0,1.0}
        \pronec{O}{7}{1}{tempcolor};
\definecolor{tempcolor}{rgb}{0.0,0.0,1.0}
        \pronec{O}{7}{3}{tempcolor};
\definecolor{tempcolor}{rgb}{0.0,0.0,1.0}
        \pronec{O}{7}{5}{tempcolor};
\definecolor{tempcolor}{rgb}{0.0,0.0,1.0}
        \pronec{O}{7}{7}{tempcolor};
\definecolor{tempcolor}{rgb}{0.0,0.0,1.0}
        \pronec{O}{7}{9}{tempcolor};
\definecolor{tempcolor}{rgb}{0.0,0.0,1.0}
        \pronec{O}{7}{11}{tempcolor};
\definecolor{tempcolor}{rgb}{1.0,0.0,0.0}
        \pronec{O}{7}{14}{tempcolor};
\definecolor{tempcolor}{rgb}{1.0,0.0,0.0}
        \pronec{O}{7}{20}{tempcolor};

\definecolor{tempcolor}{rgb}{1.0,0.0,0.0}
        \pronec{O}{8}{20}{tempcolor};

\definecolor{tempcolor}{rgb}{0.0,0.0,1.0}
        \pronec{O}{9}{1}{tempcolor};
\definecolor{tempcolor}{rgb}{0.0,0.0,1.0}
        \pronec{O}{9}{3}{tempcolor};
\definecolor{tempcolor}{rgb}{0.0,0.0,1.0}
        \pronec{O}{9}{5}{tempcolor};
\definecolor{tempcolor}{rgb}{0.0,0.0,1.0}
        \pronec{O}{9}{7}{tempcolor};
\definecolor{tempcolor}{rgb}{0.0,0.0,1.0}
        \pronec{O}{9}{9}{tempcolor};
\definecolor{tempcolor}{rgb}{0.0,0.0,1.0}
        \pronec{O}{9}{11}{tempcolor};
\definecolor{tempcolor}{rgb}{1.0,0.0,0.0}
        \pronec{O}{9}{16}{tempcolor};
\definecolor{tempcolor}{rgb}{1.0,0.0,0.0}
        \pronec{O}{9}{19}{tempcolor};

\definecolor{tempcolor}{rgb}{1.0,0.0,0.0}
        \pronec{O}{10}{16}{tempcolor};

\definecolor{tempcolor}{rgb}{0.0,0.0,1.0}
        \pronec{O}{11}{1}{tempcolor};
\definecolor{tempcolor}{rgb}{0.0,0.0,1.0}
        \pronec{O}{11}{3}{tempcolor};
\definecolor{tempcolor}{rgb}{0.0,0.0,1.0}
        \pronec{O}{11}{5}{tempcolor};
\definecolor{tempcolor}{rgb}{0.0,0.0,1.0}
        \pronec{O}{11}{7}{tempcolor};
\definecolor{tempcolor}{rgb}{0.0,0.0,1.0}
        \pronec{O}{11}{9}{tempcolor};
\definecolor{tempcolor}{rgb}{0.0,0.0,1.0}
        \pronec{O}{11}{11}{tempcolor};
\definecolor{tempcolor}{rgb}{1.0,0.0,0.0}
        \pronec{O}{11}{15}{tempcolor};
\definecolor{tempcolor}{rgb}{0.0,0.3921568691730499,0.0}
        \pronec{O}{11}{23}{tempcolor};

\definecolor{tempcolor}{rgb}{0.0,0.3921568691730499,0.0}
        \pronec{O}{12}{23}{tempcolor};
\definecolor{tempcolor}{rgb}{0.0,0.3921568691730499,0.0}
        \pronec{O}{12}{28}{tempcolor};
\definecolor{tempcolor}{rgb}{0.0,0.3921568691730499,0.0}
        \pronec{O}{12}{29}{tempcolor};

\definecolor{tempcolor}{rgb}{1.0,0.0,0.0}
        \pronec{O}{13}{3}{tempcolor};
\definecolor{tempcolor}{rgb}{0.0,0.3921568691730499,0.0}
        \pronec{O}{13}{23}{tempcolor};
\definecolor{tempcolor}{rgb}{0.0,0.3921568691730499,0.0}
        \pronec{O}{13}{24}{tempcolor};
\definecolor{tempcolor}{rgb}{0.0,0.3921568691730499,0.0}
        \pronec{O}{13}{25}{tempcolor};
\definecolor{tempcolor}{rgb}{0.0,0.3921568691730499,0.0}
        \pronec{O}{13}{26}{tempcolor};
\definecolor{tempcolor}{rgb}{0.0,0.3921568691730499,0.0}
        \pronec{O}{13}{27}{tempcolor};
\definecolor{tempcolor}{rgb}{0.0,0.3921568691730499,0.0}
        \pronec{O}{13}{28}{tempcolor};

\definecolor{tempcolor}{rgb}{1.0,0.0,0.0}
        \pronec{O}{14}{1}{tempcolor};
\definecolor{tempcolor}{rgb}{1.0,0.0,0.0}
        \pronec{O}{14}{2}{tempcolor};
\definecolor{tempcolor}{rgb}{1.0,0.0,0.0}
        \pronec{O}{14}{7}{tempcolor};

\definecolor{tempcolor}{rgb}{1.0,0.0,0.0}
        \pronec{O}{15}{5}{tempcolor};
\definecolor{tempcolor}{rgb}{1.0,0.0,0.0}
        \pronec{O}{15}{6}{tempcolor};
\definecolor{tempcolor}{rgb}{1.0,0.0,0.0}
        \pronec{O}{15}{11}{tempcolor};

\definecolor{tempcolor}{rgb}{1.0,0.0,0.0}
        \pronec{O}{16}{9}{tempcolor};
\definecolor{tempcolor}{rgb}{1.0,0.0,0.0}
        \pronec{O}{16}{10}{tempcolor};
\definecolor{tempcolor}{rgb}{0.0,0.3921568691730499,0.0}
        \pronec{O}{16}{29}{tempcolor};

\definecolor{tempcolor}{rgb}{0.0,0.3921568691730499,0.0}
        \pronec{O}{17}{23}{tempcolor};
\definecolor{tempcolor}{rgb}{0.0,0.3921568691730499,0.0}
        \pronec{O}{17}{24}{tempcolor};
\definecolor{tempcolor}{rgb}{0.0,0.3921568691730499,0.0}
        \pronec{O}{17}{25}{tempcolor};
\definecolor{tempcolor}{rgb}{0.0,0.3921568691730499,0.0}
        \pronec{O}{17}{26}{tempcolor};
\definecolor{tempcolor}{rgb}{0.0,0.3921568691730499,0.0}
        \pronec{O}{17}{27}{tempcolor};
\definecolor{tempcolor}{rgb}{0.0,0.3921568691730499,0.0}
        \pronec{O}{17}{28}{tempcolor};
\definecolor{tempcolor}{rgb}{0.0,0.3921568691730499,0.0}
        \pronec{O}{17}{29}{tempcolor};

\definecolor{tempcolor}{rgb}{1.0,0.0,0.0}
        \pronec{O}{18}{5}{tempcolor};
\definecolor{tempcolor}{rgb}{0.0,0.3921568691730499,0.0}
        \pronec{O}{18}{23}{tempcolor};
\definecolor{tempcolor}{rgb}{0.0,0.3921568691730499,0.0}
        \pronec{O}{18}{24}{tempcolor};
\definecolor{tempcolor}{rgb}{0.0,0.3921568691730499,0.0}
        \pronec{O}{18}{25}{tempcolor};
\definecolor{tempcolor}{rgb}{0.0,0.3921568691730499,0.0}
        \pronec{O}{18}{26}{tempcolor};
\definecolor{tempcolor}{rgb}{0.0,0.3921568691730499,0.0}
        \pronec{O}{18}{27}{tempcolor};
\definecolor{tempcolor}{rgb}{0.0,0.3921568691730499,0.0}
        \pronec{O}{18}{28}{tempcolor};

\definecolor{tempcolor}{rgb}{1.0,0.0,0.0}
        \pronec{O}{19}{3}{tempcolor};
\definecolor{tempcolor}{rgb}{1.0,0.0,0.0}
        \pronec{O}{19}{4}{tempcolor};
\definecolor{tempcolor}{rgb}{1.0,0.0,0.0}
        \pronec{O}{19}{9}{tempcolor};

\definecolor{tempcolor}{rgb}{1.0,0.0,0.0}
        \pronec{O}{20}{7}{tempcolor};
\definecolor{tempcolor}{rgb}{1.0,0.0,0.0}
        \pronec{O}{20}{8}{tempcolor};
\definecolor{tempcolor}{rgb}{0.0,0.3921568691730499,0.0}
        \pronec{O}{20}{29}{tempcolor};

\definecolor{tempcolor}{rgb}{0.0,0.3921568691730499,0.0}
        \pronec{O}{21}{23}{tempcolor};
\definecolor{tempcolor}{rgb}{0.0,0.3921568691730499,0.0}
        \pronec{O}{21}{24}{tempcolor};
\definecolor{tempcolor}{rgb}{0.0,0.3921568691730499,0.0}
        \pronec{O}{21}{25}{tempcolor};
\definecolor{tempcolor}{rgb}{0.0,0.3921568691730499,0.0}
        \pronec{O}{21}{26}{tempcolor};
\definecolor{tempcolor}{rgb}{0.0,0.3921568691730499,0.0}
        \pronec{O}{21}{27}{tempcolor};
\definecolor{tempcolor}{rgb}{0.0,0.3921568691730499,0.0}
        \pronec{O}{21}{28}{tempcolor};
\definecolor{tempcolor}{rgb}{0.0,0.3921568691730499,0.0}
        \pronec{O}{21}{29}{tempcolor};

\definecolor{tempcolor}{rgb}{0.0,0.3921568691730499,0.0}
        \pronec{O}{22}{23}{tempcolor};

\definecolor{tempcolor}{rgb}{0.0,0.3921568691730499,0.0}
        \pronec{O}{23}{11}{tempcolor};
\definecolor{tempcolor}{rgb}{0.0,0.3921568691730499,0.0}
        \pronec{O}{23}{12}{tempcolor};
\definecolor{tempcolor}{rgb}{0.0,0.3921568691730499,0.0}
        \pronec{O}{23}{13}{tempcolor};
\definecolor{tempcolor}{rgb}{0.0,0.3921568691730499,0.0}
        \pronec{O}{23}{17}{tempcolor};
\definecolor{tempcolor}{rgb}{0.0,0.3921568691730499,0.0}
        \pronec{O}{23}{18}{tempcolor};
\definecolor{tempcolor}{rgb}{0.0,0.3921568691730499,0.0}
        \pronec{O}{23}{21}{tempcolor};
\definecolor{tempcolor}{rgb}{0.0,0.3921568691730499,0.0}
        \pronec{O}{23}{22}{tempcolor};
\definecolor{tempcolor}{rgb}{0.0,0.3921568691730499,0.0}
        \pronec{O}{23}{23}{tempcolor};

\definecolor{tempcolor}{rgb}{0.0,0.3921568691730499,0.0}
        \pronec{O}{24}{13}{tempcolor};
\definecolor{tempcolor}{rgb}{0.0,0.3921568691730499,0.0}
        \pronec{O}{24}{17}{tempcolor};
\definecolor{tempcolor}{rgb}{0.0,0.3921568691730499,0.0}
        \pronec{O}{24}{18}{tempcolor};
\definecolor{tempcolor}{rgb}{0.0,0.3921568691730499,0.0}
        \pronec{O}{24}{21}{tempcolor};

\definecolor{tempcolor}{rgb}{0.0,0.3921568691730499,0.0}
        \pronec{O}{25}{13}{tempcolor};
\definecolor{tempcolor}{rgb}{0.0,0.3921568691730499,0.0}
        \pronec{O}{25}{17}{tempcolor};
\definecolor{tempcolor}{rgb}{0.0,0.3921568691730499,0.0}
        \pronec{O}{25}{18}{tempcolor};
\definecolor{tempcolor}{rgb}{0.0,0.3921568691730499,0.0}
        \pronec{O}{25}{21}{tempcolor};

\definecolor{tempcolor}{rgb}{0.0,0.3921568691730499,0.0}
        \pronec{O}{26}{13}{tempcolor};
\definecolor{tempcolor}{rgb}{0.0,0.3921568691730499,0.0}
        \pronec{O}{26}{17}{tempcolor};
\definecolor{tempcolor}{rgb}{0.0,0.3921568691730499,0.0}
        \pronec{O}{26}{18}{tempcolor};
\definecolor{tempcolor}{rgb}{0.0,0.3921568691730499,0.0}
        \pronec{O}{26}{21}{tempcolor};

\definecolor{tempcolor}{rgb}{0.0,0.3921568691730499,0.0}
        \pronec{O}{27}{13}{tempcolor};
\definecolor{tempcolor}{rgb}{0.0,0.3921568691730499,0.0}
        \pronec{O}{27}{17}{tempcolor};
\definecolor{tempcolor}{rgb}{0.0,0.3921568691730499,0.0}
        \pronec{O}{27}{18}{tempcolor};
\definecolor{tempcolor}{rgb}{0.0,0.3921568691730499,0.0}
        \pronec{O}{27}{21}{tempcolor};

\definecolor{tempcolor}{rgb}{0.0,0.3921568691730499,0.0}
        \pronec{O}{28}{12}{tempcolor};
\definecolor{tempcolor}{rgb}{0.0,0.3921568691730499,0.0}
        \pronec{O}{28}{13}{tempcolor};
\definecolor{tempcolor}{rgb}{0.0,0.3921568691730499,0.0}
        \pronec{O}{28}{17}{tempcolor};
\definecolor{tempcolor}{rgb}{0.0,0.3921568691730499,0.0}
        \pronec{O}{28}{18}{tempcolor};
\definecolor{tempcolor}{rgb}{0.0,0.3921568691730499,0.0}
        \pronec{O}{28}{21}{tempcolor};

\definecolor{tempcolor}{rgb}{0.0,0.3921568691730499,0.0}
        \pronec{O}{29}{12}{tempcolor};
\definecolor{tempcolor}{rgb}{0.0,0.3921568691730499,0.0}
        \pronec{O}{29}{16}{tempcolor};
\definecolor{tempcolor}{rgb}{0.0,0.3921568691730499,0.0}
        \pronec{O}{29}{17}{tempcolor};
\definecolor{tempcolor}{rgb}{0.0,0.3921568691730499,0.0}
        \pronec{O}{29}{20}{tempcolor};
\definecolor{tempcolor}{rgb}{0.0,0.3921568691730499,0.0}
        \pronec{O}{29}{21}{tempcolor};

\draw ($(O)+(0,2.25)$) node[anchor=east] {$5$};
\draw ($(O)+(0,4.75)$) node[anchor=east] {$10$};
\draw ($(O)+(0,7.25)$) node[anchor=east] {$15$};
\draw ($(O)+(0,9.75)$) node[anchor=east] {$20$};
\draw ($(O)+(0,12.25)$) node[anchor=east] {$25$};
\draw ($(O)+(2.25,-0.3)$) node {$5$};
\draw ($(O)+(4.75,-0.3)$) node {$10$};
\draw ($(O)+(7.25,-0.3)$) node {$15$};
\draw ($(O)+(9.75,-0.3)$) node {$20$};
\draw ($(O)+(12.25,-0.3)$) node {$25$};


\definecolor{tempcolor}{rgb}{0.0,0.0,1.0}
\prvtdlinec{O}{2}{29}{tempcolor};
\prvtdlinec{O}{4}{29}{tempcolor};
\prvtdlinec{O}{6}{29}{tempcolor};
\prvtdlinec{O}{8}{29}{tempcolor};
\prvtdlinec{O}{10}{29}{tempcolor};

\definecolor{tempcolor}{rgb}{0.0,0.3921568691730499,0.0}
\prvtdlinec{O}{13}{29}{tempcolor};
\prvtdlinec{O}{14}{29}{tempcolor};
\prvtdlinec{O}{15}{29}{tempcolor};
\prvtdlinec{O}{18}{29}{tempcolor};
\prvtdlinec{O}{19}{29}{tempcolor};


\definecolor{tempcolor}{rgb}{0.0,0.0,1.0}
\prvtdcolumnc{O}{2}{29}{tempcolor};
\prvtdcolumnc{O}{4}{29}{tempcolor};
\prvtdcolumnc{O}{6}{29}{tempcolor};
\prvtdcolumnc{O}{8}{29}{tempcolor};
\prvtdcolumnc{O}{10}{29}{tempcolor};
\prvtrectc{O}{0}{0}{11}{11}{tempcolor};
	
\definecolor{tempcolor}{rgb}{0.0,0.3921568691730499,0.0}
\prvtdcolumnc{O}{13}{29}{tempcolor};
\prvtdcolumnc{O}{14}{29}{tempcolor};
\prvtdcolumnc{O}{15}{29}{tempcolor};
\prvtdcolumnc{O}{18}{29}{tempcolor};
\prvtdcolumnc{O}{19}{29}{tempcolor};
\prvtrectc{O}{10}{22}{21}{29}{tempcolor};
\prvtrectc{O}{22}{10}{29}{21}{tempcolor};


    \end{tikzpicture}
    \caption{An example of instance for which an optimal density is $O(\sqrt{n})$ times the density of a solution returned by the Neighborization algorithm.}
    \label{fig:neighborizationExample}
\end{figure}

\begin{figure}[ht!]
    \centering

    \begin{tikzpicture}
        \coordinate (O) at (0,0);
        \prgrid{O}{25}{25}

        \prone{O}{1}{1};
        \prone{O}{1}{2};
        \prone{O}{1}{3};
        \prone{O}{1}{4};
        \prone{O}{1}{5};
        \prone{O}{1}{6};
        \prone{O}{1}{9};

        \prone{O}{2}{1};
        \prone{O}{2}{2};
        \prone{O}{2}{3};
        \prone{O}{2}{4};
        \prone{O}{2}{5};
        \prone{O}{2}{6};
        \prone{O}{2}{8};
        \prone{O}{2}{9};
        \prone{O}{2}{14};

        \prone{O}{3}{1};
        \prone{O}{3}{2};
        \prone{O}{3}{3};
        \prone{O}{3}{4};
        \prone{O}{3}{5};
        \prone{O}{3}{6};
        \prone{O}{3}{10};
        \prone{O}{3}{13};
        \prone{O}{3}{14};

        \prone{O}{4}{1};
        \prone{O}{4}{2};
        \prone{O}{4}{3};
        \prone{O}{4}{4};
        \prone{O}{4}{5};
        \prone{O}{4}{6};
        \prone{O}{4}{9};
        \prone{O}{4}{10};
        \prone{O}{4}{15};

        \prone{O}{5}{1};
        \prone{O}{5}{2};
        \prone{O}{5}{3};
        \prone{O}{5}{4};
        \prone{O}{5}{5};
        \prone{O}{5}{6};
        \prone{O}{5}{11};
        \prone{O}{5}{14};
        \prone{O}{5}{15};

        \prone{O}{6}{1};
        \prone{O}{6}{2};
        \prone{O}{6}{3};
        \prone{O}{6}{4};
        \prone{O}{6}{5};
        \prone{O}{6}{6};
        \prone{O}{6}{10};
        \prone{O}{6}{11};
        \prone{O}{6}{18};

        \prone{O}{7}{18};

        \prone{O}{8}{2};
        \prone{O}{8}{18};
        \prone{O}{8}{19};
        \prone{O}{8}{20};
        \prone{O}{8}{21};
        \prone{O}{8}{22};
        \prone{O}{8}{23};
        \prone{O}{8}{24};

        \prone{O}{9}{1};
        \prone{O}{9}{2};
        \prone{O}{9}{4};

        \prone{O}{10}{3};
        \prone{O}{10}{4};
        \prone{O}{10}{6};

        \prone{O}{11}{5};
        \prone{O}{11}{6};
        \prone{O}{11}{25};

        \prone{O}{12}{18};
        \prone{O}{12}{19};
        \prone{O}{12}{20};
        \prone{O}{12}{21};
        \prone{O}{12}{22};
        \prone{O}{12}{23};
        \prone{O}{12}{24};
        \prone{O}{12}{25};

        \prone{O}{13}{3};
        \prone{O}{13}{18};
        \prone{O}{13}{19};
        \prone{O}{13}{20};
        \prone{O}{13}{21};
        \prone{O}{13}{22};
        \prone{O}{13}{23};
        \prone{O}{13}{24};

        \prone{O}{14}{2};
        \prone{O}{14}{3};
        \prone{O}{14}{5};

        \prone{O}{15}{4};
        \prone{O}{15}{5};
        \prone{O}{15}{25};

        \prone{O}{16}{18};
        \prone{O}{16}{19};
        \prone{O}{16}{20};
        \prone{O}{16}{21};
        \prone{O}{16}{22};
        \prone{O}{16}{23};
        \prone{O}{16}{24};
        \prone{O}{16}{25};

        \prone{O}{17}{18};

        \prone{O}{18}{6};
        \prone{O}{18}{7};
        \prone{O}{18}{8};
        \prone{O}{18}{12};
        \prone{O}{18}{13};
        \prone{O}{18}{16};
        \prone{O}{18}{17};
        \prone{O}{18}{18};

        \prone{O}{19}{8};
        \prone{O}{19}{12};
        \prone{O}{19}{13};
        \prone{O}{19}{16};

        \prone{O}{20}{8};
        \prone{O}{20}{12};
        \prone{O}{20}{13};
        \prone{O}{20}{16};

        \prone{O}{21}{8};
        \prone{O}{21}{12};
        \prone{O}{21}{13};
        \prone{O}{21}{16};

        \prone{O}{22}{8};
        \prone{O}{22}{12};
        \prone{O}{22}{13};
        \prone{O}{22}{16};

        \prone{O}{23}{8};
        \prone{O}{23}{12};
        \prone{O}{23}{13};
        \prone{O}{23}{16};

        \prone{O}{24}{8};
        \prone{O}{24}{12};
        \prone{O}{24}{13};
        \prone{O}{24}{16};

        \prone{O}{25}{11};
        \prone{O}{25}{12};
        \prone{O}{25}{15};
        \prone{O}{25}{16};


    \end{tikzpicture}
    \caption{An optimal solution of the instance given in Figure~\ref{fig:neighborizationExample}. The density of this solution is $O(n)$.}
    \label{fig:neighborizationExampleGood}
\end{figure}

\begin{figure}
    \centering

    \begin{tikzpicture}
          \coordinate (O) at (0,0);
          \prgrid{O}{26}{26}
          
          \definecolor{tempcolor}{rgb}{0.0,0.0,1.0}
          \pronec{O}{1}{1}{tempcolor};
          \definecolor{tempcolor}{rgb}{0.0,0.0,1.0}
          \pronec{O}{1}{3}{tempcolor};
          \definecolor{tempcolor}{rgb}{0.0,0.0,1.0}
          \pronec{O}{1}{5}{tempcolor};
          \definecolor{tempcolor}{rgb}{0.0,0.0,1.0}
          \pronec{O}{1}{7}{tempcolor};
          \definecolor{tempcolor}{rgb}{0.0,0.0,1.0}
          \pronec{O}{1}{9}{tempcolor};
          \definecolor{tempcolor}{rgb}{0.0,0.0,1.0}
          \pronec{O}{1}{11}{tempcolor};
          \definecolor{tempcolor}{rgb}{1.0,0.0,0.0}
          \pronec{O}{1}{13}{tempcolor};
          
          \definecolor{tempcolor}{rgb}{1.0,0.0,0.0}
          \pronec{O}{2}{13}{tempcolor};
          
          \definecolor{tempcolor}{rgb}{0.0,0.0,1.0}
          \pronec{O}{3}{1}{tempcolor};
          \definecolor{tempcolor}{rgb}{0.0,0.0,1.0}
          \pronec{O}{3}{3}{tempcolor};
          \definecolor{tempcolor}{rgb}{0.0,0.0,1.0}
          \pronec{O}{3}{5}{tempcolor};
          \definecolor{tempcolor}{rgb}{0.0,0.0,1.0}
          \pronec{O}{3}{7}{tempcolor};
          \definecolor{tempcolor}{rgb}{0.0,0.0,1.0}
          \pronec{O}{3}{9}{tempcolor};
          \definecolor{tempcolor}{rgb}{0.0,0.0,1.0}
          \pronec{O}{3}{11}{tempcolor};
          \definecolor{tempcolor}{rgb}{1.0,0.0,0.0}
          \pronec{O}{3}{13}{tempcolor};
          \definecolor{tempcolor}{rgb}{1.0,0.0,0.0}
          \pronec{O}{3}{16}{tempcolor};
          
          \definecolor{tempcolor}{rgb}{1.0,0.0,0.0}
          \pronec{O}{4}{16}{tempcolor};
          
          \definecolor{tempcolor}{rgb}{0.0,0.0,1.0}
          \pronec{O}{5}{1}{tempcolor};
          \definecolor{tempcolor}{rgb}{0.0,0.0,1.0}
          \pronec{O}{5}{3}{tempcolor};
          \definecolor{tempcolor}{rgb}{0.0,0.0,1.0}
          \pronec{O}{5}{5}{tempcolor};
          \definecolor{tempcolor}{rgb}{0.0,0.0,1.0}
          \pronec{O}{5}{7}{tempcolor};
          \definecolor{tempcolor}{rgb}{0.0,0.0,1.0}
          \pronec{O}{5}{9}{tempcolor};
          \definecolor{tempcolor}{rgb}{0.0,0.0,1.0}
          \pronec{O}{5}{11}{tempcolor};
          \definecolor{tempcolor}{rgb}{1.0,0.0,0.0}
          \pronec{O}{5}{13}{tempcolor};
          \definecolor{tempcolor}{rgb}{1.0,0.0,0.0}
          \pronec{O}{5}{16}{tempcolor};
          
          \definecolor{tempcolor}{rgb}{1.0,0.0,0.0}
          \pronec{O}{6}{13}{tempcolor};
          
          \definecolor{tempcolor}{rgb}{0.0,0.0,1.0}
          \pronec{O}{7}{1}{tempcolor};
          \definecolor{tempcolor}{rgb}{0.0,0.0,1.0}
          \pronec{O}{7}{3}{tempcolor};
          \definecolor{tempcolor}{rgb}{0.0,0.0,1.0}
          \pronec{O}{7}{5}{tempcolor};
          \definecolor{tempcolor}{rgb}{0.0,0.0,1.0}
          \pronec{O}{7}{7}{tempcolor};
          \definecolor{tempcolor}{rgb}{0.0,0.0,1.0}
          \pronec{O}{7}{9}{tempcolor};
          \definecolor{tempcolor}{rgb}{0.0,0.0,1.0}
          \pronec{O}{7}{11}{tempcolor};
          \definecolor{tempcolor}{rgb}{1.0,0.0,0.0}
          \pronec{O}{7}{13}{tempcolor};
          \definecolor{tempcolor}{rgb}{1.0,0.0,0.0}
          \pronec{O}{7}{16}{tempcolor};
          
          \definecolor{tempcolor}{rgb}{1.0,0.0,0.0}
          \pronec{O}{8}{16}{tempcolor};
          
          \definecolor{tempcolor}{rgb}{0.0,0.0,1.0}
          \pronec{O}{9}{1}{tempcolor};
          \definecolor{tempcolor}{rgb}{0.0,0.0,1.0}
          \pronec{O}{9}{3}{tempcolor};
          \definecolor{tempcolor}{rgb}{0.0,0.0,1.0}
          \pronec{O}{9}{5}{tempcolor};
          \definecolor{tempcolor}{rgb}{0.0,0.0,1.0}
          \pronec{O}{9}{7}{tempcolor};
          \definecolor{tempcolor}{rgb}{0.0,0.0,1.0}
          \pronec{O}{9}{9}{tempcolor};
          \definecolor{tempcolor}{rgb}{0.0,0.0,1.0}
          \pronec{O}{9}{11}{tempcolor};
          \definecolor{tempcolor}{rgb}{1.0,0.0,0.0}
          \pronec{O}{9}{13}{tempcolor};
          \definecolor{tempcolor}{rgb}{1.0,0.0,0.0}
          \pronec{O}{9}{16}{tempcolor};
          
          \definecolor{tempcolor}{rgb}{1.0,0.0,0.0}
          \pronec{O}{10}{13}{tempcolor};
          
          \definecolor{tempcolor}{rgb}{0.0,0.0,1.0}
          \pronec{O}{11}{1}{tempcolor};
          \definecolor{tempcolor}{rgb}{0.0,0.0,1.0}
          \pronec{O}{11}{3}{tempcolor};
          \definecolor{tempcolor}{rgb}{0.0,0.0,1.0}
          \pronec{O}{11}{5}{tempcolor};
          \definecolor{tempcolor}{rgb}{0.0,0.0,1.0}
          \pronec{O}{11}{7}{tempcolor};
          \definecolor{tempcolor}{rgb}{0.0,0.0,1.0}
          \pronec{O}{11}{9}{tempcolor};
          \definecolor{tempcolor}{rgb}{0.0,0.0,1.0}
          \pronec{O}{11}{11}{tempcolor};
          \definecolor{tempcolor}{rgb}{1.0,0.0,0.0}
          \pronec{O}{11}{13}{tempcolor};
          \definecolor{tempcolor}{rgb}{0.0,0.0,0.0}
          \pronec{O}{11}{19}{tempcolor};
          
          \definecolor{tempcolor}{rgb}{0.0,0.0,0.0}
          \pronec{O}{12}{19}{tempcolor};
          
          \definecolor{tempcolor}{rgb}{1.0,0.0,0.0}
          \pronec{O}{13}{1}{tempcolor};
          \definecolor{tempcolor}{rgb}{1.0,0.0,0.0}
          \pronec{O}{13}{2}{tempcolor};
          \definecolor{tempcolor}{rgb}{1.0,0.0,0.0}
          \pronec{O}{13}{3}{tempcolor};
          \definecolor{tempcolor}{rgb}{1.0,0.0,0.0}
          \pronec{O}{13}{5}{tempcolor};
          \definecolor{tempcolor}{rgb}{1.0,0.0,0.0}
          \pronec{O}{13}{6}{tempcolor};
          \definecolor{tempcolor}{rgb}{1.0,0.0,0.0}
          \pronec{O}{13}{7}{tempcolor};
          \definecolor{tempcolor}{rgb}{1.0,0.0,0.0}
          \pronec{O}{13}{9}{tempcolor};
          \definecolor{tempcolor}{rgb}{1.0,0.0,0.0}
          \pronec{O}{13}{10}{tempcolor};
          \definecolor{tempcolor}{rgb}{1.0,0.0,0.0}
          \pronec{O}{13}{11}{tempcolor};
          \definecolor{tempcolor}{rgb}{0.0,0.3921568691730499,0.0}
          \pronec{O}{13}{19}{tempcolor};
          \definecolor{tempcolor}{rgb}{0.0,0.3921568691730499,0.0}
          \pronec{O}{13}{20}{tempcolor};
          \definecolor{tempcolor}{rgb}{0.0,0.3921568691730499,0.0}
          \pronec{O}{13}{21}{tempcolor};
          \definecolor{tempcolor}{rgb}{0.0,0.3921568691730499,0.0}
          \pronec{O}{13}{22}{tempcolor};
          \definecolor{tempcolor}{rgb}{0.0,0.3921568691730499,0.0}
          \pronec{O}{13}{23}{tempcolor};
          \definecolor{tempcolor}{rgb}{0.0,0.3921568691730499,0.0}
          \pronec{O}{13}{24}{tempcolor};
          \definecolor{tempcolor}{rgb}{0.0,0.3921568691730499,0.0}
          \pronec{O}{13}{25}{tempcolor};
          \definecolor{tempcolor}{rgb}{0.0,0.3921568691730499,0.0}
          \pronec{O}{13}{26}{tempcolor};
          
          \definecolor{tempcolor}{rgb}{0.0,0.3921568691730499,0.0}
          \pronec{O}{14}{19}{tempcolor};
          \definecolor{tempcolor}{rgb}{0.0,0.3921568691730499,0.0}
          \pronec{O}{14}{20}{tempcolor};
          \definecolor{tempcolor}{rgb}{0.0,0.3921568691730499,0.0}
          \pronec{O}{14}{21}{tempcolor};
          \definecolor{tempcolor}{rgb}{0.0,0.3921568691730499,0.0}
          \pronec{O}{14}{22}{tempcolor};
          \definecolor{tempcolor}{rgb}{0.0,0.3921568691730499,0.0}
          \pronec{O}{14}{23}{tempcolor};
          \definecolor{tempcolor}{rgb}{0.0,0.3921568691730499,0.0}
          \pronec{O}{14}{24}{tempcolor};
          \definecolor{tempcolor}{rgb}{0.0,0.3921568691730499,0.0}
          \pronec{O}{14}{25}{tempcolor};
          \definecolor{tempcolor}{rgb}{0.0,0.3921568691730499,0.0}
          \pronec{O}{14}{26}{tempcolor};
          
          \definecolor{tempcolor}{rgb}{0.0,0.0,0.0}
          \pronec{O}{15}{19}{tempcolor};
          
          \definecolor{tempcolor}{rgb}{1.0,0.0,0.0}
          \pronec{O}{16}{3}{tempcolor};
          \definecolor{tempcolor}{rgb}{1.0,0.0,0.0}
          \pronec{O}{16}{4}{tempcolor};
          \definecolor{tempcolor}{rgb}{1.0,0.0,0.0}
          \pronec{O}{16}{5}{tempcolor};
          \definecolor{tempcolor}{rgb}{1.0,0.0,0.0}
          \pronec{O}{16}{7}{tempcolor};
          \definecolor{tempcolor}{rgb}{1.0,0.0,0.0}
          \pronec{O}{16}{8}{tempcolor};
          \definecolor{tempcolor}{rgb}{1.0,0.0,0.0}
          \pronec{O}{16}{9}{tempcolor};
          \definecolor{tempcolor}{rgb}{0.0,0.3921568691730499,0.0}
          \pronec{O}{16}{19}{tempcolor};
          \definecolor{tempcolor}{rgb}{0.0,0.3921568691730499,0.0}
          \pronec{O}{16}{20}{tempcolor};
          \definecolor{tempcolor}{rgb}{0.0,0.3921568691730499,0.0}
          \pronec{O}{16}{21}{tempcolor};
          \definecolor{tempcolor}{rgb}{0.0,0.3921568691730499,0.0}
          \pronec{O}{16}{22}{tempcolor};
          \definecolor{tempcolor}{rgb}{0.0,0.3921568691730499,0.0}
          \pronec{O}{16}{23}{tempcolor};
          \definecolor{tempcolor}{rgb}{0.0,0.3921568691730499,0.0}
          \pronec{O}{16}{24}{tempcolor};
          \definecolor{tempcolor}{rgb}{0.0,0.3921568691730499,0.0}
          \pronec{O}{16}{25}{tempcolor};
          \definecolor{tempcolor}{rgb}{0.0,0.3921568691730499,0.0}
          \pronec{O}{16}{26}{tempcolor};
          
          \definecolor{tempcolor}{rgb}{0.0,0.3921568691730499,0.0}
          \pronec{O}{17}{19}{tempcolor};
          \definecolor{tempcolor}{rgb}{0.0,0.3921568691730499,0.0}
          \pronec{O}{17}{20}{tempcolor};
          \definecolor{tempcolor}{rgb}{0.0,0.3921568691730499,0.0}
          \pronec{O}{17}{21}{tempcolor};
          \definecolor{tempcolor}{rgb}{0.0,0.3921568691730499,0.0}
          \pronec{O}{17}{22}{tempcolor};
          \definecolor{tempcolor}{rgb}{0.0,0.3921568691730499,0.0}
          \pronec{O}{17}{23}{tempcolor};
          \definecolor{tempcolor}{rgb}{0.0,0.3921568691730499,0.0}
          \pronec{O}{17}{24}{tempcolor};
          \definecolor{tempcolor}{rgb}{0.0,0.3921568691730499,0.0}
          \pronec{O}{17}{25}{tempcolor};
          \definecolor{tempcolor}{rgb}{0.0,0.3921568691730499,0.0}
          \pronec{O}{17}{26}{tempcolor};
          
          \definecolor{tempcolor}{rgb}{0.0,0.0,0.0}
          \pronec{O}{18}{19}{tempcolor};
          
          \definecolor{tempcolor}{rgb}{0.0,0.0,0.0}
          \pronec{O}{19}{11}{tempcolor};
          \definecolor{tempcolor}{rgb}{0.0,0.0,0.0}
          \pronec{O}{19}{12}{tempcolor};
          \definecolor{tempcolor}{rgb}{0.0,0.3921568691730499,0.0}
          \pronec{O}{19}{13}{tempcolor};
          \definecolor{tempcolor}{rgb}{0.0,0.3921568691730499,0.0}
          \pronec{O}{19}{14}{tempcolor};
          \definecolor{tempcolor}{rgb}{0.0,0.0,0.0}
          \pronec{O}{19}{15}{tempcolor};
          \definecolor{tempcolor}{rgb}{0.0,0.3921568691730499,0.0}
          \pronec{O}{19}{16}{tempcolor};
          \definecolor{tempcolor}{rgb}{0.0,0.3921568691730499,0.0}
          \pronec{O}{19}{17}{tempcolor};
          \definecolor{tempcolor}{rgb}{0.0,0.0,0.0}
          \pronec{O}{19}{18}{tempcolor};
          \definecolor{tempcolor}{rgb}{0.0,0.0,0.0}
          \pronec{O}{19}{19}{tempcolor};
          
          \definecolor{tempcolor}{rgb}{0.0,0.3921568691730499,0.0}
          \pronec{O}{20}{13}{tempcolor};
          \definecolor{tempcolor}{rgb}{0.0,0.3921568691730499,0.0}
          \pronec{O}{20}{14}{tempcolor};
          \definecolor{tempcolor}{rgb}{0.0,0.3921568691730499,0.0}
          \pronec{O}{20}{16}{tempcolor};
          \definecolor{tempcolor}{rgb}{0.0,0.3921568691730499,0.0}
          \pronec{O}{20}{17}{tempcolor};
          
          \definecolor{tempcolor}{rgb}{0.0,0.3921568691730499,0.0}
          \pronec{O}{21}{13}{tempcolor};
          \definecolor{tempcolor}{rgb}{0.0,0.3921568691730499,0.0}
          \pronec{O}{21}{14}{tempcolor};
          \definecolor{tempcolor}{rgb}{0.0,0.3921568691730499,0.0}
          \pronec{O}{21}{16}{tempcolor};
          \definecolor{tempcolor}{rgb}{0.0,0.3921568691730499,0.0}
          \pronec{O}{21}{17}{tempcolor};
          
          \definecolor{tempcolor}{rgb}{0.0,0.3921568691730499,0.0}
          \pronec{O}{22}{13}{tempcolor};
          \definecolor{tempcolor}{rgb}{0.0,0.3921568691730499,0.0}
          \pronec{O}{22}{14}{tempcolor};
          \definecolor{tempcolor}{rgb}{0.0,0.3921568691730499,0.0}
          \pronec{O}{22}{16}{tempcolor};
          \definecolor{tempcolor}{rgb}{0.0,0.3921568691730499,0.0}
          \pronec{O}{22}{17}{tempcolor};
          
          \definecolor{tempcolor}{rgb}{0.0,0.3921568691730499,0.0}
          \pronec{O}{23}{13}{tempcolor};
          \definecolor{tempcolor}{rgb}{0.0,0.3921568691730499,0.0}
          \pronec{O}{23}{14}{tempcolor};
          \definecolor{tempcolor}{rgb}{0.0,0.3921568691730499,0.0}
          \pronec{O}{23}{16}{tempcolor};
          \definecolor{tempcolor}{rgb}{0.0,0.3921568691730499,0.0}
          \pronec{O}{23}{17}{tempcolor};
          
          \definecolor{tempcolor}{rgb}{0.0,0.3921568691730499,0.0}
          \pronec{O}{24}{13}{tempcolor};
          \definecolor{tempcolor}{rgb}{0.0,0.3921568691730499,0.0}
          \pronec{O}{24}{14}{tempcolor};
          \definecolor{tempcolor}{rgb}{0.0,0.3921568691730499,0.0}
          \pronec{O}{24}{16}{tempcolor};
          \definecolor{tempcolor}{rgb}{0.0,0.3921568691730499,0.0}
          \pronec{O}{24}{17}{tempcolor};
          
          \definecolor{tempcolor}{rgb}{0.0,0.3921568691730499,0.0}
          \pronec{O}{25}{13}{tempcolor};
          \definecolor{tempcolor}{rgb}{0.0,0.3921568691730499,0.0}
          \pronec{O}{25}{14}{tempcolor};
          \definecolor{tempcolor}{rgb}{0.0,0.3921568691730499,0.0}
          \pronec{O}{25}{16}{tempcolor};
          \definecolor{tempcolor}{rgb}{0.0,0.3921568691730499,0.0}
          \pronec{O}{25}{17}{tempcolor};
          
          \definecolor{tempcolor}{rgb}{0.0,0.3921568691730499,0.0}
          \pronec{O}{26}{13}{tempcolor};
          \definecolor{tempcolor}{rgb}{0.0,0.3921568691730499,0.0}
          \pronec{O}{26}{14}{tempcolor};
          \definecolor{tempcolor}{rgb}{0.0,0.3921568691730499,0.0}
          \pronec{O}{26}{16}{tempcolor};
          \definecolor{tempcolor}{rgb}{0.0,0.3921568691730499,0.0}
          \pronec{O}{26}{17}{tempcolor};
          
          \draw ($(O)+(0,2.25)$) node[anchor=east] {$5$};
          \draw ($(O)+(0,4.75)$) node[anchor=east] {$10$};
          \draw ($(O)+(0,7.25)$) node[anchor=east] {$15$};
          \draw ($(O)+(0,9.75)$) node[anchor=east] {$20$};
          \draw ($(O)+(0,12.25)$) node[anchor=east] {$25$};
          \draw ($(O)+(2.25,-0.3)$) node {$5$};
          \draw ($(O)+(4.75,-0.3)$) node {$10$};
          \draw ($(O)+(7.25,-0.3)$) node {$15$};
          \draw ($(O)+(9.75,-0.3)$) node {$20$};
          \draw ($(O)+(12.25,-0.3)$) node {$25$};

    \end{tikzpicture}
    \caption{A solution of the instance given in Figure~\ref{fig:neighborizationExample} that is returned by the neighborization algorithm. The density of this solution is $O(\sqrt{n})$.}
    \label{fig:neighborizationExampleBad}
\end{figure}


\section{Discussion}

We highly believe this instance is the worst case that can happen and that the density of a maximal solution is always higher than $4\sqrt{n}$. The result of Theorem~\ref{theo:sqrtnapprox} may then possibly be updated to the following conjecture. 

\begin{conjecture}
	An algorithm returning any maximal solution of an instance of MMC is a $\sqrt{n}$-approximation.
\end{conjecture}