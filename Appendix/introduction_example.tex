\begin{figure}[!ht]
	\center
	\begin{tikzpicture}
		\coordinate (O) at (0,0);
		
		\clip (-1,-1.25) rectangle (2.5,2.1);
		
		\prgrid{O}{4}{4}
		
		\prvtdline{O}{1}{4};
		
		\prvtdcolumn{O}{1}{4};
		\prvtdcolumn{O}{2}{4};
		\prvtdcolumn{O}{3}{4};
		
		\prone{O}{1}{2}
		\prone{O}{1}{4}
		\prone{O}{2}{3}
		\prone{O}{3}{1}
		\prone{O}{3}{3}
		\prone{O}{4}{1}
		
		
		\draw ($(O)+(-0,0.25)$) node[anchor=east] {$4$};
		\draw ($(O)+(-0,0.75)$) node[anchor=east] {$3$};
		\draw ($(O)+(-0,1.25)$) node[anchor=east] {$2$};
		\draw ($(O)+(-0,1.75)$) node[anchor=east] {$1$};
		
		\draw ($(O)+(0.25,-0.3)$) node {$1$};
		\draw ($(O)+(0.75,-0.3)$) node {$2$};
		\draw ($(O)+(1.25,-0.3)$) node {$3$};
		\draw ($(O)+(1.75,-0.3)$) node {$4$};
		
		\draw ($(O)+(1,-0.8)$) node {$(a)$};
		
	\end{tikzpicture}
	\begin{tikzpicture}
	\coordinate (O) at (0,0);
	
	\clip (-1,-1.25) rectangle (2.5,2.1);
	
	\prgrid{O}{3}{4}
	
	\prone{O}{1}{2}
	\prone{O}{1}{3}
	\prone{O}{1}{4}
	\prone{O}{2}{1}
	\prone{O}{2}{3}
	\prone{O}{3}{1}
		
	\prvtdcolumn{O}{1}{3};
	
	
	\draw ($(O)+(-0,0.25)$) node[anchor=east] {$3/4$};
	\draw ($(O)+(-0,0.75)$) node[anchor=east] {$2$};
	\draw ($(O)+(-0,1.25)$) node[anchor=east] {$1$};
	
	\draw ($(O)+(0.25,-0.3)$) node {$1$};
	\draw ($(O)+(0.75,-0.3)$) node {$2$};
	\draw ($(O)+(1.25,-0.3)$) node {$3$};
	\draw ($(O)+(1.75,-0.3)$) node {$4$};
	\draw ($(O)+(1,-0.8)$) node {$(b)$};
	\end{tikzpicture}
	\begin{tikzpicture}
	\coordinate (O) at (0,0);
	
	\clip (-1,-1.25) rectangle (2.5,2.1);
	
	\prgrid{O}{3}{3}
	
	\prone{O}{1}{1}
	\prone{O}{1}{2}
	\prone{O}{1}{3}
	\prone{O}{2}{1}
	\prone{O}{2}{2}
	\prone{O}{3}{1}
	
	
	\draw ($(O)+(-0,0.25)$) node[anchor=east] {$3/4$};
	\draw ($(O)+(-0,0.75)$) node[anchor=east] {$2$};
	\draw ($(O)+(-0,1.25)$) node[anchor=east] {$1$};
	
	\draw ($(O)+(0.25,-0.3)$) node {$1/2$};
	\draw ($(O)+(0.75,-0.3)$) node {$3$};
	\draw ($(O)+(1.25,-0.3)$) node {$4$};
	\draw ($(O)+(0.75,-0.8)$) node {$(c)$};
	
	\end{tikzpicture}
	\caption{In Figure~\ref{fig:introduction:example}.a, we give a $4 \times 4$ matrix containing 6 ones. The entries containing a zero are not written for readability. Valid contractions are represented by dotted lines and columns. It is not allowed to contract lines 1 and 2 because the two ones (1;1) and (2;1) would be brought into the same entry. Figure~\ref{fig:introduction:example}.b is the result of the contraction of lines 3 and 4 and Figure~\ref{fig:introduction:example}.c is the contraction of columns 1 and 2. The density (the number of neighbor pairs) of each matrix is respectively 4, 7 and 10.
  }
	\label{fig:introduction:example}
\end{figure}
