\begin{figure}[!ht]
	\center
	\begin{tikzpicture}
		\coordinate (O) at (0,0);
		
		\clip (-1,-1.25) rectangle (2.5,2.1);
		
		\prgrid{O}{3}{3}
		
		\prvtdline{O}{1}{3};
		
		\prvtdcolumn{O}{1}{3};
		\prvtdcolumn{O}{2}{3};
		
		\prbul{O}{1}{2}
		\prbul{O}{2}{3}
		\prbul{O}{3}{1}
		\prbul{O}{3}{3}
		
		
		\draw ($(O)+(-0,0.25)$) node[anchor=east] {$3$};
		\draw ($(O)+(-0,0.75)$) node[anchor=east] {$2$};
		\draw ($(O)+(-0,1.25)$) node[anchor=east] {$1$};
		
		\draw ($(O)+(0.25,-0.3)$) node {$1$};
		\draw ($(O)+(0.75,-0.3)$) node {$2$};
		\draw ($(O)+(1.25,-0.3)$) node {$3$};
		
		\draw ($(O)+(0.75,-0.8)$) node {$(a)$};
		
	\end{tikzpicture}
	\begin{tikzpicture}
	\coordinate (O) at (0,0);
	
	\clip (-1,-1.25) rectangle (2.5,2.1);
	
	\prgrid{O}{3}{2}
	
	\prbul{O}{1}{1}
	\prbul{O}{2}{2}
	\prbul{O}{3}{1}
	\prbul{O}{3}{2}
	
	
	\prvtdline{O}{1}{2};
	
	
	\draw ($(O)+(-0,0.25)$) node[anchor=east] {$3$};
	\draw ($(O)+(-0,0.75)$) node[anchor=east] {$2$};
	\draw ($(O)+(-0,1.25)$) node[anchor=east] {$1$};
	
	\draw ($(O)+(0.25,-0.3)$) node {$1$};
	\draw ($(O)+(0.75,-0.3)$) node {$2$};
	\draw ($(O)+(0.5,-0.8)$) node {$(b)$};
	\end{tikzpicture}
	\begin{tikzpicture}
	\coordinate (O) at (0,0);
	
	\clip (-1,-1.25) rectangle (2.5,2.1);
	
	\prgrid{O}{2}{2}
	
	\prbul{O}{1}{1}
	\prbul{O}{1}{2}
	\prbul{O}{2}{1}
	\prbul{O}{2}{2}
	
	
	\draw ($(O)+(-0,0.25)$) node[anchor=east] {$2$};
	\draw ($(O)+(-0,0.75)$) node[anchor=east] {$1$};
	
	\draw ($(O)+(0.25,-0.3)$) node {$1$};
	\draw ($(O)+(0.75,-0.3)$) node {$2$};
	\draw ($(O)+(0.5,-0.8)$) node {$(c)$};
	
	\end{tikzpicture}
	\caption{On the left, in Figure~\ref{fig:introduction:example}.a, we give a $3 \times 3$ grid containing four dots. We can contract lines 2 and 3; columns 1 and 2 and columns 2 and 3; but it is not allowed to contract lines 1 and 2 because this would lead to put the two dots of coordinates (1;3) and (2;3) into the same entry (1;3). Valid contractions are represented by dotted lines and columns. Figure~\ref{fig:introduction:example}.b is the result of the contraction of columns 1 and 2 and Figure~\ref{fig:introduction:example}.c is the result of the contraction of lines 2 and 3. }
	\label{fig:introduction:example}
\end{figure}